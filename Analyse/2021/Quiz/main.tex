\documentclass{beamer}
\usetheme{Berkeley}
%Information to be included in the title page:
\usepackage{mathtools}

\DeclarePairedDelimiter\abs{\lvert}{\rvert}%
\DeclarePairedDelimiter\norm{\lVert}{\rVert}%

% Swap the definition of \abs* and \norm*, so that \abs
% and \norm resizes the size of the brackets, and the 
% starred version does not.
\makeatletter
\let\oldabs\abs
\def\abs{\@ifstar{\oldabs}{\oldabs*}}
%
\let\oldnorm\norm
\def\norm{\@ifstar{\oldnorm}{\oldnorm*}}
\makeatother

\title{Quizz d'analyse}
\author{Students 4 Students}
\date{}

\begin{document}

\frame{\titlepage}

\begin{frame}
\frametitle{Suites, fonctions et limites - Q1}
La suite \(a_k\) est convergente et admet pour limite \(l\), s'il existe un \(\epsilon > 0\) et un \(n_0\) tel que pour tout \(n \ge 0\), on a que \(\abs{a_k-l}\le \epsilon   \)
\begin{itemize}
 \item<1-> Vrai
 \item<1-> Faux
  \item<1-> Je ne sais pas
\end{itemize}
\end{frame}

\begin{frame}
\frametitle{Suites, fonctions et limites - Q2}
Soit $g(x)$ telle que $\lim \limits_{x\to 2}g(x)=0$. \\ Soit aussi $f(x)$ une fonction positive telle que $f(x)<g(x) \ \forall x \in [1,3]$. Alors $\lim \limits_{x\to 2} f(x)=0$
\begin{itemize}
 \item<1-> Vrai
 \item<1-> Faux
  \item<1-> Je ne sais pas
\end{itemize}
\end{frame}

\begin{frame}
\frametitle{Suites, fonctions et limites - Q3}
Soit la suite \((x_n)_{n \in \mathbb{N}^*}\) convergente vers 0 et soit la suite \((y_n)>0\) pour tout \(n \in \mathbb{N}^*\). Alors la suite \((x_ny_n)\) converge.
\begin{itemize}
 \item<1-> Vrai
 \item<1-> Faux
 \item<1-> Je ne sais pas
\end{itemize}
\end{frame}


\begin{frame}
\frametitle{Suites, fonctions et limites - Bilan}
Vous voyez, c'était pas si dur !
\end{frame}

\begin{frame}
\frametitle{Continuité - Q1}
La fonction \(f : E \to F\) est continue en le point \(x_0\) si $\forall \epsilon > 0, \exists \delta > 0$ tel que $ \forall x  \in \{x : 0< \abs{x - x_0} \le \delta\}$, on a $ \abs{f(x)-f(x_0)} \le \epsilon$
\begin{itemize}
 \item<1-> Vrai
 \item<1-> Faux
 \item<1-> Je ne sais pas
\end{itemize}
\end{frame}

\begin{frame}
\frametitle{Continuité - Q2}
Soit \(f : E \to \mathbb{R}\) et \(g : E \to \mathbb{R}\) deux fonctions continues en \(x_0\). Alors \(\cfrac{f(x)}{g(x)}\) est continue en \(x_0\).
\begin{itemize}
 \item<1-> Vrai
 \item<1-> Faux
 \item<1-> Je ne sais pas
\end{itemize}
\end{frame}

\begin{frame}
\frametitle{Continuité - Q3}
Soit \(f : [1,2] \to \mathbb{R}\) telle que \(f([1,2])=]1,2[\). Alors la fonction \(f\) n'est pas continue sur \([1,2]\).
\begin{itemize}
 \item<1-> Vrai
 \item<1-> Faux
 \item<1-> Je ne sais pas
\end{itemize}
\end{frame}


\begin{frame}
\frametitle{Continuité - Bilan}
On est déjà à la moitié du programme !
\end{frame}


\begin{frame}
\frametitle{Séries - Q1}
Soit la série convergente \(\displaystyle \sum_{k=1}^\infty \abs{\cfrac{(-1)^k}{k^{3\pi k}}}\). \\ Alors la série \(\displaystyle \sum_{k=1}^\infty \cfrac{(-1)^k}{k^{3\pi k}}\) converge.
\begin{itemize}
 \item<1-> Vrai
 \item<1-> Faux
 \item<1-> Je ne sais pas
\end{itemize}
\end{frame}

\begin{frame}
\frametitle{Séries - Q2}
Soit la série \(\displaystyle \sum_{k=0}^\infty a_k\) telle que \(\lim \limits_{k \to \infty} a_k = 0\). \\ Alors la série est convergente.
\begin{itemize}
 \item<1-> Vrai
 \item<1-> Faux
 \item<1-> Je ne sais pas
\end{itemize}
\end{frame}

\begin{frame}
\frametitle{Séries - Q3}
Soit une série \(\displaystyle \sum_{k=1}^\infty a_k\) divergente et soit une suite \((b_k)\) telle que \(\lim \limits_{k \to \infty} b_k = 0\), alors, la série  \(\displaystyle \sum_{k=1}^\infty a_kb_k\) converge.
\begin{itemize}
 \item<1-> Vrai
 \item<1-> Faux
 \item<1-> Je ne sais pas
\end{itemize}
\end{frame}


\begin{frame}
\frametitle{Séries - Bilan}
Il ne reste déjà plus qu'un seul chapitre ?
\end{frame}

\begin{frame}
\frametitle{Nombres complexes - Q1}
Soit \(z = 3\left(\cfrac{1+i \sqrt 3}{2}\right)\). Alors on a aussi que \(z = 3e^{\frac{i \sqrt 3}{2}}\)
\begin{itemize}
 \item<1-> Vrai
 \item<1-> Faux
 \item<1-> Je ne sais pas
\end{itemize}
\end{frame}

\begin{frame}
\frametitle{Nombres complexes - Q2}
Soit \(z = 3(\cos{\frac{\pi}{12}}+i \sin{\frac{\pi}{12}})\). Alors on a que \(z^{24}=3^{24}e^{2\pi}\).
\begin{itemize}
 \item<1-> Vrai
 \item<1-> Faux
 \item<1-> Je ne sais pas
\end{itemize}
\end{frame}

\begin{frame}
\frametitle{Nombres complexes - Q3}
L`équation \(z^{42}=\frac{106}{8\pi}e^{i\arccos{\frac{\sqrt 3}{2}}}\) a 21 solutions.
\begin{itemize}
 \item<1-> Vrai
 \item<1-> Faux
 \item<1-> Je ne sais pas
\end{itemize}
\end{frame}

\begin{frame}
\frametitle{Nombres complexes - Bilan}
Et voilà, c'en est terminé pour le cours d'analyse !

\end{frame}



\end{document}

