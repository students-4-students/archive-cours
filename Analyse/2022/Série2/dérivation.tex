
\part{Dérivabilité}
\begin{exercice}[Dérivée via la définition]
Dériver les fonctions suivantes à l'aide de la définition, et uniquement de la définition.

On rappelle que la dérivée $f'(x)$ d'une fonction $f : D \to \mathbb{R}$ en $x \in D$ est définie par
\[
f'(x) = \lim_{h \to 0} \frac{f(x + h) - f(x)}{h}
\]
\begin{enumerate}
    \item $f(x) = x^3$
    \item $f(x) = \frac{1}{x}$
    \item $f(x) = |x| = \begin{cases} x, & \text{si } x \geq 0 \\ -x, & \text{si } x < 0 \end{cases}$
\end{enumerate}
\end{exercice}

\vspace{5mm}

\begin{exercice}[Dérivée via les propriétés]
Dériver les fonctions suivantes à l'aide de vos connaissances et des propriétés des dérivés. Pour chacune des fonctions, trouver le plus grand ensemble telle qu'elle soit dérivable sur cet ensemble.

\begin{enumerate}
    \item $f(x) = x^4 + \cos(x^2)$
    \item $f(x) = \tan(x) = \frac{\sin(x)}{\cos(x)}$
    \item $f(x) = \exp(4 (\cos(x))^2)$
\end{enumerate}
\end{exercice}

\begin{exercice}[Extremums]
Trouver les maximums/minimums des fonctions suivantes~:
\begin{enumerate}
    \item $f(x) = x^2$
    \item $f(x) = \exp \left( -(\frac{1}{2}x - 5)^2 \right)$
    \item $\!\!\!\!{^*}$ $f(x) = x^4 + \cos(x^2)$
    
    \emph{Hint~: Vous pouvez utiliser le fait que $\sin(x) < x$ pour tout $x > 0$}
\end{enumerate}
\end{exercice}