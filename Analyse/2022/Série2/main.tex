\documentclass[11.5pt,french,table]{article}
\usepackage[french]{babel}
\usepackage[margin=1in,a4paper]{geometry}

% Custom fonts. This package is only available with XeLaTex (pdflatex is a mess to deal with)
\usepackage{fontspec}
\setmainfont{GeneralSans}[
    Path = assets/fonts/,
    Extension = .otf,
    UprightFont = *-Regular,
    ItalicFont = *-Italic,
    BoldFont = *-Bold,
    BoldItalicFont = *-BoldItalic
]

% Custom titling
\usepackage{titling}
\usepackage{tcolorbox}

% Custom headers
\usepackage{fancyhdr}
\pagestyle{fancy}
\fancyhead[L]{\theauthor}
\fancyhead[C]{\itshape{\thetitle}}
\fancyhead[R]{\thedate}
\setlength{\headheight}{15pt}

% Default mathematical packages
\usepackage{amsmath}
\usepackage{amsfonts}

% Custom commands
\DeclareMathOperator{\sign}{sign}

% Exercises environment and styling
\usepackage{amsthm}
\newtheoremstyle{exercice}%
    {3pt}% Space above
    {3pt}% Space below
    {\large}% Body font
    {}% Indent amount
    {\bfseries}% Theorem head font
    {.}% Punctuation after theorem heading
    {\newline}% Space after theorem heading
    {\thmname{#1}\thmnumber{ #2}\thmnote{: #3}}% Theorem head spec (can be left empty, meaning ‘normal’)
\theoremstyle{exercice}
\newtheorem{exercice}{Exercice}

% Graphics
\usepackage{graphicx}

% Enumerate labels
\usepackage{enumitem}

% Paragraph spacing
\usepackage{parskip}

\pretitle{\begin{center} \LARGE \bfseries}
\title{Analyse -- Série 2}
\posttitle{\par \end{center}}

\renewcommand{\maketitlehookb}{
\begin{center}
\includegraphics[width=2cm]{assets/imgs/S4S_logo.png}
\end{center}
}

\author{Students 4 Students}
\date{Septembre 2022}

% Post date
\renewcommand{\maketitlehookd}{
\begin{center}
\begin{tcolorbox}[boxrule=0pt,frame empty,width=0.8\textwidth]
Cette série d'exercices teste votre compréhension des 2 dernières heures du cours d'Analyse que vous avez suivi à S4S. Pour rappel, il n'est toujours \textbf{pas attendu} de vous que vous finissiez tous les exercices dans les 2 heures qui sont allouées, au contraire. Profitez des 2 heures d'exercices pour poser vos questions aux assistants, qu'il s'agisse d'une question sur les exercices, la matière du cours ou autre.

Les Exercices 9, 10 et 11 sont facultatifs et s'adressent aux plus motivés d'entre vous.
\end{tcolorbox}
\end{center}
}

\begin{document}

\maketitle

{
\let\clearpage\relax % Avoid include page break
\part{Fonctions et continuité}
\begin{exercice}[Limite à droite, limite à gauche]\label{ex:limits}
Commençons par répondre à la question de la continuité~: remarquons que seules les fonction $h$ et $k$ sont définies au point d'étude. Ainsi, les fonctions $f$, $g$, $i$ et $j$ ne sont pas continues par définition~; il nous reste uniquement à déterminer si $h$ et $k$ sont continues.
\begin{enumerate}
    \item Pour $f(x) = \frac{\tan(x)}{x}$, il nous suffit de remarquer par la définition de la tangente que
    \[
    f(x) = \frac{\tan(x)}{x} = \frac{\sin(x)}{x} \frac{1}{\cos(x)}
    \]
    Or, on a vu lors du cours que $\displaystyle\lim_{x \to 0} \frac{\sin(x)}{x} = 1$, et $\displaystyle\lim_{x \to 0} \cos(x) = \cos(0) = 1$ par la continuité du cosinus, donc par les propriétés des limites~:
    \[
    \lim_{x \to 0} f(x) = \lim_{x \to 0} \underbrace{\frac{\sin(x)}{x}}_{\to 1} \underbrace{\frac{1}{\cos(x)}}_{\to 1} = 1
    \]
    Nul besoin pour cet exemple de différencier la limite à droite et la limite à gauche, car toutes les limites existent des deux côtés.
    
    \item Il nous faut être prudent à cause du terme en $\frac{1}{x-2}$. Par le changement de variable $y = x-2$,
    \[
    \frac{x-1}{x-2} = \frac{y+1}{y} = 1 + \frac{1}{y}
    \]
    Ainsi
    \[
    \lim_{x \to 2^-} \frac{x-1}{x-2} = \lim_{y \to 0^-} 1 + \frac{1}{y} = -\infty \quad \textrm{et} \quad \lim_{x \to 2^+} \frac{x-1}{x-2} = \lim_{y \to 0^+} 1 + \frac{1}{y} = +\infty
    \]
    Les deux limites ne sont pas égales, donc $\displaystyle\lim_{x \to 2} g(x)$ n'est pas définie.
    
    \item Ici, pas de piège ! La fonction $h(x) = \frac{x^3}{3} + 3x^2$ est bien continue en $x = 0$, par combinaison de fonctions continues. Ainsi,
    \[
    \lim_{x \to 0} \frac{x^3}{3} + 3x^2 = h(0) = 0
    \]
    
    \item Notons premièrement qu'il est difficile pour cette fonction d'obtenir de l'intuition autour de $0$. En particulier, plusieurs phénomènes sont en conflits (comment $\sin$ réagit avec $\frac{1}{x}$, comment la multiplication du sinus par $x$ agit-elle, etc.) et donc il est primordial d'avoir une idée de la direction dans laquelle aller.
    
    Le réflexe à avoir avec un sinus est de l'encadrer, pour pouvoir potentiellement appliquer le théorème des gendarmes. De $-1 \leq \sin\left(\frac{1}{x}\right) \leq 1$ on obtient autour de $0$ l'inégalité
    \[
    -x \leq x \sin\left(\frac{1}{x}\right) \leq x
    \]
    Les deux fonctions à gauche et à droite de l'inégalité convergent clairement vers $0$ lorsque $x \to 0$, et ainsi par le théorème des gendarmes~:
    \[
    \lim_{x \to 0} x \sin\left(\frac{1}{x}\right) = 0
    \]
    Ce résultat devient plus clair lorsqu'on affiche le graphique de la fonction, ainsi que les droites d'équations $y = x$ et $y = -x$ (en rouge ci-dessous)~:
    \begin{figure}[H]
    \centering
    \begin{tikzpicture}
        \begin{axis}[
            xmin = -1,
            ymin = -1,
            xmax = 1,
            ymax = 1,
            xlabel = $x$,
            ylabel = $y$,
            axis lines = middle,
            grid,
            trig format plots=rad
        ]
        
        \addplot[blue, domain=-1:-0.01,samples=600] {x * sin(1/x)};
        \addplot[blue, domain=0.01:1,samples=600] {x * sin(1/x)};
        \addplot[red, domain=-1:1] {x};
        \addplot[red, domain=-1:1]{-x};
        
        \end{axis}
    \end{tikzpicture}
    \caption{Graphe de $i(x) = x \sin\left(\frac{1}{x}\right)$ autour de $0$}
    \label{fig:i_graph}
\end{figure}
    
    
    \item En suivant l'indice donné, on obtient que
    \[
    \lim_{x \to 3} \frac{e^{x-3}-1}{x-3} = \lim_{h \to 0} \frac{e^h - 1}{h}
    \]
    On reconnaît alors -- d'une manière un peu détournée -- la définition de la \emph{dérivée} de $\exp(x) = e^x$ en $x = 0$, puisque $\exp(0) = 1$. Or $\exp'(x) = \exp(x) = e^x$ donc
    \[
    \lim_{x \to 3} \frac{e^{x-3} - 1}{x-3} = \exp'(0) = e^0 = 1
    \]
    
    \item Observons premièrement que la fonction $k$ est \emph{paire}, c'est-à-dire que le graphe de $k$ est symétrique par rapport à la droite verticale $x = 0$, car $k(-x) = k(x)$. Ainsi, $k$ admet le même comportement de part et d'autre de $0$, et donc les limites à droites et à gauche coïncideront vers une unique limite.
    
    Un détail important dans le calcul de la limite en un point est que ce calcul \emph{ne prend jamais en compte la valeur en ce point}, c'est-à-dire qu'on étudie une limite \emph{épointé}, tout autour de $0$ mais pas en $0$. En effet, on a pu calculer des limites autour d'un point où la fonction n'était même pas définie, ce qui prouve que le point $x = 0$ n'influe pas sur la limite. Peu importe donc la valeur de $k(0)$.
    
    On peut calculer désormais notre limite en utilisant ces deux remarques~:
    \[
    \lim_{x \to 0} k(x) = \lim_{x \to 0^+} \sqrt{|x|} = \lim_{x \to 0^+} \sqrt{x} = 0
    \]
    On peut vérifier de même que $\displaystyle\lim_{x \to 0^-} k(x)$ sera aussi 0. Puisque $k$ est définie en 0, on a $\displaystyle\lim_{x \to 0} k(x) \neq k(0) = 1$, donc $k$ n'est pas continue en $0$.
\end{enumerate}
\end{exercice}

\begin{exercice}[Continuité]\label{ex:weird_function}
Cet exercice a pour but de vous faire manipuler les propriétés du logarithme, en particulier ses limites autour de points bien choisis. Rappelons que
\begin{align*}
\lim_{x \to 0^+} \ln(x) & = -\infty \\
\lim_{x \to 1^-} \ln(x) & = 0^- \\
\lim_{x \to 1^+} \ln(x) & = 0^+
\end{align*}
où l'on note $0^-$ et $0^+$ le côté par lequel $0$ est approché. De plus, $\ln(x)$ n'est pas défini pour $x \leq 0$.

La fonction $u$ est continue pour $x \not\in \{-1, 0, 1\}$, par combinaison de fonctions continues. Il nous reste à calculer les limites aux points problématiques~:
\begin{enumerate}
    \item Autour de $x = 0$, on a
    \[
    \lim_{x \to 0^-} \frac{1}{\ln(|x|)} = \lim_{x \to 0^-} \frac{1}{\ln(\underbrace{-x}_{> 0})} = 0
    \]
    De manière similaire, on obtient
    \[
    \lim_{x \to 0^+} \frac{1}{\ln(|x|)} = \lim_{x \to 0^+} \frac{1}{\ln(x)} = 0
    \]
    Notez qu'on aurait pu utiliser le même argument de parité de la fonction $k$ de l'Exercice \ref{ex:limits}, grâce à la valeur absolue.
    On obtient finalement que
    \[
    \lim_{x \to 0} \frac{1}{\ln(|x|)} = 0 = u(0)
    \]
    Donc $u$ est continue en $0$.
    
    \item Autour de $x = \pm 1$, le comportement sera symétrique, i.e.
    \begin{align*}
    \lim_{x \to -1^-} \frac{1}{\ln(|x|)} & = \lim_{x \to -1^-} \frac{1}{\ln(\underbrace{-x}_{\to 1^+})} = +\infty \\
    \lim_{x \to -1^+} \frac{1}{\ln(|x|)} & = \lim_{x \to -1^+} \frac{1}{\ln(\underbrace{-x}_{\to 1^-})} = -\infty \\
    \end{align*}
    par les propriétés mentionnées en rappel. De la même manière,
    \begin{align*}
    \lim_{x \to 1^+} \frac{1}{\ln(|x|)} = \lim_{x \to 1^+} \frac{1}{\ln(x)} = +\infty \\
    \lim_{x \to 1^-} \frac{1}{\ln(|x|)} = \lim_{x \to 1^-} \frac{1}{\ln(x)} = -\infty
    \end{align*}
    Par la divergence des limites, $u$ n'est pas continue en $\pm 1$.
\end{enumerate}
On conclut alors que $u$ est continue pour tout $x \not\in \{-1, 1\}$. Sur le graphe de $u$ ci-dessous, cela correspond exactement aux points où la fonction a un saut. Notez aussi que les limites qu'on a calculé se retrouvent bien visuellement, autour des points $x = -1, 0$ et $-1$~:
\begin{figure}[H]
    \centering
    \begin{tikzpicture}
        \begin{axis}[
            xmin = -5,
            ymin = -5,
            xmax = 5,
            ymax = 5,
            xlabel = $x$,
            ylabel = $y$,
            axis lines = middle,
            grid
        ]
        
        \addplot[blue, domain=-5:-1.01] {1 / ln(-x)};
        \addplot[blue, domain=-0.99:0.99, samples = 500] {1 / ln(abs(x))};
        \addplot[blue, domain=1.01:5] {1 / ln(x)};
        
        % \addplot[color=blue,fill=blue,only marks,mark=*] coordinates{(-1,0) (0, 0) (1, 0)};
        
        \end{axis}
    \end{tikzpicture}
    \caption{Graphe de $u$}
    \label{fig:u_graph}
\end{figure}
\end{exercice}

\begin{exercice}[Prolongement par continuité]
Observons premièrement que le prolongement par continuité n'est possible que si~:
\begin{enumerate}
    \item La fonction n'est pas définie au point d'étude, et
    \item La limite existe (i.e. n'est pas infinie/indéfinie).
\end{enumerate}
Ainsi, les fonctions $h$ et $k$ sont déjà définies en $x = 0$, donc il ne fait pas de sens de parler de prolongement. On considère alors les limites obtenus dans l'Exercice~\ref{ex:limits}~:
\begin{enumerate}
    \item Pour $f(x) = \frac{\tan(x)}{x}$, la limite en $0$ est $1$, donc on peut la prolonger par continuité, et son prolongement par continuité est
    \[
    \tilde{f}(x) = \begin{cases}
    \frac{\tan(x)}{x} & \textrm{si } x \neq 0 \\
    1 & \textrm{si } x = 0
    \end{cases}
    \]
    \item $g(x) = \frac{x-1}{x-2}$ n'admet pas de limite réelle en $x = 2$, donc $g$ n'est pas prolongeable par continuité.
    \item $i(x) = x \sin\left(\frac{1}{x}\right)$ a pour limite $0$ en $x = 0$, donc on peut prolonger $i$ par continuité~:
    \[
    \tilde{i}(x) = \begin{cases}
    x \sin\left(\frac{1}{x}\right) & \textrm{si } x \neq 0 \\
    0 & \textrm{si } x = 0
    \end{cases}
    \]
    \item $j(x) = \frac{e^{x-3} - 1}{x-3}$ a pour limite $1$ en $x = 3$, donc son prolongement par continuité est~:
    \[
    \tilde{j}(x) = \begin{cases}
    \frac{e^{x-3} - 1}{x-3} & \textrm{si } x \neq 3 \\
    1 & \textrm{si } x = 3
    \end{cases}
    \]
\end{enumerate}
\end{exercice}

\begin{exercice}[Théorème des valeurs intermédiaires]
\enumeratelinefix
\begin{enumerate}
    \item Observons tout d'abord que la fonction $l$ peut se réécrire comme
    \[
    l(x) = x^3 - 3x^2 + 3x - 1 = (x-1)^3
    \]
    $l$ est évidemment continue sur $[0, 3]$, et c'est aussi une fonction croissante (sa dérivée étant $l'(x) = 3(x-1)^2 \geq 0$ pour $x \in [0, 3]$). Ainsi, son minimum est atteint en $x = 0$, avec $f(0) = -1$, et son maximum est atteint en $x = 3$, avec $f(3) = 2^3 = 8$. Par le Théorème des valeurs intermédiaires, son image est donc~:
    \[
    f([0, 3]) = [-1, 8]
    \]
    
    \item Attention, la fonction $m(x) = \frac{1}{x}$ n'est pas continue en $0$, donc en particulier elle n'est pas continue sur $[-1, 1]$ ! Il nous est impossible d'appliquer directement le Théorème des valeurs intermédiaires (abrégé T.V.I), et il nous faut procéder autrement.
    
    Notons $I = [-1, 0[$. $m$ est bien continue sur $I$ -- elle est même continue pour tout intervalle de la forme $[-1, a]$ avec $-1 < a < 0$. Sur ces intervalles, il est possible d'appliquer le T.V.I pour trouver que l'image est~:
    \[
    m([-1, a]) = \left[\frac{1}{a}, -1\right]
    \]
    Intuitivement, si on fait tendre $a$ vers $0^-$, on voit bien alors que 
    \[
    m([-1, 0[) = m(I) = \,]\!-\!\infty, -1]
    \] puisque $\displaystyle\lim_{a \to 0^-} \frac{1}{a} = -\infty$. Pour le prouver plus formellement, il nous suffit de montrer que~:
    \begin{enumerate}
        \item Pour tout $x \in I$, $m(x) \leq -1$, et
        \item Pour tout $y \leq -1$, il existe $x \in I$ tel que $m(x) = y$.
    \end{enumerate}
    
    Prouvons d'abord que $m(x) \leq -1$ pour tout $x \in I$. Il est aisé de voir que $m$ est une fonction décroissante sur $I$, et ainsi
    \[
    -1 \leq x < 0 \implies m(-1) = -1 \geq m(x)
    \]
    ce qui prouve notre inégalité. Pour prouver la surjectivité, observons que
    \[
    m(x) = \frac{1}{x} = y \iff x = \frac{1}{y}
    \]
    Or $y \leq -1 \implies x = \frac{1}{y} \geq -1$ par décroissance de $m$, donc $x \in I$. On a ainsi trouvé $x \in I$ tel que $m(x) = y$, ce qui prouve bien que l'image de $[-1, 0[$ par $m$ est $]\!-\!\infty, -1]$.
    
    Par un raisonnement similaire, on trouve que l'image de $m$ sur l'intervalle $J = ]0, 1]$ est $[1, \infty[$, et ainsi on obtient que
    \[
    m([-1, 1]) = ]\!-\!\infty, -1] \cup [1, \infty[
    \]
    
    \item Notons premièrement que la fonction $n(x) = \tan(x) = \frac{\sin(x)}{\cos(x)}$ n'est pas définie pour $\cos(x) = 0$, i.e. si $x = k\pi + \frac{\pi}{2}$ pour $k \in \mathbb{Z}$. Puisque l'intervalle $I = [-\frac{\pi}{4}, \frac{\pi}{3}]$ ne contient pas de tels points, $n$ est continue sur tout l'intervalle, par quotient de fonctions continues. Ainsi, il nous suffit de trouver le maximum et le minimum de $n$ sur $I$. On peut remarquer que $\tan$ est une fonction croissante, puisque
    \[
    \tan'(x) = \frac{1}{\cos(x)^2} \geq 0
    \]
    pour tout $x \in I$ (cf. Exercice 7), et ainsi le minimum et le maximum sont respectivement
    \[
    n\left(-\frac{\pi}{4}\right) = \frac{-\frac{\sqrt{2}}{2}}{\frac{\sqrt{2}}{2}} = -1 \quad \textrm{et} \quad n\left(\frac{\pi}{3}\right) = \frac{\frac{\sqrt{3}}{2}}{\frac{1}{2}} = \sqrt{3}
    \]
    Ainsi, par le TVI on conclut que
    \[
    n(I) = [-1, \sqrt{3}]
    \]
\end{enumerate}
\end{exercice}

\begin{exercice}[Fonctions continues et nombre de solutions]
Soit $f(x) = x^3 - 2x - 1 - \sin(x)$, pour $x \in [-\pi, \pi]$. Notons que $f$ est continue (par somme de fonctions continues), et que
\[
x^3 - 2x - 1 = \sin(x) \iff f(x) = 0
\]
Ainsi, il nous suffit d'appliquer de trouver $a, b \in [-\pi, \pi]$ tels que $f(a) \leq 0$ et $f(b) \geq 0$ pour pouvoir appliquer le théorème des valeurs intermédiaires et conclure sur l'existence d'au moins une solution.

\smallskip
\noindent Par tâtonnements, on peut trouver les paires suivantes :
\begin{enumerate}
    \item $f(-\pi) = -\pi^3 - 2\pi - 1 < 0$ et $f(-1) = - \sin(-1) = \sin(1) > 0$, donc il existe (au moins) une solution $x_1 \in ]-\pi, -1[$.
    \item $f(0) = - 1 - \sin(0) = -1 < 0$, donc il existe (au moins) une solution $x_2 \in ]-1, 0[$.
    \item $f(\pi) = \pi^3 - 2\pi - 1 > 3^3 - 2\cdot 3 - 1 > 0$, donc il existe (au moins) une solution $x_3 \in ]0, \pi[$.
\end{enumerate}
Ainsi, on a bien au minimum 3 solutions, comme on peut le voir sur le graphique de $f$~:
\begin{figure}[H]
    \centering
    \begin{tikzpicture}
        \begin{axis}[
            xmin = -3.14,
            xmax = 3.14,
            ymin = -5,
            ymax = 5,
            xlabel = $x$,
            ylabel = $y$,
            axis lines = middle,
            grid,
            trig format plots=rad
        ]
        
        \addplot[blue, domain=-3.14:3.14,samples=200] {x * x * x - 2 * x - 1 - sin(x)};
        
        \end{axis}
    \end{tikzpicture}
    \caption{Graphe de $f(x) = x^3 - 2x - 1 - \sin(x)$ sur $[-\pi, \pi]$}
    \label{fig:solutions_graph}
\end{figure}
\end{exercice}
\part{Dérivation}
\begin{exercice}[Dérivée via la définition] 
Dériver les fonctions suivantes à l’aide de la définition (et uniquement de la définition) de la dérivée.

\begin{enumerate}
    \item $f(x) = x^3$
    
    D'après la définition la dérivée de $f(x)$ est
    \begin{align*}
        f'(x) &= \lim_{h \to 0} \frac{f(x + h) - f(x)}{h} \\
              &= \lim_{h \to 0} \frac{(x + h)^3 - x^3}{h} \\
              &= \lim_{h \to 0} \frac{x^3 + 3x^2h + 3xh^2 + h^3 - x^3}{h} \\
              &= \lim_{h \to 0} \frac{h(3x^2 + 3xh)}{h} \\
              &= \lim_{h \to 0} 3x^2 + 3xh \\
              &= 3x^2
    \end{align*}
    De plus, la limite existe bien pour tout $x \in \mathbb{R}$ donc la fonction est dérivable sur $\mathbb{R}$.

    \item $f(x) = \frac{1}{x}$
    
    De la même manière, on commence par la définition~:
    \begin{align*}
        f'(x) &= \lim_{h \to 0} \frac{f(x + h) - \overbrace{f(x)}^{\neq 0}}{h} \\
              &= \lim_{h \to 0} \frac{\frac{1}{x + h} - \frac{1}{x}}{h} \\
              &= \lim_{h \to 0} \frac{\frac{x - x - h}{x (x + h)}}{h} \\
              &= \lim_{h \to 0} - \frac{1}{\underbrace{x^2 + xh}_{\neq 0 \textrm{ si } x \neq 0}} \\
              &= - \frac{1}{x^2}
    \end{align*}
    Notons que pour $x = 0$ la limite ne converge pas, donc elle n'est pas dérivable en $0$. Ainsi le domaine de dérivabilité de $f(x) = \frac{1}{x}$ est $\mathbb{R}^{*}$.
    
    \item $f(x) = |x| = \begin{cases} x, & \text{si } x \geq 0 \\ -x, & \text{si } x < 0 \end{cases}$
    
    Pour cette fonction, il faut distinguer différents cas selon la valeur de $x$~:
    \begin{enumerate}
        \item Cas $x < 0$~:
        
        Notez que $x + h < 0$ pour un $h$ suffisamment petit, et ainsi~:
        \begin{align*}
            f'(x) &= \lim_{h \to 0} \frac{f(\overbrace{x + h}^{< 0}) - f(\overbrace{x}^{< 0})}{h} \\
            &= \lim_{h \to 0} \frac{-(x + h) - (-x)}{h} \\
            &= \lim_{h \to 0} \frac{-h}{h} \\
            &= -1
        \end{align*}
        
        \item Cas $x > 0$~:
        
        Par le même raisonnement, pour $h$ suffisamment petit, $x + h > 0$ et donc~:
        \begin{align*}
            f'(x) &= \lim_{h \to 0} \frac{f(\overbrace{x + h}^{> 0}) - f(\overbrace{x}^{> 0})}{h} \\
            &= \lim_{h \to 0} \frac{(x + h) - x}{h} \\
            &= \lim_{h \to 0} \frac{h}{h} \\
            &= 1
        \end{align*}
        
        \item Cas $x = 0$~:

        Ce cas est très particulier car selon que l'on considère la limite par la gauche ($h < 0$) ou la limite par la droite ($h > 0$) le résultat est $-1$ ou $1$, respectivement. Ainsi la limite ne converge pas à un résultat unique donc la dérivée n'est pas définie en $x = 0$.
    \end{enumerate}
    
    On a ainsi que $f(x)$ est dérivable sur $\mathbb{R}^*$ et que sa dérivée est
    \[
    f'(x) = \begin{cases} 
        -1 & \text{si } x < 0 \\
        1 & \text{si } x > 0
    \end{cases}
    \]
\end{enumerate}
\end{exercice}

\begin{exercice}[Dérivée via les propriétés]
Dériver les fonctions suivantes à l'aide de vos connaissances et des propriétés des dérivées. Pour chacune des fonctions, trouver le plus grand ensemble telle qu'elle soit dérivable sur cet ensemble.

\begin{enumerate}
    \item $f(x) = x^4 + \cos(x^2)$
    
    On utilise le fait que $(x^n)' = n x^{n-1}$, donc en particulier $(x^4)' = 4x^3$ et $(x^2)' = 2x$, et on a aussi que $(\cos(x))' = - \sin(x)$, une dérivée essentielle à connaître par c\oe{}ur.
    
    Ainsi 
    \begin{align*}
        f'(x) &= \left( x^4 + \cos(x^2) \right)' & \\
              &= \left( x^4 \right)' + \left( \cos(x^2) \right)' &\text{en utilisant } (f(x) + g(x))' = f'(x) + g'(x) \\
              &= 4 x^3 - \left( x^2 \right)' \sin(x^2) &\text{en utilisant } (f(g(x)))' = g'(x) f'(g(x)) \\
              &= 4 x^3 - 2x\sin(x^2) &
    \end{align*}
    La fonction est dérivable sur $\mathbb{R}$.
    
    \item $f(x) = \tan(x) = \frac{\sin(x)}{\cos(x)}$
    
    En utilisant la dérivée d'un quotient $\left(\frac{f(x)}{g(x)}\right)' = \frac{f'(x)g(x) - f(x)g'(x)}{g(x)^2}$ et les dérivées des fonctions trigonométriques, on obtient~:
    \begin{align*}
        f'(x) &= \left( \frac{\sin(x)}{\cos(x)} \right)' \\
              &= \frac{(\sin(x))' \cos(x) - \sin(x) (\cos(x))'}{\cos(x)^2} \\
              &= \frac{\cos(x)^2 + \sin(x)^2}{\cos(x)^2} \\
              &= \frac{1}{\cos(x)^2}
    \end{align*}
    où l'on a utilisé pour la dernière égalité l'identité trigonométrique fondamentale $\sin^2(x) + \cos^2(x) = 1$.

    La fonction $\tan(x)$ est dérivable pour tout $x$ de son domaine de définition, c'est-à-dire $x \neq k\pi + \frac{\pi}{2}$ pour $k \in \mathbb{Z}$.
    
    \item $f(x) = \exp(4 (\cos(x))^2)$
    
    Par la propriété de dérivation d'une fonction composée~:
    \begin{align*}
        f'(x) &= \left( \exp(4 (\cos(x))^2) \right)' \\
        &= \left( 4 \cos(x)^2 \right)' \exp(4 (\cos(x))^2) \\
        &= - 8 \cos(x) \sin(x) \exp(4 \cos(x)^2)
    \end{align*}
    
    La fonction est dérivable sur tout son ensemble de définition, i.e. sur $\mathbb{R}$.
\end{enumerate}
\end{exercice}

\begin{exercice}[Extremum]
Trouver les maximums/minimums des fonctions suivantes~:
\begin{enumerate}
    \item $f(x) = x^2$
    
    Pour trouver le(s) maximum(s)/minimum(s) d'une fonction, on étudie le signe de sa dérivée. On sait que $f'(x) = 2x$ et que $f$ est dérivable sur $\mathbb{R}$. Un extremum est caractérisé par la propriété que $f'(x) = 0$ et que sa dérivée change de signe en $x$. Le fait que $f'(x) = 0$ \emph{n'est pas suffisant} seul pour qu'il s'agisse d'un extremum.
    
    On a donc
    \[
    f'(x) = 0 \iff x = 0
    \]
    Ainsi le seul extremum potentiel est en $x=0$. Pour savoir s'il s'agit en effet d'un extremum il faut étudier le signe de la dérivée avant et après $x = 0$. Dans ce cas, on obtient facilement que $f'(x) < 0$ pour $x < 0$ et $f'(x) > 0$ pour $x > 0$, donc $0$ est un minimum de $f$.
    
    \medskip
    \textbf{Pour aller plus loin\textellipsis}
    
    Notez que le changement de signe de la dérivée peut être étudié en\textellipsis dérivant la fonction dérivée, c'est-à-dire en étudiant la dérivée seconde. En effet, un changement de signe implique une croissance ou décroissance de la fonction dérivée, et donc il nous suffit d'obtenir la valeur de la dérivée seconde (si elle existe) pour pouvoir conclure plus facilement.
    
    En d'autre termes :
    \begin{itemize}
        \item Si $f'(x) = 0$ et $f''(x) > 0$ (i.e. $f'$ croissante) alors $f$ admet un minimum en $x$.
        \item Si $f'(x) = 0$ et $f''(x) < 0$ (i.e. $f'$ décroissante) alors $f$ admet un maximum en $x$.
        \item Si $f'(x) = f''(x) = 0$, alors on ne peut rien conclure. Prenez par exemple les fonctions $f(x) = x^4$ et $f(x) = -x^4$ autour de $0$~: elles admettent toutes deux une dérivée seconde nulle en $0$, alors que $0$ est un minimum dans le premier cas, et un maximum dans le second.
    \end{itemize}
    \medskip
    
    \item $f(x) = \exp \left(-\left( \frac{1}{2}x - 5 \right)^2 \right)$
    
    On calcule en premier lieu la dérivée de notre fonction~:
    \begin{align*}
    f'(x) & = -\left[\left(\frac{1}{2} x - 5 \right)^2\right]' \exp{\left(- \left(\frac{1}{2}x - 5 \right)^2 \right)} \\
          & = - 2 \left(\frac{1}{2}x - 5 \right) \cdot \underbrace{\left(\frac{1}{2}x - 5 \right)'}_{=\frac{1}{2}} \exp{\left(- \left(\frac{1}{2}x - 5 \right)^2 \right)} \\
          & = - \left(\frac{1}{2}x - 5 \right) \exp{\left(- \left(\frac{1}{2}x - 5 \right)^2 \right)}
    \end{align*}
    Ainsi les zéros de la dérivée sont
    \begin{align*}
    f'(x) = 0 & \iff  - \left(\frac{1}{2} x - 5 \right) \underbrace{\exp{\left( - \left( -\frac{1}{2}x - 5 \right)^2 \right)}}_{> 0} = 0 \\
        &\iff \frac{1}{2} x - 5 = 0 \\
        &\iff x = 10
    \end{align*}
    Pour savoir s'il s'agit d'un maximum ou d'un minimum, il nous faut étudier plus en détail le signe de la dérivée. En particulier, remarquons que puisque $\exp(x) > 0$ pour tout $x \in \mathbb{R}$,
    \[
    f'(x) \geq 0 \iff -\frac{1}{2}x + 5 \geq 0 \iff x \leq 10
    \]
    Ainsi, la dérivée passe d'un signe positif à négatif, donc $x = 10$ est un maximum (on monte puis on descend).
    
    \item $f(x) = x^4 + \cos(x^2)$
    
    On commence par déterminer les points tels que $f'(x) = 0$, en utilisant par l'Exercice précédent que $f'(x) = 4 x^3 - 2 x \sin(x^2)$~:
    \begin{align*}
    f'(x) = 0 & \iff 4 x^3 - 2 x \sin(x^2) = 0 \\
              & \iff 2x \sin(x^2) = 4 x^3 \\
              & \iff x \sin(x^2) = 2 x^3
    \end{align*}
    $x = 0$ est une solution triviale de l'équation. Pour $x \neq 0$, l'équation est équivalente à
    \begin{align*}
    \sin(x^2) = 2x^2
    \end{align*}
    Or $\sin(x^2) < x^2 < 2x^2$ pour $x^2 > 0 \iff x \neq 0$, et ainsi $f'(x) \neq 0$ pour tout $x \neq 0$. 
    
    % On peux démontrer que $\sin(y) < 2y$ de la façons suivante : \\
    % On a que $\sin(0) = 0$ et $\sin'(x) = \cos(x) \leq 1$, tandis que $(2x)' = 2$. Ainsi $\forall x > 0 \ : \ (2x)' > \sin'(x)$ donc $2x$ restera \emph{en dessus} de $\sin(x)$.
    
    De plus, 
    \begin{align*}
    f'(x) = 4 x^3 - 2 x \sin(x^2) < 0 & \iff x \left(4x^2 - 2\sin(x^2) \right) < 0 \\
    &\iff 2x \underbrace{\left(2x^2 - \sin(x^2) \right)}_{> 0} < 0 \\
    &\iff 2x < 0
    \end{align*}
    où l'inégalité de la seconde équivalence vient de l'inégalité $\sin(x^2) < 2x^2$ pour $x \neq 0$.
  
    Ainsi, la dérivée change d'un signe négatif à positif de part et d'autre de $0$, donc $x = 0$ est un minimum de $f$.
\end{enumerate}

\end{exercice}
\part{Pour aller plus loin\textellipsis}

\begin{exercice}[Continuité (suite)]
La fonction $u : \mathbb{R} \to \mathbb{R}$ de l'Exercice \ref{ex:weird_function} admet un prolongement par continuité en $x = 0$, car la limite existe et vaut $0$. Cependant, elle ne peut pas être prolongée sur tout $\mathbb{R}$ en une fonction $\tilde{u}$ continue, car la limite diverge en $x = \pm 1$. Ainsi, un tel $\tilde{u}$ n'existe pas.
\end{exercice}

\begin{exercice}[Rencontres du troisième type (d'extremum)]
Soit $f : \mathbb{R} \to \mathbb{R}$ définie par $f(x) = \sin(x)^3$.

\begin{enumerate}
    \item Calculer $f'(x)$.
    
    En utilisant les propriétés de la dérivée on obtient :
    \begin{align*}
        f'(x) &= \left( \sin(x)^3 \right)' & \textrm{ de la forme } \big( f(g(x)) \big)' = g'(x) f'(g(x)) \\
        &= \cos(x) \left( 3 \sin(x)^2 \right)
    \end{align*}
    Donc $f'(x) = 3 \sin(x)^2 \cos(x)$
    
    \item Trouver le(s) point(s) telle que $f'(x) = 0$.
    \begin{align*}
        f'(x) = 0 & \iff 3 \sin(x)^2 \cos(x) = 0 \\
                  & \iff \sin(x) = 0 \textrm{ ou } \cos(x) = 0 \\
                  & \iff x = k \frac{\pi}{2}, \ k \in \mathbb{Z} 
    \end{align*}
    
    \item Parmi ce(s) point(s), trouver les minimums et les maximums.
    
    Pour déterminer s'il s'agit de minimum ou de maximum on étudie le signe de la dérivée. Notons que comme $\sin(x)^2 \geq 0$, le signe de $f'$ dépend uniquement du signe de $\cos(x)$. De plus, comme $f'$ est $2\pi$-périodique, il nous suffit d'étudier son comportement sur l'intervalle $[0, 2\pi]$ pour en déduire son comportement général.
    
    \begin{table}[H]
        \centering
        \begin{tikzpicture}
            \tkzTabInit{$x$/1, $f'(x)$/1}{$0$, $\frac{\pi}{2}$, $\pi$, $\frac{3\pi}{2}$, $2\pi$}
            \tkzTabLine{z, + , z, -, z, -, z, +, z}
        \end{tikzpicture}
        \caption{Tableau de signe de $f'(x) = 3 \sin(x)^2 \cos(x)$}
        \label{tab:signe_derivee}
    \end{table}
    
    On distingue alors les 3 cas de zéros~:
    \begin{itemize}
        \item \emph{Cas n°1}~: Extremum en $x = \frac{\pi}{2} + 2k\pi$
        
        On peux voir sur le Tableau $\ref{tab:signe_derivee}$ qu'au alentour de $\frac{\pi}{2} + 2k\pi$, la dérivée change de signe, passant de positif à négatif. Ainsi les extremums de la forme $x = \frac{\pi}{2} + 2k\pi$ sont des maximums.
        
        \item \emph{Cas n°2}~: Extremum en $x = \frac{3\pi}{2} + 2k\pi$
        
        De même, on peux voir qu'au alentour de $\frac{3\pi}{2} + 2k\pi$, la dérivée passe d'un signe négatif à positif. Ainsi, les extremums de la forme $x = \frac{3\pi}{2} + 2k\pi$ sont des minimums.

        \item \emph{Cas n°3}~: Extremum à $x = k \pi$

        Étrangement, le signe au alentours de $x = k \pi$ semble ne pas changer. Ainsi, bien que la dérivé soit nulle, \emph{il ne s'agit ni d'un maximum, ni d'un minimum} !
    \end{itemize}
    
    \item
    Soit $\sign : \mathbb{R} \to \mathbb{R}$ telle que
    \[
    \sign(x) = \begin{cases}
    1 & \textrm{si } x > 0 \\
    0 & \textrm{si } x = 0 \\
    -1 & \textrm{si } x < 0
    \end{cases}
    \]
    
    Évaluer $\displaystyle \lim_{x \to 0^+} \sign(f'(x))$ et $\displaystyle \lim_{x \to 0^-} \sign(f'(x))$.
    
    \medskip
    
    On a par le tableau ci-dessus que
    \[
    \lim_{x \to 0^+} \sign(f'(x)) = 1 \quad \textrm{et} \quad \lim_{x \to 0^-} \sign(f'(x)) = \lim_{x \to 2\pi^-} \sign(f'(x)) = 1
    \]
    
    \item On peux voir que la fonction $f'(x)$ est positive au alentour de $x = 0$ (comme le démontre le point 4). Bien que $f'(0) = 0$, la dérivée reste positive donc il ne s'agit donc pas d'un minimum où d'un maximum. Peut-on se représenter quel est le problème ici ?
    
    La dérivée est une mesure de la croissance~: comme $f'(0) = 0$, au point $0$ la courbe ne monte ni ne descend. Pourtant, aux alentours de ce point la dérivé est positive et donc la fonction augmente. Ainsi la fonction admet une sorte de "plateau" en $x = 0$, car elle arrête momentanément de croître, ce qui pourrait nous tromper à penser que c'est un extremum si l'on oublie de vérifier que la dérivée change bien de signe.
    
    \begin{figure}[H]
        \centering
        \begin{tikzpicture}
            \begin{axis}[
                xmin = -3.2, xmax = 3.2,
                ymin = -1.1, ymax = 1.1,
                xtick distance = 1,
                ytick distance = 0.25,
                grid = both,
                minor tick num = 1,
                major grid style = {lightgray},
                minor grid style = {lightgray!25},
                width = 440,
                height = 300,
                legend cell align = {left},
                legend pos = north west,
                axis lines=middle,
                trig format plots=rad
            ]
                
            \addplot[
                domain = -3.2:3.2,
                samples = 300,
                smooth,
                thick,
                blue,
            ] {sin(x)^3};
                
            \end{axis}
        \end{tikzpicture}
        \caption{Graphe de $f(x) = \sin(x)^3$ sur $[-\pi, \pi]$}
        \label{fig:sin_cubed}
    \end{figure}
\end{enumerate}
\end{exercice}

\begin{exercice}[Tout mettre ensemble]
Pour quelles valeurs de $a, b \in \mathbb{R}$ la fonction $f : \mathbb{R} \to \mathbb{R}$ définie par
\[
f(x) = \begin{cases}
ax + 5 & \textrm{si } x \leq 2 \\
\sqrt{bx^2 + 1} & \textrm{si } x > 2
\end{cases}
\]
est-elle dérivable en $2$ ?

Premièrement, pour que $f$ soit dérivable en $2$, il faut nécessairement que $f$ soit continue en $2$ (cf. cours). Ainsi, on commence par étudier les limites à droite et à gauche~:
\[
\lim_{x \to 2^-} ax = 2a + 5 \quad \textrm{et} \quad \lim_{x \to 2^+} \sqrt{bx^2 + 1} = \sqrt{4b + 1}
\]
Pour que $f$ soit continue, l'équation $2a + 5 = \sqrt{4b + 1} = f(2)$ doit être vérifiée, ce qui nous donne
\[
(2a + 5)^2 = 4b + 1
\]
De plus, pour que $f$ soit dérivable, on doit avoir
\[
\lim_{h \to 0^-} \frac{f(2+h)-f(2)}{h} = \lim_{h \to 0^+} \frac{f(2+h)-f(2)}{h}
\]
où les deux limites sont obtenues par la définition de $f$~: à gauche, on obtient
\[
\lim_{h \to 0^-} \frac{a(2 + h) + 5 - (2a + 5)}{h} = \lim_{h \to 0^-} \frac{ah}{h} = a
\]
tandis qu'à droite on obtient
\begin{align*}
\lim_{h \to 0^+} \frac{\sqrt{b(2 + h)^2 + 1} - (2a + 5)}{h} & = \lim_{h \to 0^+} \frac{\sqrt{b(2 + h)^2 + 1} - (2a + 5)}{h} \frac{\sqrt{b(2 + h)^2 + 1} + (2a + 5)}{\sqrt{b(2 + h)^2 + 1} + (2a + 5)} \\
& = \lim_{h \to 0^+} \frac{b(2+h)^2 + 1 - (2a + 5)^2}{h (\sqrt{b(2 + h)^2 + 1} + 2a + 5)}
\end{align*}
En subsistuant $(2a + 5)^2 = 4b + 1$, on a
\begin{align*}
\lim_{h \to 0^+} \frac{b(2+h)^2 - 4b}{h (\sqrt{b(2 + h)^2 + 1} + \sqrt{4b + 1})} & = \lim_{h \to 0^+} \frac{b(4 + 4h + h^2) - 4b}{h (\sqrt{b(2 + h)^2 + 1} + \sqrt{4b + 1})} \\
& = \lim_{h \to 0^+} \frac{(4 + h)b}{\sqrt{b(2 + h)^2 + 1} + \sqrt{4b + 1}} \\
& = \frac{4b}{2\sqrt{4b + 1}} \\
& = \frac{2b}{\sqrt{4b + 1}}
\end{align*}
Notez que c'est le même résultat que si l'on avait dérivé $\sqrt{bx^2 + 1}$ par nos propriétés, évalué en $x = 2$. C'est en effet toujours le cas, et on peut se passer du calcul des limites ci-dessus en calculant les dérivées de part et d'autre de $2$, et en les mettant en égalité.
Cela peut accélérer le processus, notamment sous la contrainte du temps en examen.

\medskip
\noindent Le système que l'on doit résoudre est alors
\[
\begin{cases}
2a + 5 & = \sqrt{4b + 1} \\
a & = \frac{2b}{\sqrt{4b + 1}}
\end{cases}
\]
d'où l'on trouve
\[
\frac{4b}{\sqrt{4b + 1}} + 5 = \sqrt{4b + 1}
\]
Cette équation peut être résolue de différentes manières~; en isolant le terme constant, on a
\[
5 = \frac{4b + 1}{\sqrt{4b + 1}} - \frac{4b}{\sqrt{4b + 1}} = \frac{1}{\sqrt{4b + 1}}
\]
d'où l'on obtient $\sqrt{4b + 1} = \frac{1}{5} \implies 4b + 1 = \frac{1}{25}$, i.e. $b = -\frac{6}{25}$. En substituant la valeur de $b$ dans l'équation $a = \frac{2b}{\sqrt{4b + 1}}$, on obtient
\[
a = \frac{-\frac{12}{25}}{\frac{1}{5}} = -\frac{12}{5}
\]
Ainsi, $f$ est dérivable en $2$ si et seulement si $a = -\frac{12}{5}$ et $b = -\frac{6}{25}$. Ces valeurs sont uniques puisque ce sont les seules solutions du système d'équations.
\end{exercice}
}

\end{document}