\part{Pour aller plus loin\textellipsis}

\begin{exercice}[Prolongement par continuité (suite)]
Existe-t-il une fonction $\tilde{u} : \mathbb{R} \to \mathbb{R}$ telle que $\tilde{u}$ coïncide avec les valeurs de la fonction $u$ de l'Exercice \ref{ex:weird_function} sur les points où $u$ est continue, et telle que $\tilde{u}$ soit continue sur $\mathbb{R}$ ?
\end{exercice}

\begin{exercice}[Rencontres du troisième type (d'extremum)]

Soit $f(x) : \mathbb{R} \to \mathbb{R}$ définie par $f(x) = \sin(x)^3$.
\begin{enumerate}
    \item Calculer $f'(x)$.
    \item Trouver le(s) point(s) telle que $f'(x) = 0$.
    \item Parmi ce(s) point(s), trouver les minimums et les maximums.
    \item 
    Soit $\sign : \mathbb{R} \to \mathbb{R}$ la fonction réelle définie par
    \[
    \sign(x) = \begin{cases}
    1 & \textrm{si } x > 0 \\
    0 & \textrm{si } x = 0 \\
    -1 & \textrm{si } x < 0
    \end{cases}
    \]
    
    Évaluer $\displaystyle \lim_{x \to 0^+} \sign(f'(x))$ et $\displaystyle \lim_{x \to 0^-} \sign(f'(x))$.
    
    \item En conclure sur la nature de l'extremum de $f$ en $x = 0$. Est-ce un minimum, un maximum ou autre chose ? Dans le troisième cas, vous pouvez vous aider d'un graphique (e.g. en utilisant Geogebra) pour décrire ce(s) point(s).
\end{enumerate}
\end{exercice}

\begin{exercice}[Tout mettre ensemble (difficile)]
Pour quelles valeurs de $a, b \in \mathbb{R}$ la fonction $f : \mathbb{R} \to \mathbb{R}$ définie par
\[
f(x) = \begin{cases}
ax + 5 & \textrm{si } x \leq 2 \\
\sqrt{bx^2 + 1} & \textrm{si } x > 2
\end{cases}
\]
est-elle dérivable en $2$ ?

\emph{[Cet exercice est nettement plus difficile et plus calculatoire que les autres exercices, donc ne vous découragez pas si vous ne trouvez pas la solution. Étudiez la continuité, puis la dérivabilité à gauche et à droite pour trouver les contraintes sur $a$ et $b$.]}
\end{exercice}