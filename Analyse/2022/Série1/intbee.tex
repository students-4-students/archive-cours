\documentclass[11.5pt,french,table]{article}
\usepackage[french]{babel}
\usepackage[margin=1in,a4paper]{geometry}

% Custom fonts. This package is only available with XeLaTex (pdflatex is a mess to deal with)
\usepackage{fontspec}
\setmainfont{GeneralSans}[
    Path = assets/fonts/,
    Extension = .otf,
    UprightFont = *-Regular,
    ItalicFont = *-Italic,
    BoldFont = *-Bold,
    BoldItalicFont = *-BoldItalic
]

% Custom titling
\usepackage{titling}
\usepackage{tcolorbox}

% Lipsum paragraphs
\usepackage{lipsum}

% Custom headers
\usepackage{fancyhdr}
\pagestyle{fancy}
\fancyhead[L]{\theauthor}
\fancyhead[C]{\itshape{\thetitle}}
\fancyhead[R]{\thedate}
\setlength{\headheight}{15pt}

% Default mathematical packages
\usepackage{amsmath}
\usepackage{amsfonts}

% Custom commands
\newcommand{\enumeratelinefix}{\leavevmode \vspace{-\baselineskip}} % Start enumerate on next line (see amsthm documentation > break theorem style)

% Exercises environment and styling
\usepackage{amsthm}
\newtheoremstyle{exercice}%
    {3pt}% Space above
    {3pt}% Space below
    {\large}% Body font
    {}% Indent amount
    {\bfseries}% Theorem head font
    {.}% Punctuation after theorem heading
    {\newline}% Space after theorem heading
    {\thmname{#1}\thmnumber{ #2}\thmnote{: #3}}% Theorem head spec (can be left empty, meaning ‘normal’)
\theoremstyle{exercice}
\newtheorem{exercice}{Exercice}


\newcommand{\R}{\mathbb{R}}
\newenvironment{packed_enum}{
\begin{enumerate}
  \setlength{\itemsep}{11pt}
  \setlength{\parskip}{0pt}
  \setlength{\parsep}{0pt}
}{\end{enumerate}}
\newcommand{\R}{\mathbb{R}}
\newcommand{\C}{\mathbb{C}}
\DeclareMathOperator{\argcosh}{argcosh}
\DeclareMathOperator{\argsinh}{argsinh}
\DeclareMathOperator{\argtanh}{argtanh}

% Graphics
\usepackage{graphicx}

\pretitle{\begin{center}\LARGE\bfseries}
\title{Integration Bee S4S EPFL Battles}
\posttitle{\par\end{center}}

\renewcommand{\maketitlehookb}{
\begin{center}
\includegraphics[width=2cm]{assets/imgs/S4S_logo.png}
\includegraphics[width=2cm]{assets/imgs/sticker-intbee18.png}
\end{center}
}

\author{Students 4 Students}
\date{22 avril 2023}

% Post date
\renewcommand{\maketitlehookd}{
\begin{center}
\begin{tcolorbox}[boxrule=0pt,frame empty,width=0.8\textwidth]
Ce document regroupe les énoncés et corrigés des intégrales ayant été posées lors des demi-finales et finale de la première édition de Integration Bee.\\

Le format de la demie-finale était :\\
    -4 * 1v1v1v1, un.e seul.e gagnant.e\\
    -15 minutes pour faire 3 intégrales\\
    
Le format de la finale était :\\
    -1v1v1v1\\
    -20 minutes pour faire 4 intégrales
    
\end{tcolorbox}
\end{center}
}


\begin{document}

\maketitle

\section{Enoncés}
\subsection{Demi-Finale (1v1v1v1)}

\subsubsection{Seed 1}
\begin{packed_enum}
    \item $\int_{-125}^{0} \frac{\sqrt[3]{u}+5}{3u^\frac{2}{3}\sqrt{u^\frac{2}{3}+10\sqrt[3]{u}+26}}du $

    \item $\int_0^{\pi/2}\sqrt{\cos(x)\sqrt{\cos(x)\sqrt{\cos(x)...}}}dx$

    \item $\int_0^\frac{\pi}{2}\log(\cos(x))dx$
\end{packed_enum}


\subsubsection{Seed 2}

\begin{packed_enum}
    \item Sachant que $\int_{-\infty}^\infty \frac{1}{\sqrt{2 \pi}} e^{-\frac{1}{2}x^2}dx = 1\\$
    Calculer pour  $x \in \R$ :
    
    $\int_{-\infty}^\infty e^{-\frac{1}{2}(t\sqrt{2}-x)(x+t\sqrt{2})}e^{-tx} dt$

    \item $\int_0^s (\sqrt{x+a} - \sqrt{x})^s dx$

    \item $\int_0^{\pi} (1+2x)\frac{\sin^3(x)}{1+\cos^2(x)}dx$
\end{packed_enum}


\subsubsection{Seed 3}
\begin{packed_enum}
    \item $\int_{0}^{1} \frac{\sqrt{t}}{1+t}dt$

    \item $
    \int_{}^{x} \frac{\sin(55e^{t^{2}})\sin(-11e^{t^{2}})}{\frac{\cos(77e^{t^{2}})}{2\cos(11e^{t^{2}})}} + \frac{\cos(55e^{t^{2}})}{\cos(99e^{t^2})+\sin(99e^{t^2})\tan(22e^{t^2})} dt
    $ 

    \item $
    \int_1^{\infty} \frac{1}{\exp\left(\lfloor \sqrt{x} \rfloor \ln(z)\right)}dx
    $
    
\end{packed_enum}

\subsubsection{Seed 4}
    
\begin{packed_enum}

    \item Calculer, pour $T > 1$ :\\ 
    $
    \sum_{n=1}^\infty \int_{[-nT;nT] \setminus ]-1;1[} \left(\frac{\sin(t)}{t^2} + \frac{\sin^2(t)}{t}\right)^3 dt
    $
        
    \item $
    \int_{}^{x} \frac{\argcosh(t)(1 - t^{2}) + \argtanh(t)(\sqrt{t^2 - 1})}{(t^2 -1)^{\frac{3}{2}}} dt
    $

    \item $\int^y \sqrt{x-\sqrt{x+\sqrt{x-\sqrt{x + ...}}}}dx$
    
\end{packed_enum}


\subsection{Finale(1v1v1v1)}

\begin{packed_enum}
    \item Soit $r \in \mathbb{N}$ tel que $r \geq 2$. Calculer :\\
    $\int_{}^{x}\frac{\log(\prod_{k=2}^{r}\sum_{n=0}^{\infty}\frac{(\sin(2x))^n}{n!})}
    {1+\frac{1}{2}(1+\cos(2x))(1-\sin(x)^2)}dx$

    \item Calculer, pour $x > 0$ et $n \in \mathbb{N}^*$ :\\
    $
    I_n(x) = \int_1^\infty \frac{1}{s^{x+1}}\left(\ln(s)\right)^{2n}ds\\
    $
    En déduire :
    $
    S(x) = \sum_{n=1}^\infty \frac{1}{I_n(x)}
    $

    \item $
    \int_{}^{x} ie^t\frac{1 - e^{8it}}{1 + e^{8it}} \frac{1 + \tan(27t)\tan(23t)}{\tan(27t) - \tan(23t)} dt
    $
\end{packed_enum}

\section{Corrections}


\subsection{Correction des intégrales de la Demi-Finale}

\subsubsection{Seed 1}
\begin{packed_enum}
    \item $\int_{-125}^{0} \frac{\sqrt[3]{u}+5}{3u^\frac{2}{3}\sqrt{u^\frac{2}{3}+10\sqrt[3]{u}+26}}du = \frac{1}{3}\int_{-125}^{0} \frac{u^{\frac{1}{3}}+5}{u^\frac{2}{3}\sqrt{(u^\frac{1}{3}+5)^2 + 1}}du\\$ 
    Le terme commun $u^\frac{1}{3}+5$ est tentant.\\ 
    Changeons de variable: $t = u^\frac{1}{3}+5 \iff u^\frac{1}{3} = t-5 \iff u = (t-5)^3$ donc $du = 3(t-5)^2 dt$, attention aux bornes:
    $$
    I = \frac{1}{3} \int_{0}^{5} \frac{t}{((t-5)^3)^\frac{2}{3}\sqrt{t^2+1}}3(t-5)^2 \overset{\text{tout s'annule}}{=} \int_{0}^5 \frac{t}{\sqrt{t^2 + 1}}dt
    $$
    Nombre de changements de variables marchent ici. Posons $v = t^2 + 1 \iff t = \sqrt{v - 1} \implies dt = \frac{1}{2\sqrt{v - 1}}dv$. Attention aux bornes.
    $
    I = \int_1^{26} \frac{\sqrt{v-1}}{\sqrt{v-1}} \frac{1}{2\sqrt{v}}dv = [\sqrt{v}]_{1}^{26} = \sqrt{26} - 1
    $  

    \item $
\int_0^{\pi/2} \sqrt{\cos(x)\sqrt{\cos(x)\sqrt{\cos(x)...}}}dx\\
    X &= \sqrt{\cos(x)X}\\ 
    \Leftrightarrow X^2 &= \cos(x)X \\ 
    \Leftrightarrow X &= \cos(x)\\ 
    $
    On obtient ainsi : 
    \begin{align*}
    \int_0^{\pi/2} \sqrt{\cos(x)\sqrt{\cos(x)\sqrt{\cos(x)...}}}dx &= \int_0^{\pi/2} \cos(x)dx\\ &= 1
    \end{align*}
    NB: Une démonstration plus rigoureuse est possible via la limite de la suite : 
    $\begin{cases}
    X_0 = 1 \\
    X_n = \sqrt{\cos(x)X_{n-1}}
    \end{cases}$
    
    \item 
    $
    I = \int_0^\frac{\pi}{2}\log(\cos(x))dx  \\
    \int_0^\frac{\pi}{2}\log\left(\cos\left(\frac{\pi}{2}-x\right)\right)dx = 
    \int_0^\frac{\pi}{2}\log(\sin(x))dx\\
    $
    Donc :
    $
    2I = \int_0^\frac{\pi}{2} \log(\cos(x))+\log(\sin(x))dx = \int_0^\frac{\pi}{2} \log\left(\frac{1}{2}\sin(2x)\right)dx =\int_0^\frac{\pi}{2} -\log(2) + \log(\sin(2x))dx \\
    $
    Calculons ensuite :
    \begin{align*}
        \int_0^\frac{\pi}{2} \log(\sin(2x))dx &\overset{u=2x}{=} \int_0^\pi \frac{1}{2}\log(\sin(u))du\\ 
        &= \frac{1}{2}\int_0^{\frac{\pi}{2}} \log(\sin(u))du + \frac{1}{2}\int_{\frac{\pi}{2}}^\pi \log(\sin(u))du\\ 
        &= \frac{1}{2}I + \frac{1}{2}\int_{\frac{\pi}{2}}^\pi \log(\sin(u))du\\ 
        &\overset{s+\frac{\pi}{2}=u}{=} \frac{1}{2}I + \frac{1}{2} \int_0^\frac{\pi}{2} \log\left(\sin\left(s+\frac{\pi}{2}\right)\right)ds\\ 
        &= \frac{1}{2}I + \frac{1}{2} \int_0^\frac{\pi}{2} \log(\cos(s))ds\\
        &= I
    \end{align*}
    Ainsi :
    $
    2I = \frac{\pi}{2}\log\left(\frac{1}{2}\right) + I \iff I = \frac{\pi}{2}\log\left(\frac{1}{2}\right)
    $
\end{packed_enum}


\subsubsection{Seed 2}

\begin{packed_enum}
    \item Notons d'abord que :\\
    $
    -\frac{1}{2}(t\sqrt{2}-x)(x+t\sqrt{2}) -tx = \frac{1}{2}(2t^2 - x^2) - \frac{1}{2}2tx  = -\frac{t^2}{2} - \frac{(x-t)^2}{2}
    $\\
    Il faudra compléter le carré afin de nous ramener à l'intégrale suggérée au début :
    \begin{align*}
        I = \int_{-\infty}^\infty e^{-\frac{t^2}{2} - \frac{(x-t)^2}{2}} dt &= 2\pi \int_{-\infty}^\infty \frac{1}{(\sqrt{2\pi})^2} e^{-\frac{t^2}{2}} e^{ - \frac{(x-t)^2}{2}}dt
    \end{align*}
    A ce stade, l'intégrale pourrait être reconnue comme la convolution des densités de variables aléatoires indépendantes $\mathcal{N}\left(0,1\right)$, donnant une densité de variable $\mathcal{N}\left(0, 2\right)$. C'est la manière rapide de résoudre la question, mais sinon:\\ 
    \begin{align*}
    I &=  2\pi \int_{-\infty}^\infty \frac{1}{(\sqrt{2\pi})^2} e^{-\frac{1}{2} \left(t^2 + (x-t)^2\right)}dt\\ 
    &=  2\pi \int_{-\infty}^\infty \frac{1}{(\sqrt{2\pi})^2} e^{-\frac{1}{2} \left(2t^2 - 2xt + x^2\right)}dt\\ 
    &= 2\pi e^{-\frac{x^2}{2}} \int_{-\infty}^\infty \frac{1}{(\sqrt{2\pi})^2} \exp{\left(-t^2 + 2\frac{x}{2}t -\frac{x^2}{4}+\frac{x^2}{4}\right)}dt \ \text{ (complétion de carré)}\\ 
    &= 2\pi e^{-\frac{x^2}{2}} \int_{-\infty}^\infty \frac{1}{(\sqrt{2\pi})^2} e^{-\left(t-\frac{x}{4}\right)^2 + \frac{x^2}{4}}dt\\ 
    &= 2\pi \frac{1}{\sqrt{2\pi}} e^{-\frac{1}{2}\frac{x^2}{2}} \int_{-\infty}^\infty  \frac{1}{\sqrt{2\pi}} e^{-\left(t-\frac{x}{4}\right)^2}dt\\ 
    &= 2\pi \frac{1}{\sqrt{2}\sqrt{2\pi}} e^{-\frac{1}{2}\frac{x^2}{2}} \underbrace{\int_{-\infty}^\infty  \frac{1}{\frac{1}{\sqrt{2}}\sqrt{2\pi}} e^{-\frac{1}{2\frac{1}{2}}\left(t-\frac{x}{4}\right)^2}dt}_{=1}\\ 
    &= \sqrt{\pi}e^{-\frac{x^2}{4}}
    \end{align*}
    

    \item La première étape consiste en le changement de variable direct :\\
    $
    u = \sqrt{x+a}-\sqrt{x} \implies \frac{du}{dx} = \frac{1}{2}\frac{\sqrt{x+a}-\sqrt{x}}{\sqrt{x+a}\sqrt{x}}
    $\\
    En remarquant que $u^2 = a - 2\sqrt{x+a}\sqrt{x}\iff 2\sqrt{x+a}\sqrt{x} = a - u^2$, \\on obtient :
    $
    u = \sqrt{x+a}-\sqrt{x} \implies \frac{du}{dx} = \frac{u}{a-u^2}
    $
    \begin{align*}
        \int_0^s (\sqrt{x+a} - \sqrt{x})^s dx &\overset{u = \sqrt{x+a}-\sqrt{x}}{=} \int_\sqrt{a}^{\sqrt{s+a}-\sqrt{s}} u^s \frac{a-u^2}{u}du \\ 
        &= \int_\sqrt{a}^{\sqrt{s+a}-\sqrt{s}} au^{s-1} - u^{s+1}du\\ 
        &= \left[a\frac{u^s}{s} - \frac{u^{s+2}}{s+2}\right]_\sqrt{a}^{\sqrt{s+a}-\sqrt{s}}\\ 
        &= a\frac{(\sqrt{s+a}-\sqrt{s})^s - a^{\frac{s}{2}}}{s} - \frac{(\sqrt{s+a}-\sqrt{s})^{s+2} - a^\frac{s+2}{2}}{s+2}
    \end{align*}
    
    \item Il faudra utiliser la propriété suivante, valide pour toute fonction $f$ continue :\\
    $
    \int_0^\pi xf(\sin(x))dx = \frac{\pi}{2}\int_0^\pi f(\sin(x))dx
    $\\
    Puis faire des calculs assez standards, détaillés dans la vidéo suivante : \underline{\href{https://www.youtube.com/watch?v=ExtlUgIxTNY}{lien}}. \\
    Résultat :
    $
    2(1+\pi)\left(\frac{\pi}{2}-1\right)
    $
\end{packed_enum}


\subsubsection{Seed 3}

\begin{packed_enum}
    \item $
    I = \int_{0}^{1} \frac{\sqrt{t}}{1+t}dt = 2\int_{0}^{\sqrt{1}} \frac{u^2}{1+u^2}du
    $ avec $t = u^2 \rightarrow dt = 2udu$. \\Ensuite on ajoute $0 = 1 - 1$:
    $
    2\int_{0}^{\sqrt{1}} \frac{u^2}{1+u^2}du\\ = 2\int_{0}^{\sqrt{1}} \frac{1+u^2}{1+u^2}du - 2\int_{0}^{\sqrt{1}} \frac{1}{1+u^2} du \\= 2\int_{0}^{\sqrt{1}} du - 2[\arctan(u)]_{0}^{\sqrt{1}} \\= 2\sqrt{1}-2\arctan(\sqrt{1})\\
    $ D'où $I = 2 - 2\frac{\pi}{4} = 2 - \frac{\pi}{2}$ 

    \item $
    I = \int_{}^{x} \frac{\sin(55e^{t^{2}})2\cos(11e^{t^{2}})\sin(-11e^{t^{2}})}{\cos(77e^{t^{2}})} + \frac{\cos(55e^{t^{2}})}{\cos(99e^{t^2})\frac{\cos(22e^{t^2})}{\cos(22e^{t^2})}+\sin(99e^{t^2})\frac{\sin(22e^{t^2})}{\cos(22e^{t^2})}} dt\\
    = \int_{}^{x} \frac{\cos(55e^{t^{2}})\cos(22e^{t^{2}}) - \sin(55e^{t^{2}})\sin(22e^{t^{2}})}{\cos(77e^{t^{2}})} dt\\
    = \int^x \frac{\cos(77e^{t^{2}})}{\cos(77e^{t^{2}})}dt\\ 
    = x + C
    $
    
    Nous avons utilisé $\cos(a+b) = \cos(a)\cos(b) - \sin(a)\sin(b)$ ainsi que $\sin(2x) = 2\sin(x)\cos(x)$

    \item Ici il nous faut nous débarasser de la partie entière au plus vite. Observons que lorsque $x = k^2$, on a $\sqrt{x} = k$, et quand $x = (k+1)^2$, on a $\sqrt{x} = k+1$. Ainsi, $\lfloor x \rfloor = k$ pour $x \in [k^2; (k+1)^2[$.
    \begin{align*}
    \int_1^{\infty} z^{- \lfloor \sqrt{x} \rfloor}dx &= \sum_{k=1}^\infty \int_{k^2}^{(k+1)^2} z^{-k}dx\\ 
    &= \sum_{k=1}^\infty \left((k+1)^2 - k^2\right)z^{-k}\\ 
    &= \sum_{k=1}^\infty (2k+1)z^{-k}\\
    &= 2\sum_{k=1}^\infty kz^{-k} + \sum_{k=1}^\infty z^{-k}
    \end{align*}
    Ici nous rappelons que, si $|r| < 1$ :
    $$
    \sum_{k=0}^\infty r^k = \frac{1}{1-r}
    $$
    Ainsi, pour $r = z^{-1}$ :
    $$
    \sum_{k=1}^\infty z^{-k} = -1 + \sum_{k=0}^\infty z^{-k} = \frac{1}{1-z^{-1}} - 1 = \frac{z^{-1}}{1-z^{-1}} 
    $$
    Ensuite, remarquons que, en utilisant les propriétés des dérivées de séries de Taylor d'analyse 1 :
    $$
    \sum_{k=0}^{\infty} kr^{k-1} = \sum_{k=0}^{\infty} \frac{d}{dr}(r^k) = \frac{d}{dr}\left[\sum_{k=0}^\infty r^k\right] = \frac{d}{dr}\left[\frac{1}{1-r}\right] = \frac{1}{(1-r)^2}
    $$
    Appliquons ce résultat à l'étape $(*)$ du calcul suivant :
    $$
    \sum_{k=1}^\infty kz^{-k} = \sum_{k=0}^\infty k(z^{-1})^k = z^{-1} \sum_{k=0}^\infty k(z^{-1})^{k-1} \overset{(*)}{=} \frac{z^{-1}}{(1-z^{-1})^2}
    $$
    Ainsi :
    $
    \int_1^{\infty} z^{- \lfloor \sqrt{x} \rfloor}dx = \frac{2z^{-1}}{(1-z^{-1})^2} + \frac{z^{-1}}{1-z^{-1}} 
    $
\end{packed_enum}


\subsubsection{Seed 4}

\begin{packed_enum}

    \item La somme vaut $0$ ! En effet, pour chaque $n$, l'intégrande est une fonction impaire et continue intégrée sur un domaine symétrique.

    \item  $
    \int_{}^{x} \frac{\argcosh(t)(1 - t^{2}) + \argtanh(t)(\sqrt{t^2 - 1})}{(t^2 -1)^{\frac{3}{2}}} dt\\
    $
    On pose : \\
    $\cosh{u} = t$\\
    $\sinh{u} = dt$\\
    
    $\int_{}^{\argcosh{t}}\frac{(-u\sinh{u}^2 + \argtanh(\cosh{u})\sinh{u})\sinh{u}}{(\sinh(u)^2)^{\frac{3}{2}}}du\\
    =\int_{}^{\argcosh{t}}-u + \frac{\argtanh(\cosh{u})}{\sinh{u}}du\\
    =\int_{}^{\argcosh{t}}-u + \frac{\argtanh(\cosh{u})\sinh{u}}{\sinh{u}^2}du\\
    =\left[ \frac{-u^2}{2} - \frac{\argtanh(\cosh{u})^2}{2}\right]_{}^{\arcsinh{\argcosh{t}}}\\
    = -\frac{\argcosh{t}^2}{2} - \frac{\argtanh(t)^2}{2} + C
    $

    \item $\int^y \sqrt{x-\sqrt{x+\sqrt{x-\sqrt{x + ...}}}}dx
    $\\
    On pose\\ 
    $a= \sqrt{x-\sqrt{x+\sqrt{x-\sqrt{x + ...}}}}$\\
    et $b=\sqrt{x+\sqrt{x-\sqrt{x+\sqrt{x - ...}}}}$\\
    Ainsi, $a^{2} =x-b$, $b^{2}=x+a$ \\donc $a^{2}-b^{2} = -b-a \iff (a-b)(a+b)=-(a+b)\iff b=a+1$, donc $a^{2}=x-(a+1)\iff a^{2}+a+(1-x)=0$.\\
    On obtient :
    $a=\frac{-1+\sqrt{1-4(1-x)}}{2} = \frac{-1+\sqrt{4x-3}}{2}$\\
    Et donc :
    $I = \frac{1}{2} \int -1 + \sqrt{4x-3}dx = \frac{(4y-3)^\frac{3}{2}}{12} - \frac{y}{2} + C$

\end{packed_enum}


\subsection{Correction de la finale}

\begin{packed_enum}
    \item $
    \
    \int_{}^{x}\frac{\log(\prod_{k=2}^{r}e^{\sin(2x)})}
    {1+\frac{1}{2}(1+\cos(2x))(1-\sin(x)^2)}dx \\ 
    &=\int_{}^{x}\frac{\log(e^{\sum_{k=2}^{r}\sin(2x)})}
    {1+\frac{1}{2}(1+\cos(2x))(1-\sin(x)^2)}dx\\ 
    &= \int_{}^{x}\frac{(r-1)\sin(2x)}
    {1+\frac{1}{2}(2\cos(x)^2)(\cos(x)^2)}dx\\ 
    &= \int_{}^{x}\frac{2(r-1)\sin(x)\cos(x)}
    {1+ \cos(x)^4}dx\\ 
    &=(1-r)\arctan\left(\cos(x)^2\right)   
    $

    \item Il faudra utiliser la fonction gamma qui est définie pour $z \in \mathbb{C}$ tel que sa partie réelle est $> 0$ :  
    $
    \Gamma(z) = \int_0^\infty t^{z-1}e^{-t}dt
    $\\\\
    Elle a la propriété suivante : si $m \in \mathbb{N}^*$, alors $\Gamma(m+1) = m!$\\\\
    Nous pouvons alors calculer :
    \begin{align*}
        \int_1^\infty \frac{1}{s^{x+1}}\left(\ln(s)\right)^{2n}ds &= \int_1^\infty \frac{1}{s}\left(\ln(s)\right)^{2n} e^{-x\ln(s)}ds\\ 
        &\overset{t = \ln(s)}{=} \int_0^\infty t^{2n} e^{-xt}dt\\ 
        &\overset{u=xt}{=} \int_0^\infty \frac{u^{(2n+1)-1}}{x^{2n+1}}e^{-u}du\\ 
        &= \frac{\Gamma(2n+1)}{x^{2n+1}}\\
        &= \frac{(2n)!}{x^{2n+1}}
    \end{align*}
    Donc :
    $
    S(x) = x\sum_{n=1}^\infty \frac{x^{2n}}{(2n)!} = x\cosh(x)-x
    $

    \item $\int_{}^{x} ie^t\frac{1 - e^{8it}}{1 + e^{8it}} \frac{1 + \tan(27t)\tan(23t)}{\tan(27t) - \tan(23t)} dt\\
    $\\
    Nous aurons besoin de la formule suivante à l'étape $(*)$ :
    $\\\\
    \tan(a-b) = \frac{\tan(a) - \tan(b)}{1+\tan(a)\tan(b)}
    $
    \begin{align*}
        I &= \int_{}^{x} e^t\frac{i(1 - e^{8it})}{1 + e^{8it}} \frac{1 + \tan(27t)\tan(23t)}{\tan(27t) - \tan(23t)} dt \\ 
        &= \int_{}^{x} e^t\frac{e^{4it} - e^{-4it}}{i(e^{-4it} + e^{4it})} \frac{1 + \tan(27t)\tan(23t)}{\tan(27t) - \tan(23t)} dt \\
        &= \int_{}^{x} e^t\tan(4t)\frac{1 + \tan(27t)\tan(23t)}{\tan(27t) - \tan(23t)} dt \\
        &\overset{(*)}{=} \int_{}^{x} e^t\tan(4t) \frac{1}{\tan(4t)}dt\\
        &= e^x+C
    \end{align*}
    
\end{packed_enum}


\end{document}