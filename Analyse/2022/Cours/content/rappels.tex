\chapter{Rappels et fondamentaux}
% - Rappels sur les ensembles, introduction des ensembles N et R et notation $x \in A$
% - Rappels sur la valeur absolue, distance et inégalité triangulaire

Dans cette section, on introduit quelques notions fondamentales nous seront utiles pour les chapitres suivants.

\section{Ensembles}

\begin{boxdef}[Ensemble]
Un \emph{ensemble} $A$ est une collection non ordonnée d'éléments uniques. On note
\begin{equation}
A = \{x_1, x_2, \ldots, x_n\}
\end{equation}
pour l'ensemble contenant exactement les éléments $x_1$ jusqu'à $x_n$.
\end{boxdef}
Pour le cours, nous n'aurons pas besoin de définir plus rigoureusement ce qu'est un ensemble que la définition ci-dessus. Notez cependant qu'une majeure partie de la théorie mathématique se base sur cette notion fondamentale, et que vous la rencontrerez de manière plus formelle assez rapidement dans votre cursus.

On note $x \in A$ pour l'appartenance de $x$ à l'ensemble $A$. Naturellement, il nous suffit de barrer le symbole, i.e. $x \not\in A$, pour signifier que $x$ n'est pas un élément de $A$.

On définit la notion de \emph{sous-ensemble} de la même manière qu'on définirait une sous-collection~:
\begin{boxdef}[Sous-ensemble]
Un \emph{sous-ensemble} $B$ d'un ensemble $A$ est un ensemble tel que tous les éléments de $B$ sont des éléments de $A$. Symboliquement~:
\begin{equation}
\forall x \in B, x \in A
\end{equation}
On note $B \subseteq A$ pour signifier que B est un sous-ensemble de A.
\end{boxdef}

\subsection{Ensembles usuels}
Certains ensembles seront utilisés fréquemment par la suite, et ils sont tellement utiles en mathématiques qu'ils méritent une notation spéciale à eux seuls.

\begin{greybox}
\textbf{Ensembles usuels.}
\begin{enumerate}
    \item \emph{L'ensemble vide}, i.e. qui ne contient aucun élément, est noté $\emptyset$.
    \item L'ensemble des \emph{entiers non-négatifs} est l'ensemble des nombres positifs auquel on ajoute le nombre $0$\footnote{En mathématiques, on distingue la notion de \emph{nombre positif}, qui ne contient pas $0$, de celle de nombre \emph{non-négatif}, qui elle accepte $0$. Par convention, 0 n'est donc ni négatif, ni positif.}. On le note
    \begin{equation}
    \mathbb{N} = \{0, 1, 2, \ldots\}
    \end{equation}
    
    \item L'ensemble des \emph{entiers relatifs} est l'ensemble des nombres positifs, négatifs ou nuls. On le note
    \begin{equation}
    \mathbb{Z} = \{\ldots, -2, -1, 0, 1, 2, \ldots\}
    \end{equation}
    
    \item L'ensemble des \emph{nombres rationnels} est l'ensemble des fractions, de la forme $\frac{p}{q}$, où $p, q$ sont des entiers relatifs, $q \neq 0$. On le note $\mathbb{Q}$.
    
    \item L'ensemble des \emph{nombres réels} contient l'ensemble des nombres rationnels, ainsi que tous les nombres \emph{irrationels} (c'est-à-dire ne pouvant pas s'écrire comme une fraction rationnelle), tels que $\pi$, $e$, $\sqrt{2}$, etc. On le note $\mathbb{R}$.
    
    \item L'ensemble des \emph{nombres complexes} est l'ensemble des nombres de la forme $a + ib$, où $a, b$ sont des nombres réels, et $i$ est défini par $i = \sqrt{-1}$. On le note $\mathbb{C}$.
\end{enumerate}
\end{greybox}

Notons que rien ne nous a empêché de définir un ensemble infini comme une collection infinie, et qu'on a abusé des points de suspension pour pouvoir définir ces ensembles de manière plus aisée.

\subsection{Intervalles réels}
Les intervalles réels sont des sous-ensembles spéciaux de $\mathbb{R}$, qui sont définies par deux bornes. Ils contiennent tous les nombres entre ces deux bornes, ainsi que potentiellement les bornes elles-mêmes. On note ainsi
\begin{greybox}
\textbf{Notations des intervalles}.
Soit $a < b$ deux nombres réels.
\begin{enumerate}
    \item $[a, b]$ est l'ensemble des nombres réels $x$ tels que $a \leq x \leq b$
    \item $[a, b[$ est l'ensemble des nombres réels $x$ tels que $a \leq x < b$
    \item $]a, b]$ est l'ensemble des nombres réels $x$ tels que $a < x \leq b$
    \item $]a, b[$ est l'ensemble des nombres réels $x$ tels que $a < x < b$
\end{enumerate}
Si l'une des deux bornes est $\pm \infty$, on note alors
\begin{enumerate}
    \item $]\!-\!\infty, b]$ pour l'ensemble des nombres réels $x$ tels que $x \leq b$
    \item $]\!-\!\infty, b[$ pour l'ensemble des nombres réels $x$ tels que $x < b$
    \item $[a, \infty[$ pour l'ensemble des nombres réels $x$ tels que $a \leq x$
    \item $]a, \infty[$ pour l'ensemble des nombres réels $x$ tels que $a < x$
\end{enumerate}
\end{greybox}
Certains intervalles réels ont des notations particulières~:
\begin{greybox}
\begin{enumerate}
    \item $\mathbb{R}^+ = [0, \infty[$ est l'ensemble des nombre réels non-négatifs.
    \item $\mathbb{R}^- = \,]\!-\!\infty, 0]$ est l'ensemble des nombres réels non-positifs.
    \item $\mathbb{R}^* = \,]\!-\!\infty, 0[ \cup ]0, \infty[$  est l'ensemble des nombres réels non-nuls.
\end{enumerate}
\end{greybox}
et similairement $\mathbb{R}^{*+} = \,]0, \infty[$ pour l'ensemble des nombres réels positifs, $\mathbb{R}^{*-} = \,]\!-\!\infty, 0[$ pour l'ensemble des nombres réels négatifs.

\section{Valeur absolue, inégalité triangulaire}
Pour n'importe quel nombre réel $x \in \mathbb{R}$, on définit sa \emph{valeur absolue} comme la valeur non-négative qui est obtenue en enlevant le signe $+$ ou $-$ de son écriture décimale~:
\begin{boxdef}[Valeur absolue]
La \emph{valeur absolue} de $x \in \mathbb{R}$ est la valeur non-négative notée $|x|$ et définie par~:
\begin{equation}
|x| = \begin{cases}
x & \textrm{si } x \geq 0 \\
-x & \textrm{si } x < 0
\end{cases}
\end{equation}
\end{boxdef}
Vue sous un autre angle, la valeur absolue d'un nombre est sa \emph{distance} à 0, si on place le point sur la droite réelle~:
\begin{figure}[H]
    \centering
    \begin{tikzpicture}
    \draw[->] (-2, 0) to (5, 0) node[right] {$\mathbb{R}$};
    \node[circle,fill,inner sep=1pt,label={$0$},draw] (a) at (0,0){};
    \node[circle,fill,inner sep=1pt,label={$x$}, draw] (b) at (3, 0){};
    \draw[decorate, decoration = {calligraphic brace,raise=2pt,amplitude=3pt}] (b.center) -- (a.center) node[pos=0.5,below]{$|x|$};
    \end{tikzpicture}
    \caption{Droite réelle et valeur absolue}
    \label{fig:abs_real_axis}
\end{figure}
Cette notion de distance à 0 est importante, car elle permet de définir la distance \emph{entre deux nombres réels} quelconques $x, y \in \mathbb{R}$ par $d(x, y) = |x - y|$. Puisque $|x-y|$ est la distance de $x - y$ à 0, si $|x-y| = d$ alors $x - y = \pm d$, i.e. $x = y \pm d$, et ainsi $d$ est la distance entre $x$ et $y$.

Une inégalité fondamentale est \emph{l'inégalité triangulaire}~:
\begin{boxdef}[Inégalité triangulaire]
Pour tout $x, y \in \mathbb{R}$~:
\begin{equation}
|x + y| \leq |x| + |y|
\end{equation}
\end{boxdef}
Autrement dit, si l'on additionne deux nombres réels $x, y$, alors la distance à 0 de leur somme ne peut être plus grande que la somme des distances à 0 de $x$ et $y$, prises séparément. On peut se convaincre de la validité de l'inégalité triangulaire en utilisant un graphique similaire à la Figure \ref{fig:abs_real_axis} (essayez par vous-même pour vous convaincre).