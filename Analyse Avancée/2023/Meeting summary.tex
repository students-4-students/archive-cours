  \documentclass[a4paper, 12pt, french, twoside]{article}
\usepackage{graphicx,wrapfig,lipsum}
\usepackage{graphicx}
\usepackage{mathtools}
\usepackage{amsmath}
\usepackage{float}
%\usepackage[top=2.5cm, bottom=2cm, left=2cm, right=2cm, showframe]{geometry}
\usepackage[top=2.5cm, bottom=2cm, left=2cm, right=2cm]{geometry}
\usepackage{caption}
\usepackage{amsmath}
\usepackage{graphicx} % Required for inserting images
\usepackage{subcaption}
\usepackage{hyperref}
\usepackage{makecell}

\usepackage{pdfpages} % lien Pdf test

\usepackage{cleveref}
\crefrangelabelformat{equation}{(#3#1#4--#5#2#6)}
\crefname{equation}{Eq.}{Eqs.}
\Crefname{equation}{Equation}{Equations}


%\usepackage{movie15}
\usepackage{epstopdf}
\usepackage{subcaption}
\usepackage{multicol}

%a abréviations
\def \be {\begin{equation}}
\def \ee {\end{equation}}
\def \dd  {{\rm d}}
\def \bm {\begin{pmatrix}}
\def \em {\end{pmatrix}}






\title{Meeting summary}

\begin{document}

\maketitle  
\section{Global}
\begin{itemize}
    \item 20/07 \begin{itemize}
        \item \textbf{Premier jet le 06/08}
        \item Rencontre, explication du plan
        \item Création des groupes et des sujets. Mise en place de deadlines.
        \item Gaétan écriture volante
        \item Chacun peut amener des sources et idées: le but est de \textbf{s'amuser :)} 
    \end{itemize}
    \item 27/07
        \begin{itemize}
            \item Discussion des éléments à mettre pour chaque équipe
            \item Introduction et acceptation de nouvelles catégories ajoutées
            \item On peut mettre des mots en gras et italique
            \item All good sinon et tout le monde s'amuse!
        \end{itemize}
    \item 10/08
        \begin{itemize}
            \item Finir les parties que l'on a déjà écrites et puis progresser sur les deux autres chapitres et les séries
            \item discussion sur comment apporter du travail de groupe avec des pauses réflexions (question sans maths, sans démo facile si on a bien compris les concepts) qui poussent à communiquer avec les autres.
            
        \end{itemize}
   \item 24/08
    \begin{itemize}
        \item But d'avoir fini le poly dimanche
        \item sabri a des exos qu'il faut juste taper pour toute la série 2 et la fin de la 1 
        \item Mayeul passera sur la série 3 une fois le chap 4 de fini
        \item Faire des box pour les parties avancées
        \item Ouverture en lecture seule dimanche
    \end{itemize}
\end{itemize}



\section{Mayeul \& Iris}
\begin{itemize}
    \item 20/07 S'occupent de: Chapitre 2 Partie 1 
    \item 27/07 Rajouter des exemples et la présentation. Introduire la notion de mesure avec la valeur absolue. Discussion sur quels exemples tirés du poly à ajouter.
    
    \item 10/08 Rajouter les éléments de croissance, décroissance, monotonie des suites. Finir sa partie et ensuite passer sur les séries. Mayeul passe sur les fonctions
\end{itemize}


\section{Romain \& Adam}
\begin{itemize}
    \item 20/07 S'occupent de: Chapitre 2 Partie 2
    \item 27/07 Rajouter lim sup/lim inf. Rajouter des explications topologiques sur la définition d'un espace complet. Rajouter des "Pour le lecteur intéressé" pour ne pas obliger tout lecteur à comprendre ces notions. Préciser un peu le rôle du point d'accumulation (Adam a une idée d'exo challenge).
    
    \item 10/08 Passer sur la série 1. Romain passe aussi sur les séries
\end{itemize}


\section{Marilou}
\begin{itemize}
    \item 20/07 S'occupent de: Chapitre 1
    \item 27/07 Rajouter des "Pour le lecteur intéressé" pour ne pas obliger tout lecteur à comprendre ces notions. 
    
    \item 10/08 Rajouter de l'intuitif sur la notion de densité, peut-être avec un schéma. Passe sur les fonctions aussi
\end{itemize}


\section{Sabri}
\begin{itemize}
    \item 25/07 Commencera plus tard et s'occupera des exercices!
    
    \item 10/08 Bonnes idées pour les exos en utilisant des schémas pour mieux comprendre toutes les defs de convergence.
\end{itemize}



\end{document}