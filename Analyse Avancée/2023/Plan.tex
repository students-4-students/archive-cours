\documentclass{article}
\usepackage{graphicx} % Required for inserting images

\title{Plan}


\begin{document}

\maketitle
% \section{But}
% Ceci est une idée des choses que j'aimerai bien que l'on mette dans le poly. Tout est tiré du polycopié de Genoud, Analyse Avancée 1 pour Physiciens. Le nom des sections indique dans quel chapitre du poly de Genoud j'évoque la référence.\\
% Ce sont bien sur des idées, rien de fixe ! :)\\


\section{Intro}
\begin{itemize}
    \item Première partie sur ce qu'est l'analyse, pourquoi étudier l'analyse ? Lien avec ce que les gens connaissent fonctions, définition de la dérivée qui fait apparaître l'infinitésimal, etc
    \item Différentes techniques de preuve et explication
    \item Vocabulaire mathématique : notation d'un ensemble, des liens entre variables, etc
\end{itemize}


\section{Chapitre 1}

\begin{itemize}   
    \item Différence intuitive entre les ensembles de nombres (entier, rationnel, réel) pour parler de la continuité, de la def de dénombrable. Expliquer la densité de Q dans R etc
    \item Définition de sup et inf 
    % \item Définition 1.3
    % \item Notion d'injective, surjective et bijective (sert à parler des ensembles et pour qu'ils se représentent les ensembles mentalement. Le formalisme est fait en algèbre je crois)
    \item Croissance, décroissance, monotonie d'une fonction 
\end{itemize}


\section{Chapitre 2}
\begin{itemize}
    \item 1.1.4: valeur absolue et propriétés
    \item Introduction au concept d'infinitésimal et de $\varepsilon$
    \item preuve que $a=b$ ssi $|a-b|<\varepsilon$ pour tout epsilon alors  pour introduire le concept de mesure (ou autre preuve ) et donner une utilisation de ce $\varepsilon$
    \item Limites de suites avec propriétées d'addition de suites, théorème des gendarmes et diverses propositions
    \item Limite infinie d'une suite  
    \item concept de sous suite
    \item BW concept mais pas la preuve. Juste comprendre le concept
    \item Suite de Cauchy et les théorèmes qui viennent avec
    
\end{itemize}

\section{Chapitre 3: séries}
\begin{itemize}
    \item Définition d'une série et diverses théorèmes et propositions de base sur les séries + exemples simples
    \item Critéres de convergence : pas celui de Leibniz ni celui de condensation. Ne pas mettre les preuves des critères par soucis de pertinence
    \item Expliquer la permutation de l'ordre des termes pour le fun sans preuve ou quoi que ce soit.
\end{itemize}

\section{Chapitre 4: Fonctions}
\begin{itemize}
    \item Rappel sur ce qu'ils savent de base, définition du domaine de définition d'une fonction, définition de maximum et minimum avec $|x-a|<\delta \implies ...$
    \item Limites d'une fonction. Ne pas mettre la preuve du théorème 4.2.4 et de la limite de l'addition de deux fonctions, du produit du deux fonctions etc
    \item Fonctions continues, définition et exemple de points de discontinuités.
\end{itemize}




\section{Exercices}


\section{Recommandations Team Pédagogique}

\begin{itemize}
    \item Mettre un exo de groupe (entre eux ou un assistant dirige un groupe). 
    \item Mettre des kahoot lorsque l'on présentera le cours
    \item Privilégiez le travail de groupe et la cohésion inter-étudiant (kahoot en groupe pendant le cours et exo en groupe)
\end{itemize}

\end{document}
