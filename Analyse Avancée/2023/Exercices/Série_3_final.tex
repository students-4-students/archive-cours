\documentclass{article}
\usepackage[margin=1in,a4paper]{geometry}
\usepackage[utf8]{inputenc}
\usepackage[cyr]{aeguill}
\usepackage[francais]{babel}
\usepackage{hyperref}
\usepackage{amsmath}
\usepackage{gensymb}
\usepackage{enumitem,amssymb}
\newlist{checks}{itemize}{2}
\setlist[checks]{label=$\square$}
\usepackage{graphicx}
\usepackage{amsthm}
\usepackage{amsfonts}
\usepackage{multicol}
\usepackage{pgfplots}
\pgfplotsset{compat=newest}
\usetikzlibrary{calc}
\usepackage{mathtools}
\usepackage{array}
\usepackage[T1]{fontenc}
\usepackage{lmodern}
\usepackage{tabularx}
\usepackage{fancyhdr}
\usepackage{pst-func}
\usepackage{xcolor}
\usepackage{nicefrac}
\usepackage{mdframed}
\usepackage[boxed,vlined]{algorithm2e}
\usepackage{cleveref}
\newcommand{\Lim}[1]{\raisebox{0.5ex}{\scalebox{1}{$\displaystyle \lim_{#1}\;$}}}
\usepackage{float}
%\usepackage[top=2.5cm, bottom=2cm, left=2cm, right=2cm, showframe]{geometry}
\usepackage[top=2.5cm, bottom=2cm, left=2cm, right=2cm]{geometry}
\newcommand{\N}{\mathbb{N}}
\newcommand{\R}{\mathbb{R}}
\renewcommand{\C}{\mathbb{C}}
\renewcommand{\P}{\mathbb{P}}
\newcommand{\w}{\omega}
\newcommand{\p}{\partial}
\newcommand{\cross}{\times}
\newcommand{\Col}{\text{Col}}
\newcommand{\Tr}{\text{Tr}}
\newcommand{\Cc}{{\mathbb{C}}}
\newcommand{\Ct}{\Cc^\times}
\newcommand{\Hh}{{\mathbb{H}}}
\newcommand{\Nn}{{\mathbb{N}}}
\newcommand{\Zz}{{\mathbb{Z}}}
\newcommand{\Zzn}{\Zz/n\Zz}
\newcommand{\ZzNt}{(\Zz/N\Zz)^\times}%quotient 
\newcommand{\Rr}{{\mathbb{R}}}
\newcommand{\Rt}{{\Rr^\times}}
\newcommand{\Qt}{{\Qq^\times}}
\newcommand{\Qq}{{\mathbb{Q}}}
\newcommand{\bigzero}{\makebox(0,0){\text{\huge0}}}
\DeclareMathOperator{\Ima}{Im}
\DeclareMathOperator{\Vect}{Vect}
\usepackage{mathtools, stmaryrd}
\usepackage{xparse} \DeclarePairedDelimiterX{\Iintv}[1]{\llbracket}{\rrbracket}{\iintvargs{#1}}
\NewDocumentCommand{\iintvargs}{>{\SplitArgument{1}{,}}m}
{\iintvargsaux#1} %
\NewDocumentCommand{\iintvargsaux}{mm} {#1\mkern1.5mu..\mkern1.5mu#2}

% Custom titling
\usepackage{titling}
\usepackage{tcolorbox}

\title{\textbf{Analyse Avancée -- Série 3}}

\renewcommand{\maketitlehookb}{
\begin{center}
\includegraphics[width=2cm]{Cours/assets/imgs/logo-round.png}
\end{center}
}

\author{Students 4 Students}
\date{Septembre 2022}

% Exercises environment and styling
\usepackage{amsthm}
\newtheoremstyle{exercice}%
    {3pt}% Space above
    {3pt}% Space below
    {\large}% Body font
    {}% Indent amount
    {\bfseries}% Theorem head font
    {}% Punctuation after theorem heading
    {\newline}% Space after theorem heading
    {\thmname{#1}\thmnumber{ #2}\thmnote{: #3}}% Theorem head spec (can be left empty, meaning ‘normal’)
\theoremstyle{exercice}
\newtheorem{exercice}{Exercice}


\begin{document}

% Header
\pagestyle{fancy}
\fancyhead[L]{Students 4 Students}
\fancyhead[C]{\textit{Analyse avancée - Série 1}}
\fancyhead[R]{Septembre 2022}

\maketitle


\begin{exercice}[Convergence de séries]
    
Étudier la convergence des séries $\sum_n u_n $ avec : 

\begin{enumerate}
    \begin{multicols}{2}
    \item $u_n= \dfrac{n}{n^3+1}$
    \item $u_n=\dfrac{1}{\sqrt{n}}$
    \item $u_n=\dfrac{x^n}{n!}$, $x\in \Rr$
    \item $u_n=(-1)^n \dfrac{x^{2n}}{(2n)!}$ avec $n!=n\cdot (n-1)\cdot \cdot \cdot 3\cdot 2\cdot 1$
    \item $u_n=\left (\dfrac{n-1}{2n+1}\right )^n$
    \end{multicols}
\end{enumerate}

\end{exercice}

\begin{exercice}[Preuve]
    Montrer la remarque suivante:\\

Pour deux ensemble $B,A$ tel que $B\subset A $ on a $\underset{x\in B}{\sup}f(x)\leq \underset{x\in A}{\sup}f(x)$, $f$ une fonction définie sur $A$.
\end{exercice}


\begin{exercice}[Etude de fonctions]
    Donner le domaine de définition, l'infimum, le suprémum, le maximum et le minimum s'ils existent de chaque fonction suivante:

\begin{enumerate}
    \begin{multicols}{2}
        \item $f(x)=x$
        \item $f(x)=x^2$
        \item $f(x)=\dfrac{1}{x}$
        \item $e^x$
        \item $\dfrac{1}{x\cdot(x-1)\cdot(x-2)\cdot\cdot\cdot (x-n)}$, $n\in \Nn$ fixé. 
    \end{multicols} 
\end{enumerate}
\end{exercice}

\begin{exercice}[Création de fonctions]
    
Donner un exemple de fonction bornée sur $[0,1]$ qui n'atteint ni son infimum ni son supremum.
\end{exercice}

\begin{exercice}[Exercice avancé ]
 Soient $a, b \in \Rr, a < b$, et considérons une fonction continue $f : [a, b] \to [a, b]$. Prouver qu’il existe $c \in[a, b]$ tel que $f(c)=c$. 
    
\end{exercice}

\end{document}