\documentclass[a4paper, 12pt, french, twoside]{article}
\usepackage{graphicx,wrapfig,lipsum}
\usepackage{graphicx}
\usepackage{mathtools}
\usepackage{amsmath}
%\frenchbsetup{StandardLists=true} à inclure si on utilise 
\usepackage[french]{babel}
\usepackage{enumitem}
\usepackage{float}
%\usepackage[top=2.5cm, bottom=2cm, left=2cm, right=2cm, showframe]{geometry}
\usepackage[top=2.5cm, bottom=2cm, left=2cm, right=2cm]{geometry}
\usepackage{caption}
\usepackage{amsmath}
\usepackage{graphicx} % Required for inserting images
\usepackage{subcaption}
\usepackage{hyperref}
\usepackage{makecell}
\usepackage{amsfonts}
\usepackage{amsthm}


\newtheorem{theorem}{Théorème}[section]
\newtheorem{corollary}{Corollaire}[theorem]
\newtheorem{lemma}[theorem]{Lemme}
\newtheorem{proposition}{Proposition}[theorem]
\newtheorem{defi}{Définition}[theorem]
\newtheorem{rem}{Remarque}[theorem]

\usepackage{pdfpages} % lien Pdf test
 \usepackage{tcolorbox}

\usepackage{cleveref}
\crefrangelabelformat{equation}{(#3#1#4--#5#2#6)}
\crefname{equation}{Eq.}{Eqs.}
\Crefname{equation}{Equation}{Equations}


%\usepackage{movie15}
\usepackage{epstopdf}
\usepackage{subcaption}
\usepackage{multicol}

%a abréviations
\def \be {\begin{equation}}
\def \ee {\end{equation}}
\def \dd  {{\rm d}}
\def \bm {\begin{pmatrix}}
\def \em {\end{pmatrix}}

%Ensembles 
\newcommand{\Cc}{{\mathbb{C}}}
\newcommand{\Ct}{\Cc^\times}
\newcommand{\Hh}{{\mathbb{H}}}
\newcommand{\Nn}{{\mathbb{N}}}
\newcommand{\Zz}{{\mathbb{Z}}}
\newcommand{\Zzn}{\Zz/n\Zz}
\newcommand{\ZzNt}{(\Zz/N\Zz)^\times}%quotient 
\newcommand{\Rr}{{\mathbb{R}}}
\newcommand{\Rt}{{\Rr^\times}}
\newcommand{\Qt}{{\Qq^\times}}
\newcommand{\Qq}{{\mathbb{Q}}}
%to do later
\newcommand{\later}[1]{\textcolor{orange}{[#1]}}
\newcommand{\com}[1]{\textcolor{magenta}{[#1]}}

%to make hyper refs not surrounded with red
\hypersetup{pdfborder=0 0 0}
\hypersetup{
    colorlinks=true,
    linkcolor=blue,
    filecolor=magenta,      
    urlcolor=cyan,
    pdftitle={Overleaf Example},
    pdfpagemode=FullScreen,
    }
    
\title{Série 3}

\begin{document}

\maketitle
\section{Convergence de séries}

Étudier la convergence des séries $\sum_n u_n $ avec : 

\begin{enumerate}
    \begin{multicols}{2}
    \item $u_n= \dfrac{n}{n^3+1}$
    \item $u_n=\dfrac{1}{\sqrt{n}}$
    \item $u_n=\dfrac{x^n}{n!}$, $x\in \Rr$
    \item $u_n=(-1)^n \dfrac{x^{2n}}{(2n)!}$ avec $n!=n\cdot (n-1)\cdot \cdot \cdot 3\cdot 2\cdot 1$
    \item $u_n=\left (\dfrac{n-1}{2n+1}\right )^n$
    \end{multicols}
\end{enumerate}

\section{Preuve}
Montrer la remarque suivante:\\

Pour deux ensemble $B,A$ tel que $B\subset A $ on a $\underset{x\in B}{\sup}f(x)\leq \underset{x\in A}{\sup}f(x)$, $f$ une fonction définie sur $A$.

\section{Etude de fonctions}
Donner le domaine de définition, l'infimum, le suprémum, le maximum et le minimum s'ils existent de chaque fonction suivante:

\begin{enumerate}
    \begin{multicols}{2}
        \item $f(x)=x$
        \item $f(x)=x^2$
        \item $f(x)=\dfrac{1}{x}$
        \item $e^x$
        \item $\dfrac{1}{x\cdot(x-1)\cdot(x-2)\cdot\cdot\cdot (x-n)}$, $n\in \Nn$ fixé. 
    \end{multicols} 
\end{enumerate}
\section{Création de fonctions}
Donner un exemple de fonction bornée sur $[0,1]$ qui n'atteint ni son infimum ni son supremum.

\section{Exercice *}
 Soient $a, b \in \Rr, a < b$, et considérons une fonction continue $f : [a, b] \to [a, b]$. Prouver qu’il existe $c \in[a, b]$ tel que $f(c)=c$. 
\end{document}