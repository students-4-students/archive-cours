\documentclass[a4paper, 12pt, french, twoside]{article}
\usepackage{graphicx,wrapfig,lipsum}
\usepackage{graphicx}
\usepackage{mathtools}
\usepackage{amsmath}
%\frenchbsetup{StandardLists=true} à inclure si on utilise 
\usepackage[french]{babel}
\usepackage{enumitem}
\usepackage{float}
%\usepackage[top=2.5cm, bottom=2cm, left=2cm, right=2cm, showframe]{geometry}
\usepackage[top=2.5cm, bottom=2cm, left=2cm, right=2cm]{geometry}
\usepackage{caption}
\usepackage{amsmath}
\usepackage{graphicx} % Required for inserting images
\usepackage{subcaption}
\usepackage{hyperref}
\usepackage{makecell}
\usepackage{amsfonts}
\usepackage{amsthm}


\newtheorem{theorem}{Théorème}[section]
\newtheorem{corollary}{Corollaire}[theorem]
\newtheorem{lemma}[theorem]{Lemme}
\newtheorem{proposition}{Proposition}[theorem]
\newtheorem{defi}{Définition}[theorem]
\newtheorem{rem}{Remarque}[theorem]

\usepackage{pdfpages} % lien Pdf test
 \usepackage{tcolorbox}

\usepackage{cleveref}
\crefrangelabelformat{equation}{(#3#1#4--#5#2#6)}
\crefname{equation}{Eq.}{Eqs.}
\Crefname{equation}{Equation}{Equations}


%\usepackage{movie15}
\usepackage{epstopdf}
\usepackage{subcaption}
\usepackage{multicol}

%a abréviations
\def \be {\begin{equation}}
\def \ee {\end{equation}}
\def \dd  {{\rm d}}
\def \bm {\begin{pmatrix}}
\def \em {\end{pmatrix}}

%Ensembles 
\newcommand{\Cc}{{\mathbb{C}}}
\newcommand{\Ct}{\Cc^\times}
\newcommand{\Hh}{{\mathbb{H}}}
\newcommand{\Nn}{{\mathbb{N}}}
\newcommand{\Zz}{{\mathbb{Z}}}
\newcommand{\Zzn}{\Zz/n\Zz}
\newcommand{\ZzNt}{(\Zz/N\Zz)^\times}%quotient 
\newcommand{\Rr}{{\mathbb{R}}}
\newcommand{\Rt}{{\Rr^\times}}
\newcommand{\Qt}{{\Qq^\times}}
\newcommand{\Qq}{{\mathbb{Q}}}
%to do later
\newcommand{\later}[1]{\textcolor{orange}{[#1]}}
\newcommand{\com}[1]{\textcolor{magenta}{[#1]}}

%to make hyper refs not surrounded with red
\hypersetup{pdfborder=0 0 0}
\hypersetup{
    colorlinks=true,
    linkcolor=blue,
    filecolor=magenta,      
    urlcolor=cyan,
    pdftitle={Overleaf Example},
    pdfpagemode=FullScreen,
    }
    
\title{Corrigé 3}

\begin{document}

\maketitle

\section{Convergence de séries}

\begin{enumerate}
    \item On observe que $0\leq\dfrac{n}{n^3+1}\leq \dfrac{n}{n^3}=\dfrac{1}{n^2} $. En utilisant le critère de comparaison et sachant que l'on a montré dans le cours que la série de terme général $\dfrac{1}{n^2}$ converge, la série est donc convergente.
    \item Il est possible de procéder par comparaison en observant que $\dfrac{1}{n}\leq \dfrac{1}{\sqrt{n}}$. Or nous avons vu dans le cours que la série de terme général $\dfrac{1}{n}$ diverge. Ainsi, la série étudiée diverge aussi.
    \item Pour cette série, la présence d'une factorielle nous indique d'utiliser le critère du quotient. Ainsi:
    \begin{equation}
        \lim_{n\to \infty}\left| \dfrac{u_{n+1}}{u_n}\right| =\lim_{n\to \infty} \left|  \dfrac{x^{n+1}}{(n+1)!}  \dfrac{n!}{x^n}  \right |=\lim_{n\to \infty} \left|  \dfrac{x}{(n+1)} \right |=0.
    \end{equation}
    Cette série converge donc. Par ailleurs, cette série a pour limite exp($x$), la fonction exponentielle.
    
    \item Même méthode, on utilise le critère de D'Alembert. Donc: 
    \begin{equation}
        \lim_{n\to \infty}\left| \dfrac{u_{n+1}}{u_n}\right| =\lim_{n\to \infty} \left|(-1)^{n+1} \dfrac{x^{2(n+1)}}{(2(n+1))!}\dfrac{(2n)!}{(-1)^{n} x^{2n}}\right|
    \end{equation}
    \begin{equation}
        =\lim_{n\to \infty}\left| - \dfrac{x^2}{(2n+2)\cdot (2n+1} \right|=0.
    \end{equation}
    La limite étant 0 $<$ 1, la série converge. Par ailleurs, cette série converge vers $\cos (x)$. On remarque un lien vers la série de l'exponentielle qui sera vu au cours de vos cours d'analyse. 
    \item La puissance en $n$ nous indique que l'usage du critère de Cauchy est le plus censé. Ainsi:

    \begin{equation}
        \lim_{n\to \infty}\left| \left(  \left (\dfrac{n-1}{2n+1}\right )^n\right)^{\frac{1}{n}}  \right |= \lim_{n\to \infty}\left|   \left (\dfrac{n-1}{2n+1}\right )  \right |=\dfrac{1}{2}<1.
    \end{equation}
    La série converge donc bien.
\end{enumerate}

\section{Preuve}
Par définition, on a  $\underset{x\in A}{\sup}f(x)$ comme étant le supremum de l'ensemble $\{f(x) ; x\in A\}$. De plus, en séparant l'ensemble $A$, il est possible de noter que $\underset{x\in A}{\sup}f(x)$ est le suprémum de $\{f(x) ; x\in B\} \cup \{f(x) ; x\in A\backslash B\}$. Ainsi, $\underset{x\in A}{\sup}f(x) = \max \{\underset{x\in B}{\sup}f(x), \underset{x\in A \backslash B}{\sup}f(x) \}$. Ainsi, il devient évident que $\underset{x\in B}{\sup}f(x) \leq \underset{x\in A}{\sup}f(x)$.

\section{Etude de fonctions}
\begin{enumerate}
    \item Cette fonction est définie sur tous l'ensemble des réels. De plus, elle admet comme infimum $-\infty$ et comme supremum $\infty$. Elle n'a donc pas de maximum ou de minimum.
    \item Cette fonction est définie sur tous l'ensemble des réels. De plus, elle admet comme infimum $0$ et comme supremum $\infty$. Elle n'a donc pas de maximum mais a un minimum qui est 0.
    \item Cette fonction est définie sur tous l'ensemble des réels privé de 0. De plus, elle admet comme infimum $-\infty$ et comme supremum $\infty$. Elle n'a donc pas de maximum ou de minimum.
    \item Cette fonction est définie sur tous l'ensemble des réels. De plus, elle admet comme infimum $0$ et comme supremum $\infty$. Elle n'a donc pas de maximum ou de minimum.
    \item Cette fonction est définie sur tous l'ensemble des réels privée des nombres entiers de 0 à $n$. Il s'écrit $\Rr \backslash {0, 1,...,n}$ De plus, elle admet comme infimum $-\infty$ et comme supremum $\infty$. Elle n'a donc pas de maximum ou de minimum.
\end{enumerate}

\section{Création de fonctions}
Il suffit de créer une fonction définie sur un interval ouvert. En effet, considérons la fonction suivante:

\begin{equation}
    f(x)=\begin{cases}
        2x+1, x\in (0,1)\\
        2, x=0 \text{ et } x=1
    \end{cases}.
\end{equation}
Cette fonction admet comme infimum 1 et supremum 3 qui ne sont jamais atteint bien que la fonction soit bornée.


\section{Exercice *}
Posons la fonction $F(x)=f(x)-x$. On a que $f(a), f(b)\in [a,b] $. Ainsi, on voit que:

\begin{equation}
    \begin{cases}
        F(a)= f(a)-a \geq 0\\
        F(b)= f(b)-b\leq 0
    \end{cases}
\end{equation}
La fonction étant continue, il est possible d'utiliser le théorème des valeurs intermédiaires, vu au lycée/gymnase ou qui sera vu plus tard. Par ce théorème, on conclut qu'il existe un point $a\leq c\leq b$ tel que $F(c)=0\iff f(c)=c$.
\end{document}