\documentclass[11pt,french,table]{article}
\usepackage[french]{babel}
\usepackage[margin=1in,a4paper]{geometry}
\usepackage{multicol}

% Custom fonts. This package is only available with XeLaTex (pdflatex is a mess to deal with)
\usepackage{fontspec}
\setmainfont{GeneralSans}[
    Path = assets/fonts/,
    Extension = .otf,
    UprightFont = *-Regular,
    ItalicFont = *-Italic,
    BoldFont = *-Bold,
    BoldItalicFont = *-BoldItalic
]

% Custom titling
\usepackage{titling}
\usepackage{tcolorbox}

% Lipsum paragraphs
\usepackage{lipsum}

% Custom headers
\usepackage{fancyhdr}
\pagestyle{fancy}
\fancyhead[L]{\theauthor}
\fancyhead[C]{\itshape{\thetitle}}
\fancyhead[R]{\thedate}
\setlength{\headheight}{20pt}

% Default mathematical packages
\usepackage{amsmath}
\usepackage{amsfonts}

% Exercises environment and styling
\usepackage{amsthm}
\newtheoremstyle{exercice}%
    {3pt}% Space above
    {3pt}% Space below
    {\large}% Body font
    {}% Indent amount
    {\bfseries}% Theorem head font
    {.}% Punctuation after theorem heading
    {\newline}% Space after theorem heading
    {\thmname{#1}\thmnumber{ #2}\thmnote{: #3}}% Theorem head spec (can be left empty, meaning ‘normal’)
\theoremstyle{exercice}
\newtheorem{exercice}{Exercice}

% Graphics
\usepackage{graphicx}

\pretitle{\begin{center}\LARGE\bfseries}
\title{Analyse Avancée I - Série I}
\posttitle{\par\end{center}}

\renewcommand{\maketitlehookb}{
\begin{center}
\includegraphics[width=2cm]{assets/imgs/S4S_logo.png}
\end{center}
}

\author{Students 4 Students}
\date{Septembre 2022}

\renewcommand{\maketitlehookd}{
\begin{center}
\begin{tcolorbox}[boxrule=0pt,frame empty,width=0.8\textwidth]
Cette première série porte sur les deux premières heures de cours. Aucune calculatrice n'est nécessaire. \\Vous noterez également que l'exercice 5 comporte des $"*"$. Il s'agit d'un exercice plus compliqué que les autres. Vous pouvez donc le faire en fin de série, ou bien tout simplement en discuter entre vous et avec les assistant·es. 
\end{tcolorbox}
\scriptsize{Cette série vous est délivrée par Lounès, Louis et Till.} 
\end{center}
}

\begin{document}

\maketitle
\vspace{1em}
\begin{exercice}[]
 Pour les nombres suivant indiquez s'ils appartiennent à $\mathbb{N}, \mathbb{Z}, \mathbb{Q}$ ou $\mathbb{R}$.
    \begin{multicols}{2}
    \begin{enumerate}
    \item[(a)] $3.3$
        \item[(b)] $\sqrt{3}$
        \item[(c)] $-5$
        \item[(d)] $\pi$
      
    \columnbreak  
     \item[(e)] $e$
        \item[(f)] $3.\Bar{3}$
        \item[(g)] $5.54351684513416757453421345$
        \item[(h)] $-\frac{6415323333}{3}$
        \item[(i)] $\sqrt{5}$
      \columnbreak 
      
     \end{enumerate}
\end{multicols}
\end{exercice}
\vspace{1em}
\begin{exercice}
 Pour les ensembles suivants, donnez le $\sup$ et l'$\inf$. Dire s'ils sont bornés et si oui donner le maximum et le minimum. 
\begin{multicols}{2}
\begin{enumerate}
    \item[A $=$] $\{4,5,9,3,2,1,12,56,4,9\}$
    \item[B $=$] $]5,6]\cup [5,9] \cup ]1,2[$
    \item[C $=$]$\mathbb{R}$
    \item[D $=$]$\{x \in \mathbb{R}^{*}_{+} \mid x<9 \}$
    \columnbreak
    \item[E $=$]$\{x \in \mathbb{R} \mid x<\frac{8}{n}+4 \ \forall n \in \mathbb{N}^*` \}$
    \item[F $=$] \{$r\in \mathbb{Q}$ : $0\leq r \leq \sqrt{2}$\}
    \item[\textbf{Bonus} : G $=$] \{$n^{(-1)^n}$ : $n\in \mathbb{N}$\}
\end{enumerate}
\end{multicols}
\end{exercice}
\vspace{1em}
\begin{exercice}
 Montrer que : 
\begin{align*}
    \lvert x \lvert = \sqrt{x^2}
\end{align*}
Démontrer l'inégalité triangulaire : 
\begin{align*}
    \lvert a+b \lvert \leq \lvert a \lvert +\lvert b\lvert 
\end{align*}
\textit{Indice : Utilisez le carré de la valeur absolue. } \\
\end{exercice}
\vspace{1em}
\begin{exercice}
 Indiquer pour les fonctions suivantes s'il s'agit d'une suite ou non. 
\begin{multicols}{2}
\begin{enumerate}
    \item [(a)] $x_n=n, \ n \in \mathbb{N}$
    \item [(b)] $f(u)=e^u, u \in \mathbb{N}$
    \item [(c)] $f(x)=\sqrt{x}, \ x \in \mathbb{N}$
    \columnbreak
     \item [(d)] $f(x)=\sqrt{x}, \ x \in \mathbb{R}$
      \item [(e)] $S_n=\sum_{k=1}^{n} \frac{1}{k}, \ n,k \in \mathbb{N}$
      \item [(f)] $S=\sum_{n=1}^{\infty} \frac{1}{n^2}, \ n \in \mathbb{N}$
\end{enumerate}
\end{multicols}
\end{exercice}
\vspace{1em}
\begin{exercice}
 * - Montrer que si $r\in \mathbb{Q}$ et $x\in \mathbb{R}_{ / \mathbb{Q}}$ alors $x+r\in \mathbb{R}_{ / \mathbb{Q}}$ et $r\cdot x\in \mathbb{R}_{ / \mathbb{Q}}$.\\
** - Sachant que $\sqrt{2}\notin \mathbb{Q}$, en déduire  qu'entre deux nombres rationnels il y a toujours un nombre irrationnel.
\end{exercice}
\vspace{1em}
\begin{exercice}[Exercices de révisions du gymnase/lycée :]
Résolvez les équations suivantes : 
\begin{align*}
1)& \ \displaystyle\left\lvert\frac{x-4}{x+4} \right\rvert < x-4 \ \ \text{ \textit{Indice : Distinguez les cas}}\\
2)& \ x^2+3x+2=0 \\
3)& \ 2x^2+5x+2=0 \\
4)& \ -x^2-3x-1=0\\
5)& \ \sqrt{6-x}=x
\end{align*}
Montrer les formules suivantes par récurrence : 
\begin{align*}
    \sum_{k=0}^n k &= 1+2+\cdots+n= \frac{n(n+1)}{2}, \ n \in \mathbb{N}\\
    * \sum_{k=0}^n q^k &= \frac{1-q^{n+1}}{1-q}, \ n \in \mathbb{N}, \ q \in \mathbb{R}_{/ \{0,1\}}
\end{align*} 
\paragraph{Rappel : Principe de récurrence}
       On cherche à prouver simultanément un ensemble de
propriétés $P_n$ dépendant d’un entier naturel $n$. On procède de la manière suivante, en respectant ces étapes qui devront être apparentes dans la rédaction d'une récurrence : \\
$\bullet$ Énoncé clair et précis des propriétés $P_n$ et du fait qu’on va réaliser une récurrence. \\
$\bullet$ Initialisation : on vérifie que $P_0$ est vraie (habituellement un calcul très simple). \\
$\bullet$ Hérédité : on suppose $P_n$ vraie pour un entier $n$ quelconque (c’est l’hypothèse de récurrence)
et on prouve $P_{n+1}$ à l’aide de cette hypothèse (si on n’utilise pas l’hypothèse de récurrence, c’est qu’on n’avait pas besoin de faire une récurrence !).\\
$\bullet$ Conclusion : En invoquant le principe de récurrence, on peut affirmer avoir démontré $P_n$ pour tout entier $n$.
\end{exercice}
\end{document}