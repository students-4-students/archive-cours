\documentclass[11pt,french,table]{article}
\usepackage[french]{babel}
\usepackage[margin=1in,a4paper]{geometry}

% Custom fonts. This package is only available with XeLaTex (pdflatex is a mess to deal with)
\usepackage{fontspec}
\setmainfont{GeneralSans}[
    Path = assets/fonts/,
    Extension = .otf,
    UprightFont = *-Regular,
    ItalicFont = *-Italic,
    BoldFont = *-Bold,
    BoldItalicFont = *-BoldItalic
]

% Custom titling
\usepackage{titling}
\usepackage{tcolorbox}

% Lipsum paragraphs
\usepackage{lipsum}

% Custom headers
\usepackage{fancyhdr}
\pagestyle{fancy}
\fancyhead[L]{\theauthor}
\fancyhead[C]{\itshape{\thetitle}}
\fancyhead[R]{\thedate}
\setlength{\headheight}{15pt}

% Default mathematical packages
\usepackage{amsmath}
\usepackage{amsfonts}
\usepackage{multicol}

% Exercises environment and styling
\usepackage{amsthm}
\newtheoremstyle{exercice}%
    {3pt}% Space above
    {3pt}% Space below
    {\large}% Body font
    {}% Indent amount
    {\bfseries}% Theorem head font
    {.}% Punctuation after theorem heading
    {\newline}% Space after theorem heading
    {\thmname{#1}\thmnumber{ #2}\thmnote{: #3}}% Theorem head spec (can be left empty, meaning ‘normal’)
\theoremstyle{exercice}
\newtheorem{exercice}{Exercice}

% Graphics
\usepackage{graphicx}

\pretitle{\begin{center}\LARGE\bfseries}
\title{Analyse Avancée I - Série II}
\posttitle{\par\end{center}}

\renewcommand{\maketitlehookb}{
\begin{center}
\includegraphics[width=2cm]{assets/imgs/S4S_logo.png}
\end{center}
}

\author{Students 4 Students}
\date{Septembre 2022}

\renewcommand{\maketitlehookd}{
\begin{center}
\begin{tcolorbox}[boxrule=0pt,frame empty,width=0.8\textwidth]
Passionné de mathématiques, le pape Pie XII souhaitait qu'on l'appelle Pie puissance 12.
\end{tcolorbox}
\scriptsize{Cette série vous est délivrée par Lounès, Louis et Till.} 
\end{center}
}

\begin{document}

\maketitle



\begin{exercice}[]

Écrivez les cinq premiers termes des suites suivantes. 
\begin{multicols}{2}
  \begin{enumerate}
   
        \item[(a)] $a_n=\frac{1}{3n+1}$
        \item[(b)] $b_n=\frac{3n+1}{4n-1}$
   
    \columnbreak  % uniquement si tu veux changer de colonne à un endroit précis
    \item[(c)] $c_n=\frac{n}{3^n}$
    \item[(d)] $d_n = \sin{\big(\frac{n\pi}{4}\big)}$
     \end{enumerate}
\end{multicols}

\end{exercice}
\vspace{1em}
\begin{exercice}
Pour chaque suite de l'exercice 2.1, dites si elle converge ou pas et trouver la limite. Si vous n'arrivez à trouver la limite formellement, utilisez un calculateur graphique (ex : desmos.com). 
\end{exercice}
\vspace{1em}
\begin{exercice}
   Pour chaque suite ci-dessous, déterminer si elle converge ou non et, si elle converge, donner sa limite. Aucune preuve n'est demandée. Il est possible (et recommandé) d'utiliser un calculateur graphique (www.desmos.com). 
\begin{multicols}{4}
    \begin{enumerate}
   
        \item[(a)] $a_n=\frac{n}{n+1}$
        \item[(b)] $b_n=\frac{n^2+3}{n^2-3}$
   \item[(c)] $c_n=2^{-n}$
   \item[(d)] $d_n=1+\frac{2}{n}$
    \columnbreak  
    \item[(e)] $e_n=2+(-1)^n$
    \item[(f)] $f_n =\frac{(\sqrt[3]{n})^5}{(3\sqrt{n})^4}$
    \item[(g)] $g_n=n!$
    \item[(h)] $h_n=(-1)^n\cdot n$
    \columnbreak 
    \item[(i)] $i_n=\frac{(-1)^n}{n}$
    \item[(j)] $j_n=\frac{7n^3+8n}{2n^3-3}$
    \item[(k)] $k_n=\frac{9n^2-18}{6n+18}$
    \item[(l)] $l_n=\sin{(\frac{n\pi}{2})}$
      \columnbreak 
      \item[(m)] $m_n=\sin{(\frac{2n\pi}{3})}n$
      \item[(n)] $n_n=\frac{1}{n}\sin{(n)}$
      \item[(o)] $o_n=\frac{3^n}{n!}$
      \item[(p)] $p_n=\frac{6n+4}{9n^2+7}$
     \end{enumerate}
\end{multicols}
\end{exercice}
\vspace{1em}
\begin{exercice}
Déterminer les limites suivantes. Montrer toutes les étapes de démonstration.
\begin{enumerate}
    \item[(a)] lim $x_n$ où $x_n=\sqrt{n^2+1}-n$. \\ \textit{Indice : Montrer d'abord que } $x_n=\frac{1}{\sqrt{n^2+1}+n}$
    \item[(b)] lim $(\sqrt{n^2+n}-n)$,
    \item[(c)] lim $(\sqrt{4n^2+n}-2n)$.
    \item[\textit{Indications : }] \textit{ Utilisez l'identité remarquable : $a^2-b^2=(a+b)(a-b)$}
\end{enumerate}
\end{exercice}
\vspace{1em}
\begin{exercice}
Donnez un exemple d' :
\begin{enumerate}
    \item [(a)] Une suite $(x_n)$ de nombres irrationnels ayant une limite lim $x_n$ qui est un nombre rationnel.
    \item[*(b)] Une suite $(r_n)$ de nombres rationnels ayant une limite lim $r_n$ qui est un nombre irrationnel. 
\end{enumerate}
\end{exercice}
\vspace{1em}
\begin{exercice}
\begin{enumerate}
    \item [(i)] Donnez la définition de la limite d'une suite .
    \item[(ii)] Prouvez à l'aide de la définition de la limite que 
\begin{enumerate}
    \item[(a)] lim $\frac{1}{n^2}=0$
    \item[*(b)] lim$\frac{4n^3+3n}{n^3-6}=4$
    \item[(c)] la suite $x_n=(-1)^n$ ne converge pas. \\ \textit{Indice : Supposez que la suite converge et montrez que cela n'est finalement pas le cas. }
\end{enumerate}
\end{enumerate}
\end{exercice}
\vspace{1em}
\begin{exercice}
 Montrer la propriété suivante des limites de suites réelles:
$$\text{Si } \lim_{n \to \infty} x_n = x \; \text{  et  } \; \lim_{n \to \infty} y_n = y, \; \text{  alors  } \; \lim_{n \to \infty} x_ny_n = xy.$$
\textit{Indice: utiliser l'égalité \; $| x_ny_n - xy| = |x(y_n - y) + y_n(x_n-x) |$.} \\
\end{exercice}
\end{document}