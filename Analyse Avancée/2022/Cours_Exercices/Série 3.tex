\documentclass[11pt,french,table]{article}
\usepackage[french]{babel}
\usepackage[margin=1in,a4paper]{geometry}
\usepackage{multicol}

% Custom fonts. This package is only available with XeLaTex (pdflatex is a mess to deal with)
\usepackage{fontspec}
\setmainfont{GeneralSans}[
    Path = assets/fonts/,
    Extension = .otf,
    UprightFont = *-Regular,
    ItalicFont = *-Italic,
    BoldFont = *-Bold,
    BoldItalicFont = *-BoldItalic
]

% Custom titling
\usepackage{titling}
\usepackage{tcolorbox}

% Lipsum paragraphs
\usepackage{lipsum}

% Custom headers
\usepackage{fancyhdr}
\pagestyle{fancy}
\fancyhead[L]{\theauthor}
\fancyhead[C]{\itshape{\thetitle}}
\fancyhead[R]{\thedate}
\setlength{\headheight}{20pt}

% Default mathematical packages
\usepackage{amsmath}
\usepackage{amsfonts}

% Exercises environment and styling
\usepackage{amsthm}
\newtheoremstyle{exercice}%
    {3pt}% Space above
    {3pt}% Space below
    {\large}% Body font
    {}% Indent amount
    {\bfseries}% Theorem head font
    {.}% Punctuation after theorem heading
    {\newline}% Space after theorem heading
    {\thmname{#1}\thmnumber{ #2}\thmnote{: #3}}% Theorem head spec (can be left empty, meaning ‘normal’)
\theoremstyle{exercice}
\newtheorem{exercice}{Exercice}

% Graphics
\usepackage{graphicx}

\pretitle{\begin{center}\LARGE\bfseries}
\title{Analyse Avancée I - Série III}
\posttitle{\par\end{center}}

\renewcommand{\maketitlehookb}{
\begin{center}
\includegraphics[width=2cm]{assets/imgs/S4S_logo.png}
\end{center}
}

\author{Students 4 Students}
\date{Septembre 2022}

\renewcommand{\maketitlehookd}{
\begin{center}
\begin{tcolorbox}[boxrule=0pt,frame empty,width=0.8\textwidth]
"Les hautes mathématiques sont l’autre musique de la pensée", \begin{center}George Steiner.\end{center} 
\end{tcolorbox}
\scriptsize{Cette série vous est délivrée par Lounès, Louis et Till.} 
\end{center}
}

\begin{document}

\maketitle

\begin{exercice}
Pour la suite $x_n=\frac{1}{n}$, tracer le graphe de la suite. Choisissez ensuite un $\epsilon$ arbitrairement petit (ex $\epsilon=0.1$) et tracez le rang $N(\epsilon)$ associé. Tracez également le $\epsilon$.
\end{exercice}
\vspace{1em}
\begin{exercice}
Soient $(u_n)$ et $(v_n)$ deux suites réelles. Dire si les assertions suivantes sont vraies ou fausses? Lorsqu'elles sont vraies, les démontrer. Lorsqu'elles sont fausses, donner un contre-exemple. 
\begin{enumerate}
    \item Si $(u_n)$ et $(v_n)$ divergent, alors $(u_n+v_n)$ diverge. 
    \item Si $(u_n)$ et $(v_n)$ divergent, alors $(u_n \times v_n)$ diverge. 
    \item Si $(u_n)$ converge et $(v_n)$ diverge, alors $(u_n+v_n)$ diverge.
    \item Si $(u_n)$ converge et $(v_n)$ diverge, alors $(u_n\times v_n)$ diverge. 
    \item Si $(u_n)$ n'est pas majorée, alors $(u_n)$ tends vers $+\infty$. 
    \item Si $(u_n)$ est positive et tend vers $0$, alors $(u_n)$ est décroissante à partir d'un certain rang. 
\end{enumerate}
\end{exercice}
\vspace{1em}
\begin{exercice}
Pour les égalités suivantes dite si elles sont vraies ou fausses (démontrez dans le cas vrai et donnez un contre-exemple pour le cas opposé) : 
\begin{enumerate}
  
\begin{multicols}{2}
    \begin{enumerate}
     \item[(a)]  $\sum_{k=0}^{\infty}x_k= \sum_{n=0}^{\infty} x_n $\\
     \item[(b)]  $\sum_{k=0}^{n-1} \frac{1}{n}=1$ \\
     \item[(c)]  $\sum_{k=1}^{n} \frac{1}{k}=1$ \\
     \columnbreak
     \item[(d)]  \ $\sum_{k=0}^n x_k=\sum_{k=1}^{n+1} x_{k-1}$ \\
     \item[(e)]  \ $\sum_{k=0}^{100}k=5050$ \\
     \item[(f)]  \ $\text{Si } \lim_{n\xrightarrow{}\infty} x_n =0 \text{ alors } \sum_{n=0}^{\infty} x_k \text{ converge.}$
\end{enumerate}
\end{multicols}
\end{enumerate}
\end{exercice}

\vspace{1em}
\begin{exercice}
  Montrer que toute suite convergente est une suite de Cauchy. \\
Si $x_n$ converge vers $l \in \mathbb{R}$, montrer que $$\text{Pour tout  } \epsilon > 0, \; \text{  il existe  } N_\epsilon > 0 \; \text{ tel que }$$ $$n,m > N_\epsilon \implies | x_n - x_m | < \epsilon.$$
\end{exercice}
\vspace{1em}
\begin{exercice}
   *
Montrer que la suite $(u_n)$ définie par 
\begin{equation*}
    u_n=(-1)^n+\frac{1}{n}
\end{equation*}
n'est pas convergente. Montrer ensuite que cependant elle est bornée. Qu'en déduisez vous ? \paragraph{}
\textit{Indication : On prendra garde à ne pas parler de limite d'une suite sans avoir au préalable qu'elle converge !!\\
De plus, utilisez le résultat sur les sous-suites : Soit $(u_n)$ une suite qui converge vers sa limite $l$ alors toute sous-suite $(u_n)_{k\in \mathbb{N}}$ a pour limite $l$.}
\end{exercice}
\vspace{1em}
\begin{exercice} 
\vspace{1em}
Démontrer le théorème suivant : \\
 Soit $(x_n)\subset \mathbb{R}$. Si $(x_n )$ est croissante (resp. décroissante) et bornée supérieurement (resp. bornée inférieurement), alors $(x_n)$ est convergente.
\end{exercice}
\vspace{1em}
\begin{exercice}
Montrer en utilisant les suites de Cauchy que la suite $(x_n)$, définie par $x_n=(-1)^n$ est divergente. \\ \textit{Indication :  utilisez la négation de la définition d'être de Cauchy. }
\end{exercice}
\end{document}