\documentclass[11pt,french,table]{article}
\usepackage[french]{babel}
\usepackage[margin=1in,a4paper]{geometry}
\usepackage{multicol}

% Custom fonts. This package is only available with XeLaTex (pdflatex is a mess to deal with)
\usepackage{fontspec}
\setmainfont{GeneralSans}[
    Path = assets/fonts/,
    Extension = .otf,
    UprightFont = *-Regular,
    ItalicFont = *-Italic,
    BoldFont = *-Bold,
    BoldItalicFont = *-BoldItalic
]

% Custom titling
\usepackage{titling}
\usepackage{tcolorbox}

% Lipsum paragraphs
\usepackage{lipsum}

% Custom headers
\usepackage{fancyhdr}
\pagestyle{fancy}
\fancyhead[L]{\theauthor}
\fancyhead[C]{\itshape{\thetitle}}
\fancyhead[R]{\thedate}
\setlength{\headheight}{20pt}

% Default mathematical packages
\usepackage{amsmath}
\usepackage{amsfonts}

% Exercises environment and styling
\usepackage{amsthm}
\newtheoremstyle{exercice}%
    {3pt}% Space above
    {3pt}% Space below
    {\large}% Body font
    {}% Indent amount
    {\bfseries}% Theorem head font
    {.}% Punctuation after theorem heading
    {\newline}% Space after theorem heading
    {\thmname{#1}\thmnumber{ #2}\thmnote{: #3}}% Theorem head spec (can be left empty, meaning ‘normal’)
\theoremstyle{exercice}
\newtheorem{exercice}{Exercice}
\newtheoremstyle{corrigé}%
    {3pt}% Space above
    {3pt}% Space below
    {\large}% Body font
    {}% Indent amount
    {\bfseries}% Theorem head font
    {.}% Punctuation after theorem heading
    {\newline}% Space after theorem heading
    {\thmname{#1}\thmnumber{ #2}\thmnote{: #3}}% Theorem head spec (can be left empty, meaning ‘normal’)
\theoremstyle{corrigé}
\newtheorem{corrigé}{Corrigé Exercice}
% Graphics
\usepackage{graphicx}

\pretitle{\begin{center}\LARGE\bfseries}
\title{Algèbre Linéaire Avancée I - Corrigé III}
\posttitle{\par\end{center}}

\renewcommand{\maketitlehookb}{
\begin{center}
\includegraphics[width=2cm]{assets/imgs/S4S_logo.png}
\end{center}
}

\author{Students 4 Students}
\date{Septembre 2022}

\begin{document}

\maketitle
\vspace{1em}
\begin{corrigé}
1) Une droite quelconque n'est pas forcément un sous-espace vectoriel du plan. Si elle ne passe pas par l'origine, elle ne contient pas le vecteur nul. \\
Il faut se souvenir que, pour une approche géométrique, les vecteurs sont représentés comme \textit{partant de l'origine}. Vous pouvez vous convaincre, en faisant un schéma, que si $v$ et $w$ partent de l'origine et arrivent sur une droite \textit{qui ne passe pas par l'origine}, alors $v+w$ n'arrive \textit{pas} sur cette droite. \\
Le critère de SEV n'est donc pas vérifié.
\\
\\
2) Comme vu ci-dessus et discuté en cours, une droite qui passe par l'origine est un SEV. On peut s'en convaincre géométriquement, ou algébriquement. \\
Algébriquement, une droite passant par l'origine est un sous-ensemble de $\mathbb{R}^2$ qui s'écrit
$$D = \{  \lambda \cdot (a, b) : \lambda \in \mathbb{R} \}, \text{ pour un vecteur directeur } v= (a,b).$$
Faire une combinaison linéaire de $v_1 = \lambda_1 \cdot (a, b)$ et $v_2 = \lambda_2 \cdot (a,b)$ dans $D$ donne:
$$\mu_1 v_1 + \mu_2 v_2 = \big( (\mu_1 \lambda_1 + \mu_2\lambda_2) a, (\mu_1 \lambda_1 + \mu_2\lambda_2) b \big) = (\mu_1 \lambda_1 + \mu_2\lambda_2) \cdot (a,b) \in D,$$
et donc le critère de sous-espace vectoriel est vérifié. \\
\\
3) $W$ est un sous-espace vectoriel. À nouveau, on applique le critère de sous-espace vectoriel. \\
Si $v=(x_1,...,x_n)$ et $w=(y_1,...,y_n)$ sont dans $W$, alors
$$x_1 + \cdots + x_n = 0 = y_1 + \cdots + y_n.$$
Or,
$$v+w = (x_1 + y_1,...,x_n+y_n),$$
et donc la somme des coordonnées de $v+w$ donne bien 
$$x_1 + y_1 + \cdots + x_n + y_n = (x_1 + \cdots + x_n) + (y_1 + \cdots + y_n) = 0 + 0 = 0,$$
c'est-à-dire que $v+w \in W$. Le critère de SEV est vérifié.
\end{corrigé}
\begin{corrigé}
1) Montrons que $W_1+W_2$ est un sous-espace vectoriel de $V$. Pour cela, on montre que $W_1+W_2$ est stable par combinaison linéaire dans $K$. \\ \\ 
Soient $\lambda$ et $\mu \in K$ et $w,w' \in W_1+W_2$. Comme $w,w' \in W_1+W_2$, il existe $w_1,w_1' \in W_1$ et $w_2, w_2' \in W_2$ tels que $w=w_1+w_2$ et $w'=w_1'+w_2'$. Alors: \\
$$\lambda w + \mu w' = \lambda(w_1+w_2) + \mu(w_1'+w_2') = (\lambda w_1 + \mu w_1') + (\lambda w_2 + \mu w_2').$$
Or, comme $W_1$ est un SEV, on sait que $\lambda w_1 + \mu w_1' \in W_1$. Comme $W_2$ est un SEV, on sait que $\lambda w_2 + \mu w_2' \in W_2$. \\
\\
Ainsi, $\lambda w + \mu w'$ s'écrit bien comme la somme d'un élément de $W_1$ et d'un élément de $W_2$. On conclut bien que $$\lambda w + \mu w'\in W_1 + W_2,$$
et donc le critère de sous-espace vectoriel est vérifié. \\ \\
2) De la même manière, on utilise le critère de sous-espace vectoriel. Soient $\lambda, \mu \in K$ et $v,w \in W_1 \cap W_2$. Il suffit de montrer que 
\[
\lambda v+ \mu w \in W_1 \cap W_2, \; \text{ c.à.d. } \; \lambda v+ \mu w \in W_1 \; \text{et} \; \lambda v+ \mu w \in W_2
\]
En fait, la preuve est la même que pour les sous-groupes (série 2). Comme $W_1$ et $W_2$ sont individuellement des sous-espaces vectoriels, $\lambda v+ \mu w \in W_1$ et $\lambda v+ \mu w \in W_2$ sont vérifiés, et donc $$\lambda v+ \mu w \in W_1 \cap W_2.$$

%3) On regarde dans quels cas il est possible que $\text{Vec}(W_1 \cup  W_2)=W_1 \cup W_2$. Remarquez que l'inclusion $W_1 \cup W_2 \subseteq \text{Vec}(W_1 \cup W_2)$ est immédiate car pour chaque élément $w \in W_1 \cup W_2$ il suffit de considérer $\{w\}\subseteq W_1 \cup W_2$ et $\lambda=1_K \in K$ (donc $1_K \cdot w \in\text{Vec} (W_1\cup W_2)$)

%On étudie maintenant l'autre inclusion. Soit $w \in \text{Vec} (W_1 \cup W_2)$. Par définition il existent $\{w_1,..,w_n\} \subseteq W_1 \cup W_2$ et $\lambda_1,..,\lambda_n \in K$ tels que $w=\lambda_1 w_1+..+\lambda_n w_n$. On aimerait maintenant voir dans quels cas $w \in W_1 \cup W_2$ (i.e $w$ appartient à $W_1$ \textit{ou} $w$ appartient à $W_2$). On remarque que pour que $w$ soit dans $W_1$, il suffit que $W_2 \subseteq W_1$ (pour le cas où $w$ soit dans $W_2$ il suffit de considérer $W_1 \subseteq W_2$). Alors, si $W_2 \subseteq W_1$, comme $W_1$ et $W_2$ sans les sous espaces vectoriels, on a clairement $w \in W_1 \subseteq W_1\cup W_2$.

%Le cas $w \in W_2$ peut être traité exactement de même manière.

\end{corrigé}
\vspace{2em}
\begin{corrigé}
1) \\
a) L'application $g$ est linéaire. \\
Pour $(v_1,w_1)$ et $(v_2,w_2)\in V \times W$, et $\lambda, \mu \in K$, on a 
\begin{align*}
	g \big( \lambda(v_1,w_1)+ \mu(v_2,w_2) \big) &=g(\lambda v_1+ \mu v_2, \lambda w_1+ \mu w_2)\\ 
				     &=\lambda v_1+ \mu v_2 + \lambda w_1+ \mu w_2\\ 
				     &=\lambda v_1+ \lambda w_1 + \mu v_2+ \mu w_2 \\
&=\lambda (v_1+w_1)+ \mu(v_2+w_2) \\
&=\lambda g(v_1,w_1)+ \mu g(v_2,w_2),
\end{align*}
et donc on a bien $g \big( \lambda(v_1,w_1)+ \mu(v_2,w_2) \big) = \lambda g(v_1,w_1)+ \mu g(v_2,w_2)$, c'est-à-dire que $g$ est linéaire. \\
\\
b) $\pi_i$ est linéaire. Pour $\lambda, \mu \in K$ et $(a_1,..,a_n),(b_1,..,b_n) \in K^n$, on a 
\begin{align*}
	\pi_i \big( \lambda(a_1,..,a_n)+\mu(b_1,..,b_n) \big) &=\pi_i(\lambda a_1+\mu b_1,..,\lambda a_n+ \mu b_n)\\ 
					      &=\lambda a_i+ \mu b_i=\lambda \pi_i(a_1,..,a_n)+ \mu \pi_i(b_1,..,b_n),
\end{align*}
et donc $\pi_i$ est linéaire. \\
\\
c) L'application $f$ n'est pas linéaire. En effet pour les scalaires $\lambda, \mu \in K$ et les vecteurs $x = (x_1,x_2)$ et $y= (y_1,y_2) \in K^2$,
$$f( \lambda x + \mu y) = f (\lambda x_1 + \mu y_1, \lambda x_2 + \mu y_2) = (\lambda x_1 + \mu y_1) \cdot (\lambda x_2 + \mu y_2).$$
D'un autre côté, on a 
$$\lambda f(x) + \mu f(y) = \lambda \cdot x_1 \cdot y_1 + \mu \cdot x_2 \cdot y_2.$$
Pour montrer que ces deux expressions ne sont pas égales, considérons un cas qui simplifie les calculs: $x_1 = 0$  et $y_2 = 0$. Alors:
$$f( \lambda x + \mu y) = (\lambda x_1 + \mu y_1) \cdot (\lambda x_2 + \mu y_2) = \mu \cdot y_1 \cdot \lambda \cdot x_2 \text{ et}$$
$$\lambda \cdot x_1 \cdot y_1 + \mu \cdot x_2 \cdot y_2 = 0 + 0 = 0.$$
En choisissant $\mu$, $y_1$, $\lambda$ et $x_2$ non-nuls, par intégrité (on travaille dans un corps) on a $\mu \cdot y_1 \cdot \lambda \cdot x_2 \neq 0$. Les deux expressions ne sont pas égales pour ce cas spécifique, donc elles ne peuvent pas être égales en général. \\
\\
2) Supposons que $f(X)$ est libre dans $W$ est montrons que $X$ est libre dans $V$. \\
Pour $\lambda_1,...,\lambda_n \in K$ on a 
\begin{align*}
	\lambda_1 x_1+..+\lambda_k x_k&=0\\ 
	\iff f(\lambda_1x_1+..\lambda_kx_k)&=f(0)=0\\
	\iff \lambda_1f(x_1)+..+\lambda_kf(x_k)&=0
\end{align*}
Or, comme $f(X) = \{ f(x_1),...,f(x_n) \}$ est libre, on a que forcément $\lambda_1= \cdots =\lambda_k=0$. \\
Ainsi, $\lambda_1 v_1 + \cdots + \lambda_n v_n = 0$, et donc on a obtenu l'implication:
$$\lambda_1 x_1 + \cdot + \lambda_k x_k = 0 \implies \lambda_1 = \cdots = \lambda_k = 0,$$
ce qui veut dire que $X$ est libre. \\
\\
3) Soit $y \in f(V)$. Il existe $v \in V$ tel que $f(v)=y$. De plus, comme $X$ génère $V$, il existe $\lambda_1,...,\lambda_k \in K$ tels que $v=\lambda_1 x_1+ \cdots +\lambda_k x_k$. Ainsi,
\begin{align*}
f(v)=f(\lambda_1 x_1+..+\lambda_k x_k)=\lambda_1 f(x_1)+..+\lambda_k f(x_k)=y.
\end{align*}
$y$ s'écrit donc comme combinaison linéaire d'éléments de $f(X) = \{f(x_1,...,f(x_k) \}$. Comme $y$ était arbitrairement choisi dans $f(V)$, on conclut que $f(X)$ engendre $f(V)$.
\end{corrigé}
\vspace{2em}
\begin{corrigé}
Soient $V,W$ deux $K$ espaces vectoriels de dimension finie. On montre l'énoncé par double implication. \\
\\
\underline{$"\implies"$}: Si $V$ et $W$ sont isomorphes, alors il existe un isomorphisme $\varphi: V \longrightarrow W$. Par le rappel, il envoie toute base de $V$ sur une base de $W$. Comme tout élément dans $W$ est atteint exactement une fois ($\varphi$ est une bijection), alors une base $B \subseteq V$ contenant $n$ éléments est envoyée vers un ensemble contenant $n$ éléments. Par le rappel, cet ensemble est une base. Donc les bases de $W$ ont la même taille que les bases de $V$, et $\dim(V) = \dim(W)$.
\\
\\
\underline{$"\impliedby"$}: Supposons maintenant que $V$ et $W$ sont de la même dimension (disons $n \in \mathbb{N}$) et  considérons leur bases respectives $\mathcal{B}_V=\{v_1,..,v_n\}$ et $\mathcal{B}_W=\{w_1,..,w_n\}$. Il suffit de considérer l'application $\phi:V\mapsto W$ caractérisée par
\[
\phi(v_i)=w_i, \; \forall i \leq n.
\]
Comme cette application envoie une base de $V$ sur une base de $W$, on conclut par le rappel que $\phi$ est un isomorphisme, et donc que $V$ et $W$ sont isomorphes.
\end{corrigé}
\end{document}