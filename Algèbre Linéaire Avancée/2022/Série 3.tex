\documentclass[11pt,french,table]{article}
\usepackage[french]{babel}
\usepackage[margin=1in,a4paper]{geometry}
\usepackage{multicol}

% Custom fonts. This package is only available with XeLaTex (pdflatex is a mess to deal with)
\usepackage{fontspec}
\setmainfont{GeneralSans}[
    Path = assets/fonts/,
    Extension = .otf,
    UprightFont = *-Regular,
    ItalicFont = *-Italic,
    BoldFont = *-Bold,
    BoldItalicFont = *-BoldItalic
]

% Custom titling
\usepackage{titling}
\usepackage{tcolorbox}

% Lipsum paragraphs
\usepackage{lipsum}

% Custom headers
\usepackage{fancyhdr}
\pagestyle{fancy}
\fancyhead[L]{\theauthor}
\fancyhead[C]{\itshape{\thetitle}}
\fancyhead[R]{\thedate}
\setlength{\headheight}{20pt}

% Default mathematical packages
\usepackage{amsmath}
\usepackage{amsfonts}

% Exercises environment and styling
\usepackage{amsthm}
\newtheoremstyle{exercice}%
    {3pt}% Space above
    {3pt}% Space below
    {\large}% Body font
    {}% Indent amount
    {\bfseries}% Theorem head font
    {.}% Punctuation after theorem heading
    {\newline}% Space after theorem heading
    {\thmname{#1}\thmnumber{ #2}\thmnote{: #3}}% Theorem head spec (can be left empty, meaning ‘normal’)
\theoremstyle{exercice}
\newtheorem{exercice}{Exercice}
\newcommand{\C}{\mathbb{C}}
\newcommand{\R}{\mathbb{R}}
\newcommand{\Q}{\mathbb{Q}}
\newcommand{\Z}{\mathbb{Z}}
\newcommand{\N}{\mathbb{N}}

\DeclareMathOperator{\Vect}{Vec}
% Graphics
\usepackage{graphicx}

\pretitle{\begin{center}\LARGE\bfseries}
\title{Algèbre Linéaire Avancée I - Série III}
\posttitle{\par\end{center}}

\renewcommand{\maketitlehookb}{
\begin{center}
\includegraphics[width=2cm]{assets/imgs/S4S_logo.png}
\end{center}
}

\author{Students 4 Students}
\date{Septembre 2022}

\renewcommand{\maketitlehookd}{
}

\begin{document}

\maketitle
\vspace{1em}
\begin{exercice}
Pour le corps $K$, le $K$-espace vectoriel $V$ et le sous-ensemble $W \subseteq V$, dire si $W$ est un sous-espace vectoriel de $V$. \\
\\
1) $K = \R$, $V = \R^2$, et $W$ une droite quelconque du plan. \\ \\
2) $K = \R$, $V = \R^2$, et $W$ une droite du plan passant par l'origine. \\ \\
3) $K$ un corps quelconque, $V = K^n$ et $$W = \{ (x_1,...,x_n) \in K^n : x_1 + \cdots + x_n = 0_K \}.$$
\end{exercice}
\vspace{2em}
\begin{exercice}
Soient $V$ un $K$-espace vectoriel et $W_1, W_2 \subseteq V$ des sous-espaces. \\ \\
\textnormal{1)} On définit $W_1 + W_2 = \{ w_1 + w_2 \in V : w_1 \in W_1, \, w_2 \in W_2 \} \subseteq V$. Montrer que c'est un sous-espace vectoriel. \\ \\
\textnormal{2)} Montrer que $W_1 \cap W_2$ est un sous-espace vectoriel.
\end{exercice}
\vspace{2em}
\begin{exercice}
Soient $K$ un corps et $V,W$ deux $K$-espaces vectoriels. \\ \\
1) Déterminez si les applications suivantes sont linéaires en justifiant votre réponse.
\\ \\
\indent a) Pour $V+W$ l'espace vectoriel comme vu dans l'\textbf{Exercice 2},
\begin{align*}
	g: V\times W &\longrightarrow V+W \\
	(v,w) &\longmapsto v+w
.\end{align*}
\indent b) Pour $i \leq n$,
\begin{align*}
	\pi_i: K^n &\longrightarrow K \\
	(a_1,..,a_n) &\longmapsto a_i.
\end{align*}
\indent c) Le produit scalaire,
\begin{align*}
	f : K^2 &\longrightarrow  K\\
	(x,y) &\longmapsto x \cdot y.
\end{align*} \\
2) Soit $X=\{x_1,..,x_k\} \subseteq V$ et $f:V \longrightarrow W$ une application linéaire. Si l'ensemble $f(X) = \{f(x_1),...,f(x_n) \}$ est libre, montrer que $X$ est libre. \\
\\
3) Supposons que $X$ du point précédent engendre $V$. Montrer que $f(X)$ engendre $f(V) = \text{Im}(f)$.
\end{exercice}
\vspace{2em}
\begin{exercice}[Théorème de cours*]
Soient $V$ et $W$ deux espaces vectoriels sur un corps $K$. \\
À partir du résultat suivant: $$\varphi : V \longrightarrow W \text{ est un isomorphisme si et seulement si il envoie toute base de } V \text{ sur une base de } W,$$
prouvez formellement le \textbf{Théorème 3.3.4}:
$$ V \text{ et } W \text{ sont isomorphes si et seulement si ils sont de même dimension.}$$
\end{exercice}
\vspace{2em}
\end{document}