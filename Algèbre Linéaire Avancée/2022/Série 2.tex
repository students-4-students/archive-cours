\documentclass[11pt,french,table]{article}
\usepackage[french]{babel}
\usepackage[margin=1in,a4paper]{geometry}
\usepackage{multicol}

% Custom fonts. This package is only available with XeLaTex (pdflatex is a mess to deal with)
\usepackage{fontspec}
\setmainfont{GeneralSans}[
    Path = assets/fonts/,
    Extension = .otf,
    UprightFont = *-Regular,
    ItalicFont = *-Italic,
    BoldFont = *-Bold,
    BoldItalicFont = *-BoldItalic
]

% Custom titling
\usepackage{titling}
\usepackage{tcolorbox}

% Lipsum paragraphs
\usepackage{lipsum}

% Custom headers
\usepackage{fancyhdr}
\pagestyle{fancy}
\fancyhead[L]{\theauthor}
\fancyhead[C]{\itshape{\thetitle}}
\fancyhead[R]{\thedate}
\setlength{\headheight}{20pt}

% Default mathematical packages
\usepackage{amsmath}
\usepackage{amsfonts}

% Exercises environment and styling
\usepackage{amsthm}
\newtheoremstyle{exercice}%
    {3pt}% Space above
    {3pt}% Space below
    {\large}% Body font
    {}% Indent amount
    {\bfseries}% Theorem head font
    {.}% Punctuation after theorem heading
    {\newline}% Space after theorem heading
    {\thmname{#1}\thmnumber{ #2}\thmnote{: #3}}% Theorem head spec (can be left empty, meaning ‘normal’)
\theoremstyle{exercice}
\newtheorem{exercice}{Exercice}
\newcommand{\C}{\mathbb{C}}
\newcommand{\R}{\mathbb{R}}
\newcommand{\Q}{\mathbb{Q}}
\newcommand{\Z}{\mathbb{Z}}
\newcommand{\N}{\mathbb{N}}

% Graphics
\usepackage{graphicx}

\pretitle{\begin{center}\LARGE\bfseries}
\title{Algèbre Linéaire Avancée I - Série II}
\posttitle{\par\end{center}}

\renewcommand{\maketitlehookb}{
\begin{center}
\includegraphics[width=2cm]{assets/imgs/S4S_logo.png}
\end{center}
}

\author{Students 4 Students}
\date{Septembre 2022}

\renewcommand{\maketitlehookd}{
}

\begin{document}

\maketitle
\vspace{1em}
\begin{exercice}
Soit $(G,*)$ un groupe pour lequel l'élément neutre est noté $e$ et l'inverse de $x \in G$ est notée $x^{-1}$. \\
À partir des 3 propriétés qui définissent les groupes, montrer les propriétés suivantes, pour tous $x,y,z \in G$: \\
\\
1) $z * x = z * y \implies x = y$ (simplification à gauche). \\
2) $x*z =y*z \implies x = y$ (simplification à droite). \\
3) $\left( x^{-1} \right) ^{-1} = x$ (involutivité de l'inverse). \\
4) $(x * y) ^{-1} = y^{-1} * x^{-1}$ (anti-commutativité de l'inverse).
\end{exercice}
\vspace{2em}
\begin{exercice}
Pour $(G,\star)$ et $(H,*)$ deux groupes, leur produit cartésien $G \times H$ reste un groupe pour la loi de groupe ``coordonnée par coordonnée''. À quelle condition ce groupe est-il abélien?
\end{exercice}
\vspace{2em}
\begin{exercice}
Pour le groupe $G$ et le sous-ensemble $H \subseteq G$, est-ce que $H$ est un sous-groupe de $G$? Justifier. \\ \\
1) $G = (\Z,+)$ et $H = \N$. \\
2) $G = (\R^*, \ \cdot \ )$ et $H = \R_+^*$. \\
3) Pour $(G_1,\star_1)$ et $(G_2,\star_2)$ deux groupes, $G = G_1 \times G_2$ et $H = G_1 \times \{e_2\}$, où $e_2$ est l'élément neutre de $G_2$.
\end{exercice}
\vspace{2em}
\begin{exercice}
Montrer que la composée de deux morphismes de groupes reste un morphisme de groupes. \\
Si $(G,\star)$, $(H,*)$ et $(F,\diamond)$ sont des groupes et $\varphi : G \longrightarrow H$ et $\psi : H \longrightarrow F$ sont deux morphismes de groupes, montrer que $\psi \circ \varphi : G \longrightarrow F$ est un morphisme de groupes.
\end{exercice}
\vspace{2em}    
\begin{exercice}
Soient $H,K \subseteq G$ deux sous-groupes de $(G,\star)$. Montrer que $H \cap K$ est un sous groupe de $H$ et de $K$ (et a fortiori de $G$). \\
Est-ce que l'énoncé ferait du sens sans l'existence du groupe $G$?
\end{exercice}
\vspace{2em}
\begin{exercice}
Soient $G(\star)$ et $(H,*)$ deux groupes, et $\phi : G \longrightarrow H$ un morphisme de groupes. \\
1) Montrer que $\ker(\phi)$ est un sous-groupe de $G$. \\
2) Montrer que $\text{Im}(\phi)$ est un sous-groupe de $H$.
\end{exercice}
\vspace{2em}
\begin{exercice}
Soit $(G,+)$ un groupe abélien et $(H,*)$ un autre groupe. \\
S'il existe un morphisme de groupes $\phi : G \longrightarrow H$ surjectif, montrer que $H$ est abélien.
\end{exercice}
\vspace{2em}
\begin{exercice}[Anneaux et morphismes*]
Soient $(A,+,*)$ et $(B,+,\star)$ deux anneaux et $\varphi : A \longrightarrow B$ un morphisme d'anneaux (non-trivial), c'est-à-dire que $$\varphi(a * b + c) = \varphi(a) \star \varphi(b) + \varphi(c) \; \text{ pour tous } \; a,b,c \in A.$$
1) Montrer que $\varphi(0_A) = 0_B$ et $\varphi(1_A) = 1_B$. \\
\\
2) Montrer que si $x,y \in \ker(\varphi)$, alors $x + y \in \ker(\varphi)$ et $x * y \in \ker(\varphi)$. \\
Malgré ces propriétés, on ne considère généralement pas $\ker(\varphi)$ comme un sous-anneau de $A$. Sauriez-vous dire pourquoi? \\
\\
3) Supposons que $A$ est un corps et que $\varphi$ n'est pas le morphisme nul. \\
Montrer que \\
\indent a) $\varphi$ est injectif. \\
\indent b) $\varphi(A)$ est inversible: pour tout $x \in A \setminus \{0_A\}$, il existe $y \in B$ tel que $$y \star \varphi(x) = \varphi(x) \star y = 1_B.$$
\textit{Indication: le critère d'injectivité vu pour les groupes est également valable pour les anneaux (et les corps).}
\end{exercice}






\end{document}