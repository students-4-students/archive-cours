\documentclass[11pt,french,table]{article}
\usepackage[french]{babel}
\usepackage[margin=1in,a4paper]{geometry}
\usepackage{multicol}

% Custom fonts. This package is only available with XeLaTex (pdflatex is a mess to deal with)
\usepackage{fontspec}
\setmainfont{GeneralSans}[
    Path = assets/fonts/,
    Extension = .otf,
    UprightFont = *-Regular,
    ItalicFont = *-Italic,
    BoldFont = *-Bold,
    BoldItalicFont = *-BoldItalic
]

% Custom titling
\usepackage{titling}
\usepackage{tcolorbox}

% Lipsum paragraphs
\usepackage{lipsum}

% Custom headers
\usepackage{fancyhdr}
\pagestyle{fancy}
\fancyhead[L]{\theauthor}
\fancyhead[C]{\itshape{\thetitle}}
\fancyhead[R]{\thedate}
\setlength{\headheight}{20pt}

% Default mathematical packages
\usepackage{amsmath}
\usepackage{amsfonts}

% Exercises environment and styling
\usepackage{amsthm}
\newtheoremstyle{exercice}%
    {3pt}% Space above
    {3pt}% Space below
    {\large}% Body font
    {}% Indent amount
    {\bfseries}% Theorem head font
    {.}% Punctuation after theorem heading
    {\newline}% Space after theorem heading
    {\thmname{#1}\thmnumber{ #2}\thmnote{: #3}}% Theorem head spec (can be left empty, meaning ‘normal’)
\theoremstyle{exercice}
\newtheorem{exercice}{Exercice}
\newtheoremstyle{corrigé}%
    {3pt}% Space above
    {3pt}% Space below
    {\large}% Body font
    {}% Indent amount
    {\bfseries}% Theorem head font
    {.}% Punctuation after theorem heading
    {\newline}% Space after theorem heading
    {\thmname{#1}\thmnumber{ #2}\thmnote{: #3}}% Theorem head spec (can be left empty, meaning ‘normal’)
\theoremstyle{corrigé}
\newtheorem{corrigé}{Corrigé Exercice}
% Graphics
\usepackage{graphicx}

\pretitle{\begin{center}\LARGE\bfseries}
\title{Algèbre Linéaire Avancée I - Corrigé I}
\posttitle{\par\end{center}}

\renewcommand{\maketitlehookb}{
\begin{center}
\includegraphics[width=2cm]{assets/imgs/S4S_logo.png}
\end{center}
}

\author{Students 4 Students}
\date{Septembre 2022}

\begin{document}

\maketitle
\vspace{1em}
\begin{corrigé}
a) Unions: \\
$$A \cup B = \{ 1,3,5,6,7 \}, \quad A \cup C = \{ 1,2,3,5,7 \}, \quad B \cup C = \{ 2,3,5,6,7 \}$$ $$A \cup B \cup C = \{ 1,2,3,5,6,7 \}.$$ Intersections: \\
$$A \cap B = \{ 5,7 \}, \quad A \cap C = \{ 3 \}, \quad B \cap C = \emptyset, \quad A \cap B \cap C = \emptyset.$$
\vspace{1em}
b) $A \setminus B = \{ 1,3 \}, \; A \setminus C = \{ 1,5,7 \}, \; B \setminus C = B, \; B \setminus A = \{ 6 \}, \; C \setminus A = \{ 2 \}, \; C \setminus B = C$.
c) $$\mathcal{P}(B) = \Big\{ \emptyset, \{5\},\{6\},\{7\},\{5,6\},\{5,7\},\{6,7\},B \Big\}.$$
\end{corrigé}
\vspace{2em}
\begin{corrigé}
On souhaite prouver une équivalence, c'est-à-dire une double implication. \\
\\
\underline{``$\implies$''}: On suppose que $A \subseteq B$, et on veut montrer que $(X \setminus B) \subseteq (X \setminus A)$. \\
Pour montrer cette inclusion, on choisit un élément quelconque de $X \setminus B$, et on montre qu'il appartient forcément à $X \setminus A$. \\
\\
Soit donc $x \in X \setminus B$: $x$ est dans $X$ sans être dans $B$. \\
Supposons par l'absurde que $x$ est dans $A$: comme $A \subseteq B$, on aurait $x \in B$, ce qui serait une contradiction avec ``$x \in X \setminus B$''. \\
On conclut que $x$ ne peut pas être $A$: il est dans $X$ mais pas dans $A$, et donc $x \in X \setminus A$. \\
\\
\underline{``$\impliedby$''} On suppose que $(X \setminus B) \subseteq (X \setminus A)$, et on veut montrer que $A \subseteq B$. On procède comme ci-dessus. \\
\\
Soit $x \in A$. Supposons par l'absurde que $x$ n'appartient pas à $B$: alors, $x \in X \setminus B$, et donc par hypothèse $x \in X \setminus A$. \\
De nouveau, on a une contradiction avec ``$x \in A$": on conclut que $x$ ne peut pas ``ne pas être dans $B$'', et donc $x \in B$.
\end{corrigé}
\vspace{2em}
\begin{corrigé}
L'union $A \cup B$ regroupe tous les éléments qui sont dans $A$, dans $B$ ou dans les deux. Ici, tous les éléments de $A$ sont déjà dans $B$ (par $A \subseteq B$) et donc $$A \cup B = B.$$ L'ensemble $A$, parce qu'il est compris dans $B$, est "redondant" pour l'opération d'union. \\
\\
L'intersection $A \cap B$ regroupe tous les éléments qui sont \textit{à la fois} dans $A$ et dans $B$. Ici, les éléments de $A$ sont aussi des éléments de $B$. Les autres éléments de $B$ ne sont par définition pas dans $A$: on a donc $$A \cap B = A.$$
\vspace{1em}
Enfin, la différence $A \setminus B$ regroupe les éléments qui sont dans $A$ \textit{sans} être dans $B$. Or, ici tous les éléments qui sont dans $A$ sont forcément aussi dans $B$: il n'y a aucun élément qui soit dans $A$ sans être dans $B$.
$$A \setminus B = \emptyset.$$
\end{corrigé}
\vspace{2em}
\begin{corrigé}
a) À partir des hypothèses, on aimerait montrer que $A \subseteq B$, c'est-à-dire que
$$\text{Pour tout } b \in A, \; b \in C.$$
Soit donc un $b \in B$ quelconque. On distingue deux cas: soit $b \in A$, soit $b \notin A$. \\ \\
Dans le premier cas, comme $b \in A$ et $b \in B$, on sait que $b \in A \cap B$. Par la seconde hypothèse, $A \cap B \subseteq A \cap C$ et donc $b \in A \cap C$. \\
En particulier, cela veut dire que $b \in C$. \\
Dans le second cas, $b \notin A$. On sait cependant que $b \in A \cup B$, et par la première hypothèse, $A \cup B \subseteq A \cup C$: ainsi $b \in A \cup C$ : soit $b$ est dans $A$, soit il est dans $C$. \\
On a supposé ici que $b \notin A$: on a donc forcément $b \in C$. \\
\\
Dans tous les cas, $b \in C$. \\
On a montré que $b \in B \implies b \in C$, et donc $B \subseteq C$. \\ \\ \\
b) On veut montrer une égalité entre les ensembles $A$ et $B$: on procède par double inclusion. \\
\\
\underline{$A \subseteq B$}. \\
Soit $a \in A$. On sait que, par définiton, $a \in A \cup B$. Comme $A \cup B = A \cap B$ par hypothèse, on sait que $a \in A \cap B$. \\
En particulier, on conclut que $a \in B$. Cela prouve la première inclusion. \\
\\
\underline{$B \subseteq A$}. \\
Le raisonnement est exactement le même: soit $b \in B$. Alors:
$$b \in A \cup B \implies b \in A \cap B \implies b \in A.$$
Cela prouve la seconde inclusion.
\end{corrigé}
\vspace{2em}
\begin{corrigé}
On souhaite montrer qu'une application est bijective: il faut montrer qu'elle est injective, puis qu'elle est surjective. \\
La définition de l'application réciproque est $$f^{-1}(b) = a \text{ tel que } f^{-1}(\{b\}) = \{a\}.$$
\underline{$f^{-1}$ est injective.} On veut montrer que $$f^{-1}(b_1) = f^{-1}(b_2) \implies b_1 = b_2.$$
Soient donc $b_1,b_2 \in B$ tels que $f^{-1}(b_1) = f^{-1}(b_2)$. \\
Cela veut dire que $$b_1 = f \Big( f^{-1}(b_1) \Big) = f \Big( f^{-1}(b_2) \Big) = b_2,$$
et donc on a prouvé l'injectivité.
\\
\\
\underline{$f^{-1}$ est surjective}. On veut montrer que tout élément $x \in X$ est "atteint" par $f^{-1}$. \\
Soit donc $x \in X$. On veut montrer qu'il existe un $y \in Y$ tel que $f^{-1}(y) = x$. \\ \\
On peut deviner que cet élément $y$ est $f(x)$. En effet:
$$f^{-1} \Big( f(x) \Big) = x.$$
Ainsi, pour tout élément $x \in X$, il existe un élément $y=f(x)$ dans $Y$ tel que $f^{-1}(y)=x$. On a donc prouvé la surjectivité. \\
\\
L'application $f^{-1}$ est injective et surjective: elle est bijective.
\end{corrigé}
\vspace{2em}
\begin{corrigé}
1) Comme vu en cours, cette application n'est ni injective ni surjective. \\ \\
L'application n'est pas injective car $f(-1) = f(1)$, et plus généralement $f(-x) = x$. Une application est dite injective lorsque deux éléments distincts n'ont jamais la même image: ici $-1$ et $1$ sont des éléments distincts dont l'image est la même. \\ \\
Elle n'est pas surjective parce que certains éléments de l'ensemble d'arrivée $\mathbb{R}$ ne sont pas atteints: c'est le cas de $-1$. \\
L'équation $x^2=-1$ n'a pas de solution dans $\mathbb{R}$: il n'existe aucun élément $x \in \mathbb{R}$ tel que $f(x) = -1$. \\ \\
Comme vu en cours, en restreignant l'ensemble de départ à $\mathbb{R}_+$, on règle le premier problème et on obtient une injection. En restreignant l'ensemble d'arrivée à $\mathbb{R}_+$, on règle le second problème et on obtient une surjection. \\
\\
\\
2) Cette application n'est pas injective mais est surjective. \\
\\
L'application n'est pas injective car certains éléments ont plus d'un antécédent. Par exemple, $(0,2)$ et $(2,0)$ donnent la même image $f(0,2) = f(2,0) = 2$. \\
\\
Elle est surjective parce que, pour tout $x \in \mathbb{R}$, l'antécédent $(x,0)$ vérifie $f(x,0) = x$. \\
On peut trouver un antécédent à tout élément de l'ensemble d'arrivée, et donc $f$ est surjective. \\
\\
\\
3) L'application est bijective. \\
\\
L'application est injective car $$f(x_1,x_2) = f(y_1,y_2) \implies \left\{ \begin{array}{rcl}
x_1+x_2 &=& y_1+y_2 \\
x_1-x_2 &=& y_1-y_2
\end{array} \right.
$$
En résolvant ce système, on aboutit aux conclusions $x_1 = x_2$ et $y_1 = y_2$. On a montré:
$$f(x_1,x_2) = f(y_1,y_2) \implies (x_1,x_2)=(y_1,y_2),$$
qui est la définition d'injectivité. \\
\\
Elle est surjective car $$f(x,y) = (a,b) \iff (x,y) = \Big( \dfrac{a-b}{2}, \dfrac{a+b}{2} \Big),$$
et donc tout élément $(a,b) \in \mathbb{R}^2$ admet une préimage. \\
La manière de le prouver est la résolution du système
$$f(x,y) = (a,b) \iff
\left\{ \begin{array}{rcl}
     x+y &=& a \\
     x-y &=& b
\end{array} \right.$$
\hspace{1em} \\
4) Cette application est surjective mais n'est pas injective. \\
\\
L'application est surjective par définition de l'ensemble $\mathbb{Q}$: tout élément $x \in \mathbb{Q}$, par définition, peut s'écrire comme $x = \frac{p}{q}$ avec $p \in \mathbb{Z}$  et $q \in \mathbb{Z}^*$. \\
Cependant, cette écriture n'est pas unique: $\frac{2}{4} = \frac{1}{2}$, et donc $f(1,2) = f(2,4)$ alors que $(1,2) \neq (2,4)$. \\
C'est ce manque d'unicité qui empêche l'application d'être injective, et donc (comme vu en cours) de la définir dans l'autre sens. \\
\\
\\
5) Cette application est une bijection. \\
En fait, au lieu de passer par les preuves de l'injectivité et de la surjectivité, on peut raisonner ainsi: si on trouve une application inverse $f^{-1}$, alors $f$ est une bijection. \\
\\
$f$ est définie comme une application qui envoie tout sous-ensemble $A \subseteq \mathbb{N}$ vers $\mathbb{N} \setminus A$, son \textit{complémentaire dans $\mathbb{N}$} (notons que $f$ envoie bien des \textit{ensembles} vers d'autres \textit{ensembles}). \\ \\
En fait, on se rend compte que $f$ est sa propre inverse. C'est ce qu'on appelle l'\textit{involutivité du complémentaire}: $$\mathbb{N} \setminus \big( \mathbb{N} \setminus A \big) = A.$$

\end{corrigé}
\vspace{2em}
\begin{corrigé}[Relations d'équivalence*]
1)
    a) C'est une relation d'équivalence. \\
    \textbf{Réflexivité} : clairement, tout élément $a \in \Z$ a la même parité que lui-même. Alors $(a,a)\in R$.\\
    \textbf{Symétrie} Si $a$ a la meme parité que $b$, forcément $b$ a la même parité que $a$ et donc $(a,b) \in R \iff (b,a)$.\\
    \textbf{Transitivité}: si $a$ et $b$ ont la même parité, et $b$ et $c$ ont la même parité, alors soit $a$ et $b$ sont pairs et donc $c$ est pair, soit ils sont impairs et donc $c$ est impair. Ils ont tous les trois la même parité, donc en particulier $(a,b) \in R$ et $(b,c) \in R \implies (a,c) \in R$. \\
    $R$ possède 2 classes d'équivalence: l'ensemble des entiers pairs, et l'ensemble des entiers impairs. \\
    \\
    b) Cette relation n'est pas symétrique: par exemple pour $a=4$ et $b=2$, on a que $a$ est divisible par $b$ (donc $(a,b) \in R$) mais $b$ n'est pas divisible par $a$ (donc $(b,a) \notin R$). \\
    $R$ n'est donc pas une relation d'équivalence. \\ \\
    c) C'est une relation d'équivalence (c'est le cas général de l'exemple a) où on avait $m = 2$). \\
    \textbf{Réflexivité}: Pour tout $a \in A$, $a-a = 0$ et par définition $m$ divise 0 (car on peut écrire $0=mk$, avec $k=0$). Ainsi $(a,a) \in R$. \\
    \textbf{Symétrie}: Si $m$ divise $a-b$, il existe un $k \in \mathbb{Z}$ tel que $km=a-b$. On conclut qu'il existe $-k \in \mathbb{Z}$ tel que $-km=b-a$, et donc $m$ divise $b-a$. On a donc $(a,b) \in R \iff (b,a) \in R$. \\
    \textbf{Transitivité}: on suppose que $m$ divise $a-b$ et $b-c$, c'est-à-dire $k_1m=a-b$ et $k_2m=b-c$ pour des $k_1,k_2 \in \mathbb{Z}$. On en déduit que $k_1m+k_2m= (a-b)+(b-c)=a-c$. On a donc trouvé $k=k_1+k_2$ tel que $km=a-c$, c'est-à-dire que $m$ divise $a-c$. \\
    On a bien que si $(a,b)$ et $(b,c) \in R$, alors $(a,c) \in R$. \\
    $R$ possède $m$ classes d'équivalence: l'ensemble des entiers divisibles par $m$, l'ensemble des entiers dont le reste est 1 pour la division euclidienne par $m$, l'ensemble des entiers dont le reste est 2 pour la division euclidienne par $m$,..., etc. jusqu'à l'ensemble des entiers dont le reste est $m-1$ pour la division euclidienne par $m$.
    \\ \\
    d) C'est une relation d'équivalence: c'est en fait celle qui permet de définir rigoureusement l'ensemble $\mathbb{Q}$. \\ \\
    \textbf{Réflexivité}: pour tout $(a,b) \in (\mathbb{Z}^*)^2$, on a $ab = ab$, et donc $\big((a,b),(a,b)\big) \in R$. \\
    \textbf{Symétrie}: si $\big((a,b),(c,d)\big) \in R$, c'est que $ad=bc$, donc $bc=ad$ et $\big((c,d),(a,b)\big) \in R$. \\
    Ainsi, $\big((a,b),(c,d)\big) \in R \iff \big((c,d),(a,b)\big) \in R$. \\
   \textbf{Transitivité}: si $\big((a,b),(c,d)\big)$ et $\big((c,d),(e,f)\big) \in R$, alors $$\textnormal{(1) } ad=bc \text{ et (2) } cf=de.$$
    En multipliant (2) par $b$, on obtient $bcf=bde$. En utilisant (1), cela revient à $adf=bde$. \\
    Par hypothèse, tous les rééls considérés sont non-nuls et donc on peut simplifier par $d$ pour obtenir $af=be$, c'est-à-dire $\big((a,b),(e,f)\big) \in R$. \\
    \\
    Notons que si l'ensemble considéré était $\mathbb{Z}^2$ au lieu de $\mathbb{Z}^*^2$, la transitivité ne serait pas vérifiée. \\
    En effet, on aurait $1 \cdot 0 = 2 \cdot 0$, et $0 \cdot 3 = 1 \cdot 0$, mais pas $1 \cdot 3 = 2 \cdot 1$, soit
    $\big((1,2),(0,0)\big)$ et $\big((0,0),(1,3)\big) \in R$, mais pas $\big((1,2),(1,3)\big) \in R$. \\
    Chaque classe d'équivalence de $R$ est de la forme $$\{ (a,b) \in \mathbb{Z}^2 : \dfrac{a}{b} = x\}.$$
    L'ensemble des classes d'équivalence est donc en bijection avec $\mathbb{Q}$. \\
    \\
    e) Cette relation n'est pas réflexive: $2$ et $2$ ne sont pas premiers entre eux, parce qu'ils admettent $2$ comme diviseur commun non trivial. \\
    Ce n'est donc pas une relation d'équivalence. \\ \\
    f) Cette relation n'est pas transitive: $2$ et $6$ ont $2$ comme diviseur commun, $6$ et $3$ ont $3$ comme diviseur commun, mais $2$ et $3$ n'ont pas de diviseurs en commun. \\
    Ce n'est pas une relation d'équivalence. \\
    \\ \\
    2) a) Le maximum est$\vert A^2 \vert = \vert A \vert ^2$, parce que $R$ doit être un sous-ensemble de $A^2$. Il est atteint par la relation ``totale'': la relation $R$ pour laquelle tous les éléments de $A \times A$ sont dans $R$. \\
    Par exemple, pour $A = \mathbb{N}$, la relation d'équivalence $(a,b) \in R \iff a+b \in \mathbb{N}$ est totale: $R = \mathbb{N}^2$. \\ \\
    b) Le minimum est $\vert A \vert$. \\
    En effet, la relation doit être réflexive, et donc on sait que pour tout $a \in A$, $(a,a) \in R$. On a $\vert R \vert \geq \vert A \vert$. \\
    Il est atteint par la relation d'égalité $(a,b) \in R \iff a=b$.
    \end{corrigé}






\end{document} 