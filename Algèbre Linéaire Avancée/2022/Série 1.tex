\documentclass[11pt,french,table]{article}
\usepackage[french]{babel}
\usepackage[margin=1in,a4paper]{geometry}
\usepackage{multicol}

% Custom fonts. This package is only available with XeLaTex (pdflatex is a mess to deal with)
\usepackage{fontspec}
\setmainfont{GeneralSans}[
    Path = assets/fonts/,
    Extension = .otf,
    UprightFont = *-Regular,
    ItalicFont = *-Italic,
    BoldFont = *-Bold,
    BoldItalicFont = *-BoldItalic
]

% Custom titling
\usepackage{titling}
\usepackage{tcolorbox}

% Lipsum paragraphs
\usepackage{lipsum}

% Custom headers
\usepackage{fancyhdr}
\pagestyle{fancy}
\fancyhead[L]{\theauthor}
\fancyhead[C]{\itshape{\thetitle}}
\fancyhead[R]{\thedate}
\setlength{\headheight}{20pt}

% Default mathematical packages
\usepackage{amsmath}
\usepackage{amsfonts}

% Exercises environment and styling
\usepackage{amsthm}
\newtheoremstyle{exercice}%
    {3pt}% Space above
    {3pt}% Space below
    {\large}% Body font
    {}% Indent amount
    {\bfseries}% Theorem head font
    {.}% Punctuation after theorem heading
    {\newline}% Space after theorem heading
    {\thmname{#1}\thmnumber{ #2}\thmnote{: #3}}% Theorem head spec (can be left empty, meaning ‘normal’)
\theoremstyle{exercice}
\newtheorem{exercice}{Exercice}
\newcommand{\C}{\mathbb{C}}
\newcommand{\R}{\mathbb{R}}
\newcommand{\Q}{\mathbb{Q}}
\newcommand{\Z}{\mathbb{Z}}
\newcommand{\N}{\mathbb{N}}

% Graphics
\usepackage{graphicx}

\pretitle{\begin{center}\LARGE\bfseries}
\title{Algèbre Linéaire Avancée I - Série I}
\posttitle{\par
\end{center}}

\renewcommand{\maketitlehookb}{
\begin{center}
\includegraphics[width=2cm]{assets/imgs/S4S_logo.png}
\end{center}
}

\author{Students 4 Students}
\date{Septembre 2022}

\renewcommand{\maketitlehookd}{
}

\begin{document}

\maketitle
\begin{exercice}
On considère les ensembles $A= \{ 1,3,5,7 \}$, $B = \{5,6,7\}$ et $C = \{2,3\}$. \\ \\
a) Déterminer les toutes les unions et intersections possibles de ces ensembles. \\ \\
b) Déterminer toutes les différences (deux-à-deux) de ces ensembles. \\ \\
c) Pour un ensemble $X$, notons $\mathcal{P}(X)$ l'ensemble de ses sous-ensembles (les éléments de $\mathcal{P}(X)$ sont donc bien des \textit{ensembles}). Expliciter $\mathcal{P}(B)$.
\end{exercice}
\vspace{2em}
\begin{exercice}
Soient $A$ et $B$ deux sous-ensembles de $X$. \\
Démontrer soigneusement l'équivalence suivante: $$A \subseteq B \iff (X \setminus B) \subseteq (X \setminus A).$$
\textit{Note:} si on traduit cette énoncé dans le langage des propositions logiques, c'est la règle de la \textit{contraposée}. Elle exprime le fait que ``$P$ est vraie $\implies$ $Q$ est vraie'' est équivalent à ``$Q$ est fausse $\implies$ $P$ est fausse''.
\end{exercice}
\vspace{2em}
\begin{exercice}
Soient $A$ et $B$ deux ensembles tels que $A \subseteq B$. Expliciter $A \cup B$, $A \cap B$ et $A \setminus B$.
\end{exercice}
\vspace{2em}
\begin{exercice}
Soient $A$, $B$ et $C$ trois ensembles. \\
\\
a) Supposons que $A \cup B \subseteq A \cup C$ et $A \cap B \subseteq A \cap C$. Montrer que $B \subseteq C$. \\ \\
b) Démontrer soigneusement l'implication suivante:
$$A \cup B = A \cap B \implies A = B.$$
\end{exercice}
\vspace{2em}
\begin{exercice}
Soient $A$ et $B$ deux ensembles, et $f: A \longrightarrow B$ une bijection. \\
Montrer que son application réciproque $f^{-1} : B \longrightarrow A$ est également une bijection.
\end{exercice}
\vspace{2em}
\begin{exercice}
Dans chacun des cas suivants, dire si l'application $f : E \longrightarrow F$ est injective, surjective et/ou bijective. Lorsqu'elle est bijective, donner la formule de son inverse. 
\begin{align*}
\textnormal{1)}& \quad E = \R, F=\R, \hspace{3em} f(x) = x^2. \\
\textnormal{2)}& \quad E = \R^2, F= \R, \hspace{2.6em} f(x,y) = x + y. \\
\textnormal{3)}& \quad E = \R^2, F= \R^2, \hspace{2.3em} f(x,y) = (x + y,x-y). \\
\textnormal{4)}& \quad E = \Z \times \Z^*,F= \Q, \hspace{0.8em} f(x,y) =  \dfrac{x}{y}. \\
\textnormal{5)}& \quad E = F= \mathcal{P}(\N), \hspace{2.7em} f(A) = \N \setminus A.
\end{align*}
\end{exercice}
\vspace{2em}
\begin{exercice}[Relations d'équivalence*]
1) Décider, dans chacun des cas, si $R$ est une relation d'équivalence sur $E \times E$. Si oui, lister ses classes d'équivalence. \\ \\
\indent a) $E = \Z$, $(a,b) \in R \iff a$ et $b$ sont de même parité. \\
\\
\indent b) $E = \N$, $(a,b) \in R \iff a$ est divisible par $b$. \\ \\
\indent c) $E = \Z$, $m \in \Z^*$, $(a,b) \in R \iff m$ divise $a-b$. \\ \\
\indent d) $E = (\Z^*)^2$, $\big( (a,b),(c,d) \big) \in R \iff ad = bc$. \\ \\
\indent e) $E = \Z^*$, $(a,b) \in R \iff a$ et $b$ sont premiers entre eux. \\ \\
\indent f) $E = \Z$, $(a,b) \in R \iff a$ et $b$ ont au moins un diviseur $d > 1$ en commun. \\ \\ \\
2) Soit $A$ un ensemble fini non vide. \\ \\
\indent a) Pour $R \subseteq A \times A$ une relation d'équivalence, quel est le maximum de $\vert R \vert$? Quelle \indent est la relation d'équivalence qui atteint ce maximum? \\ \\
\indent b) Même question pour le minimum.
\end{exercice}
\end{document}