\textbf{\ul{Corrigé 1}:} \vskip 1em

\vskip 3em

\textbf{\ul{Corrigé 2}:} \vskip 1em

\vskip 3em

\textbf{\ul{Corrigé 3}:} \vskip 1em

a) Unions: \newline
$$A \cup B = \{ 1,3,5,6,7 \}, \quad A \cup C = \{ 1,2,3,5,7 \}, \quad B \cup C = \{ 2,3,5,6,7 \}$$ $$A \cup B \cup C = \{ 1,2,3,5,6,7 \}.$$ Intersections: \newline
$$A \cap B = \{ 5,7 \}, \quad A \cap C = \{ 3 \}, \quad B \cap C = \emptyset, \quad A \cap B \cap C = \emptyset.$$
\vspace{1em}
b) $A \setminus B = \{ 1,3 \}, \; A \setminus C = \{ 1,5,7 \}, \; B \setminus C = B, \; B \setminus A = \{ 6 \}, \; C \setminus A = \{ 2 \}, \; C \setminus B = C$.
c) $$\mathcal{P}(B) = \Big\{ \emptyset, \{5\},\{6\},\{7\},\{5,6\},\{5,7\},\{6,7\},B \Big\}.$$
\vskip 3em

\textbf{\ul{Corrigé 4}:} \vskip 1em
1) Comme vu en cours, cette application n'est ni injective ni surjective. \newline \newline
L'application n'est pas injective car $f(-1) = f(1)$, et plus généralement $f(-x) = x$. Une application est dite injective lorsque deux éléments distincts n'ont jamais la même image: ici $-1$ et $1$ sont des éléments distincts dont l'image est la même. \newline \newline
Elle n'est pas surjective parce que certains éléments de l'ensemble d'arrivée $\mathbb{R}$ ne sont pas atteints: c'est le cas de $-1$. \newline
L'équation $x^2=-1$ n'a pas de solution dans $\mathbb{R}$: il n'existe aucun élément $x \in \mathbb{R}$ tel que $f(x) = -1$. \newline \newline
Comme vu en cours, en restreignant l'ensemble de départ à $\mathbb{R}_+$, on règle le premier problème et on obtient une injection. En restreignant l'ensemble d'arrivée à $\mathbb{R}_+$, on règle le second problème et on obtient une surjection. \newline
\newline
\newline
2) Cette application n'est pas injective mais est surjective. \newline
\newline
L'application n'est pas injective car certains éléments ont plus d'un antécédent. Par exemple, $(0,2)$ et $(2,0)$ donnent la même image $f(0,2) = f(2,0) = 2$. \newline
\newline
Elle est surjective parce que, pour tout $x \in \mathbb{R}$, l'antécédent $(x,0)$ vérifie $f(x,0) = x$. \newline
On peut trouver un antécédent à tout élément de l'ensemble d'arrivée, et donc $f$ est surjective. \newline
\newline
\newline
3) L'application est bijective. \newline
\newline
L'application est injective car $$f(x_1,x_2) = f(y_1,y_2) \implies \left\{ \begin{array}{rcl}
x_1+x_2 &=& y_1+y_2 \newline
x_1-x_2 &=& y_1-y_2
\end{array} \right.
$$
En résolvant ce système, on aboutit aux conclusions $x_1 = x_2$ et $y_1 = y_2$. On a montré:
$$f(x_1,x_2) = f(y_1,y_2) \implies (x_1,x_2)=(y_1,y_2),$$
qui est la définition d'injectivité. \newline
\newline
Elle est surjective car $$f(x,y) = (a,b) \iff (x,y) = \Big( \dfrac{a-b}{2}, \dfrac{a+b}{2} \Big),$$
et donc tout élément $(a,b) \in \mathbb{R}^2$ admet une préimage. \newline
La manière de le prouver est la résolution du système
$$f(x,y) = (a,b) \iff
\left\{ \begin{array}{rcl}
     x+y &=& a \newline
     x-y &=& b
\end{array} \right.$$
\hspace{1em} \newline
4) Cette application est surjective mais n'est pas injective. \newline
\newline
L'application est surjective par définition de l'ensemble $\mathbb{Q}$: tout élément $x \in \mathbb{Q}$, par définition, peut s'écrire comme $x = \frac{p}{q}$ avec $p \in \mathbb{Z}$  et $q \in \mathbb{Z}^*$. \newline
Cependant, cette écriture n'est pas unique: $\frac{2}{4} = \frac{1}{2}$, et donc $f(1,2) = f(2,4)$ alors que $(1,2) \neq (2,4)$. \newline
C'est ce manque d'unicité qui empêche l'application d'être injective, et donc (comme vu en cours) de la définir dans l'autre sens. \newline
\newline
\newline
5) Cette application est une bijection. \newline
En fait, au lieu de passer par les preuves de l'injectivité et de la surjectivité, on peut raisonner ainsi: si on trouve une application inverse $f^{-1}$, alors $f$ est une bijection. \newline
\newline
$f$ est définie comme une application qui envoie tout sous-ensemble $A \subseteq \mathbb{N}$ vers $\mathbb{N} \setminus A$, son \textit{complémentaire dans $\mathbb{N}$} (notons que $f$ envoie bien des \textit{ensembles} vers d'autres \textit{ensembles}). \newline \newline
En fait, on se rend compte que $f$ est sa propre inverse. C'est ce qu'on appelle l'\textit{involutivité du complémentaire}: $$\mathbb{N} \setminus \big( \mathbb{N} \setminus A \big) = A.$$
\vskip 3em

\textbf{\ul{Corrigé 5}:} \vskip 1em
L'union $A \cup B$ regroupe tous les éléments qui sont dans $A$, dans $B$ ou dans les deux. Ici, tous les éléments de $A$ sont déjà dans $B$ (par $A \subseteq B$) et donc $$A \cup B = B.$$ L'ensemble $A$, parce qu'il est compris dans $B$, est "redondant" pour l'opération d'union. \\
\\
L'intersection $A \cap B$ regroupe tous les éléments qui sont \textit{à la fois} dans $A$ et dans $B$. Ici, les éléments de $A$ sont aussi des éléments de $B$. Les autres éléments de $B$ ne sont par définition pas dans $A$: on a donc $$A \cap B = A.$$
\vspace{1em}
Enfin, la différence $A \setminus B$ regroupe les éléments qui sont dans $A$ \textit{sans} être dans $B$. Or, ici tous les éléments qui sont dans $A$ sont forcément aussi dans $B$: il n'y a aucun élément qui soit dans $A$ sans être dans $B$.
$$A \setminus B = \emptyset.$$
\vskip 3em

\textbf{\ul{Corrigé 6}:} \vskip 1em
a) On veut montrer une égalité entre les ensembles $A$ et $B$: on procède par double inclusion. \newline
\newline
\underline{$A \subseteq B$}. \newline
Soit $a \in A$. On sait que, par définiton, $a \in A \cup B$. Comme $A \cup B = A \cap B$ par hypothèse, on sait que $a \in A \cap B$. \newline
En particulier, on conclut que $a \in B$. Cela prouve la première inclusion. \newline
\newline
\underline{$B \subseteq A$}. \newline
Le raisonnement est exactement le même: soit $b \in B$. Alors:
$$b \in A \cup B \implies b \in A \cap B \implies b \in A.$$
Cela prouve la seconde inclusion. \newline \newline \newline
b) À partir des hypothèses, on aimerait montrer que $A \subseteq B$, c'est-à-dire que
$$\text{Pour tout } b \in A, \; b \in C.$$
Soit donc un $b \in B$ quelconque. On distingue deux cas: soit $b \in A$, soit $b \notin A$. \newline \newline
Dans le premier cas, comme $b \in A$ et $b \in B$, on sait que $b \in A \cap B$. Par la seconde hypothèse, $A \cap B \subseteq A \cap C$ et donc $b \in A \cap C$. \newline
En particulier, cela veut dire que $b \in C$. \newline
Dans le second cas, $b \notin A$. On sait cependant que $b \in A \cup B$, et par la première hypothèse, $A \cup B \subseteq A \cup C$: ainsi $b \in A \cup C$ : soit $b$ est dans $A$, soit il est dans $C$. \newline
On a supposé ici que $b \notin A$: on a donc forcément $b \in C$. \newline
\newline
Dans tous les cas, $b \in C$. \newline
On a montré que $b \in B \implies b \in C$, et donc $B \subseteq C$.