% PREAMBULE
\begin{center}
\begin{tcolorbox}[boxrule=0pt,frame empty,width=\textwidth]
\textbf{Préambule :} Enfin des exercices d'algèbre linéaire ! Toutes les notions abordées dans ces exercices seront vues en long et en large pendant le semestre, n'hésitez donc pas à sauter des questions si vous sentez que vous y passez trop de temps.
\end{tcolorbox}
\end{center}

% EXERCICES

\begin{exercice}
Résoudre le système linéaire suivant avec la méthode "classique" puis avec la matrice augmentée, comparer les résultats:
\begin{equation*}
    \begin{cases}
    2x + y = 2 \\
    -2x + 2y = 1 \\
\end{cases}
\end{equation*} \\
\end{exercice}

\begin{exercice}
Vérifier si les espaces suivants sont des espaces vectoriels sur $\R$ ou non.

\begin{enumerate}
    \item $\mathbb Z^2$ muni des opérations usuelles d'addition de deux vecteurs et de la multiplication par un scalaire
        $$\begin{pmatrix} a \\ b \end{pmatrix} + \begin{pmatrix} c \\ d \end{pmatrix} = \begin{pmatrix} a + c \\ b + d \end{pmatrix}, \quad \lambda \cdot \begin{pmatrix} a \\ b \end{pmatrix} = \begin{pmatrix} \lambda a \\ \lambda b \end{pmatrix}$$
        où $a,b,c,d \in \mathbb Z$ et $\lambda \in \R$.
    \item $\P_n = \{c_0 + c_1 \cdot t + ... + c_{n} \cdot t^{n} \ \mid \ c_0, c_1, ..., c_{n} \in \R\}$, l'ensemble des polynômes de degré $\leq n$ à coefficients réels, muni des opérations usuelles d'addition de deux polynômes et de la multiplication par un scalaire.
    \item $(\R_+)^n = \{(x_1,...,x_n) \in \mathbb R^n : x_1 > 0, ..., x_n > 0\}$ muni des opérations usuelles d'addition de deux vecteurs et de la multiplication par un scalaire.
    \item $(\R_+)^n = \{(x_1,...,x_n) \in \mathbb R^n : x_1 > 0, ..., x_n > 0\}$ muni des opérations non-standards suivantes :
    $$\text{Addition : } \begin{pmatrix}
            x_1 \\ x_2 \\ \vdots \\ x_n
        \end{pmatrix} \oplus 
        \begin{pmatrix}
            y_1 \\ y_2 \\ \vdots \\ y_n
        \end{pmatrix} =
        \begin{pmatrix}
            x_1y_1 \\ x_2y_2 \\ \vdots \\ x_ny_n
        \end{pmatrix}$$
    
        $$\text{Multiplication par un scalaire : } \lambda \otimes \begin{pmatrix}
            x_1 \\ x_2 \\ \vdots \\ x_n
        \end{pmatrix} =
        \begin{pmatrix}
            x_1^{\lambda} \\ x_2^{\lambda} \\ \vdots \\ x_n^{\lambda}
        \end{pmatrix}$$
    où $x_i > 0, y_i > 0$ pour $i = 1, ..., n$ et $\lambda \in \R$. \\
    
\end{enumerate}
\end{exercice}

\begin{exercice}
Déterminer si les applications suivantes sont linéaires ou non:
\begin{enumerate}
    \item $T_1 \colon \R^3 \to \R$ définie par
    \[
        T_1 (\vec{x}) = T_1(x_1, x_2, x_3) = 3x_1 - x_2 - 15x_3
    \]
    \item $T_2: \R^2 \to \P_2$ définie par 
    $$T_2(\vec{x}) = T_2(x_1, x_2) = 1 + x_1 t + (x_1 + 3x_2)t^2$$
    \item $f_n: \P_n \to \R^{n+1}$ définie par, pour $p \in \P_n$, $p(t) = a_0 + a_1 t + \cdots + a_n t^n$: 
    $$f_n(p) = f_n(a_0 + a_1 t + \cdots + a_n t^n) = \begin{pmatrix}
        a_0 \\ a_1 \\ \vdots \\ a_n
    \end{pmatrix}$$ \\
\end{enumerate}
\end{exercice}

\begin{exercice}
    Soient $V, W$ des espaces vectoriels et $T: V \to W$ une application linéaire.
    \begin{enumerate}
        \item Montrer que $T(0_V) = 0_W$.
        \item Montrer que $T$ est injective si et seulement si ($\iff$) le seul antécédent de $0_W$ est $0_V$.
    \end{enumerate}
\end{exercice}