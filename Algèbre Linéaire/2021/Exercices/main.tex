\documentclass{article}
\usepackage[margin=1in,a4paper]{geometry}
\usepackage[utf8]{inputenc}
\usepackage[cyr]{aeguill}
\usepackage[francais]{babel}
\usepackage{hyperref}
\usepackage{amsmath}
\usepackage{gensymb}
\usepackage{enumitem,amssymb}
\newlist{checks}{itemize}{2}
\setlist[checks]{label=$\square$}
\usepackage{graphicx}
\usepackage{amsthm}
\usepackage{amsfonts}
\usepackage{pdfpages}
\usepackage{multicol}
\usepackage{pgfplots}
\pgfplotsset{compat=newest}
\usetikzlibrary{calc}
\usepackage{pgfplots}
\usepackage{mathtools}
\usepackage{array}
\usepackage[T1]{fontenc}
\usepackage{lmodern}
\usepackage{tabularx}
\usepackage{fancyhdr}
\usepackage{pst-func}
\usepackage{xcolor}
\usepackage{nicefrac}
\usepackage{mdframed}
\usepackage[boxed,vlined]{algorithm2e}
\usepackage{cleveref}
% \usepackage{graphicx}
\newcommand{\Lim}[1]{\raisebox{0.5ex}{\scalebox{1}{$\displaystyle \lim_{#1}\;$}}}
\usepackage{graphicx}
\usepackage{tkz-tab}

\newcommand{\N}{\mathbb{N}}
\newcommand{\R}{\mathbb{R}}
\newcommand{\C}{\mathbb{C}}
\newcommand{\Pn}{\mathbb{P}}
\newcommand{\w}{\omega}
\newcommand{\p}{\partial}
\newcommand{\cross}{\times}
\newcommand{\Col}{\text{Col}}
\newcommand{\Tr}{\text{Tr}}
\newcommand{\bigzero}{\makebox(0,0){\text{\huge0}}}
\DeclareMathOperator{\sinc}{sinc}
\DeclareMathOperator{\interior}{int}
\DeclareMathOperator{\adh}{adh}
\DeclareMathOperator{\argcosh}{argcosh}
\DeclareMathOperator{\argsinh}{argsinh}
\DeclareMathOperator{\Ima}{Im}
\DeclareMathOperator{\Vect}{Vect}
\usepackage{mathtools, stmaryrd}
\usepackage{xparse} \DeclarePairedDelimiterX{\Iintv}[1]{\llbracket}{\rrbracket}{\iintvargs{#1}}
\NewDocumentCommand{\iintvargs}{>{\SplitArgument{1}{,}}m}
{\iintvargsaux#1} %
\NewDocumentCommand{\iintvargsaux}{mm} {#1\mkern1.5mu..\mkern1.5mu#2}


\title{Students for Students \\Algèbre linéaire - Série}
\author{}
\date{}

\begin{document}
\tikzset{%
   point/.style = {fill=black,inner sep=1pt, circle, minimum width=3pt,align=right,rotate=60},
   } 
\tikzstyle{weight} = [font=\scriptsize]  
\tikzstyle{vertex}=[circle,fill=blue!20]

\maketitle

\subsection*{Préambule}
\textbf{Nous n'attendons pas de vous que vous finissiez tous ces exercices en trois heures. Pour les sections qui n'ont pas algèbre linéaire \textit{avancée}, nous vous conseillons de vous concentrer sur les exercices de la première page. Pour les sections MA et PH, nous conseillons de commencer par la deuxième page.}\\
Nous resterons à votre disposition le reste de la semaine pour vos éventuelles questions sur ces exercices.

\subsection*{Exercice 1}
\noindent Calculer les formes échelonnées réduites des matrices suivantes et préciser, pour chaque matrice, si son application matricielle associée est surjective et/ou injective :
\begin{multicols}{2}
\begin{enumerate}
    \item $\begin{bmatrix}
    4 & -1 & 3 & -6\\
    1 & -2 & 1 & 0\\
    0 & 5 & -2 & 8
    \end{bmatrix}$
    \item $\begin{bmatrix}
    1 & -5 & 4 & -3\\
    2 & -7 & 3 & -2\\
    -2 & 1 & 7 & -1
    \end{bmatrix}$ \\
    \item $\begin{bmatrix}
    1 & -3 & 2 \\
    -5 & -2  & 7  \\
    4 & -10  & 6 \\
    10 & -7 & -3
    \end{bmatrix}$
    \item $\begin{bmatrix}
    -4 & 5 & 9\\
    1 & -2 & 1\\
    0 & 2 & -8 
    \end{bmatrix}$
\end{enumerate}
\end{multicols}

\subsection*{Exercice 2}
\noindent Trouver l'ensemble de solutions des systèmes linéaires suivants :
\begin{multicols}{2}
\begin{enumerate}
    \item $\begin{cases}
    2x_1 - 6x_3 = -8\\
    x_2 + 2x_3 = 3\\
    3x_1 + 6x_2 -2x_3 = -4
    \end{cases}$
    \item $\begin{cases}
    2x_1 -7x_2+ 3x_3 = -2\\
    -2x_1 + x_2 + 7x_3 = -1\\
    x_1 -5x_2 + 4x_3 = -3\\
    \end{cases}$ 
    \item $\begin{bmatrix}
    1 & -4 & 7\\
    0 & 3 & -5\\
    -2 & 5 & -9
    \end{bmatrix}\begin{bmatrix}
    x_1 \\ x_2 \\ x_3
    \end{bmatrix} =
    \begin{bmatrix}
    1 \\ -3 \\ 1
    \end{bmatrix}$
    \item $\begin{bmatrix}
    2 & 3\\
    6 & 5\\
    2 & -5
    \end{bmatrix}x = 
    \begin{bmatrix}
    -1 \\ 0 \\ 7
    \end{bmatrix}$\\
    
    \noindent D'abord, $x \in \R^n$ où $n = $ ?
\end{enumerate}
\end{multicols}

\subsection*{Exercice 3}
\noindent Soit $b \in \R$, et soit la matrice $A = \begin{bmatrix} b & 1 \\ 1 & 0 \end{bmatrix} \in \R^{2 \cross 2}$. \\
Calculer $A^{-1}$ en fonction de $b$. \\
\textit{Indication: attention à la division par 0.} 

\subsection*{Exercice 4}
\begin{enumerate}
    \item Soient $a,b,c,d \in \R$, et soit $A = \begin{bmatrix} a & b \\ c & d \end{bmatrix} \in \R^{2 \cross 2}$. \\
    Quelle relation doivent vérifier les coefficients de $A$ pour que son inverse existe ? Sous l'hypothèse que cette relation est vérifiée, calculer $A^{-1}$.\\
    \textit{Indication: attention à la division par 0.}
    \item Vérifier que le résultat obtenu est bien en accord avec celui de l'exercice 3.
\end{enumerate} 

\subsection*{Exercice 5}
\noindent Nous notons l'ensemble des polynômes à coefficients réels de degré au plus $n$ par $\Pn_n$. De plus, pour $p \in \Pn_n$ et $t \in \R$, on écrira:
$$p(t) = a_0 + a_1 t + \cdots + a_n t^n = \sum_{i=1}^{n} a_i t^i, \; a_i \in \R$$
Notons qu'il existe une bijection naturelle entre $\Pn_n$ et $\R^{n+1}$, qui nous permet de faire correspondre un polynôme de degré au plus $n$ à un vecteur dans $\R^{n+1}$. Ceci est pratique, car nous pouvons utiliser ce que nous connaissons sur les vecteurs de $\R^m$ et les matrices pour résoudre des problèmes liés aux polynômes. 
\begin{enumerate}
    \item Soit $f_n: \Pn_n \to \R^{n+1}$ cette bijection.
    \begin{enumerate}
        \item Expliciter $f_n$, i.e, étant donné $p \in \Pn_n$, décrire $f_n(p)$.
        \item Vérifier que $\forall a, b \in \Pn_n$, $f_n(a + b) = f_n(a) + f_n(b)$ et que $\forall p \in \Pn_n$ et $\forall \lambda \in \R$, $f_n(\lambda p) = \lambda f_n(p)$. Notons que nous définissons $(a+b)(t) = a(t) + b(t)$.
    \end{enumerate}
    \item Soit $p \in \Pn_3$, et soit l'application
    \begin{align*}
        T: \ &\Pn_2 \to \Pn_1\\
        &p \mapsto T(p) \text{ tel que }\forall t \in \R \ T(p)(t) = (2a_0 + 3a_1) + (a_0 + 4a_2)t
    \end{align*}
    En utilisant la bijection $f_n$ trouvée dans la question 1 ainsi que sa fonction réciproque $f_n^{-1} : \R^{n+1} \to \Pn_n$ (telle que $f_n(f_n^{-1})(v) = v$ et $f_n^{-1}(f_n(p)) =p $), trouver la matrice associée à: 
    \begin{align*}
        f_1 \circ T \circ f_2^{-1}: \ &\R^3 \to \R^2\\
        &x \mapsto f_1(T(f_2^{-1}(x)))
    \end{align*}
    \textit{Indication : Il s'agit de trouver la matrice $A$ telle que $f_1(T(f_2^{-1}(x)))= Ax$. Explicitez alors $f_1(T(f_2^{-1}(x)))$, i.e, regardez ce que vaut cette quantité pour un $x$ arbitraire.}
    \item L'application matricielle associée à la matrice trouvée en $2$ est-elle surjective ? Injective ?\\
    
\end{enumerate}

\subsection*{Exercice 6}
\noindent Soit $A \in \R^{n \cross n}$. Nous avons vu en cours que l'application $T(x)=Ax$ va re-dimensionner et pivoter le vecteur $x$. Il est intéressant d'étudier l'ensemble des $x \in \R^n$ tels que l'application $T$ ne les pivote pas, i.e les $x \in \R^n$ qui ne changent pas de direction après avoir été transformés par $T$ et qui ne subissent qu'un re-dimensionnement. \\
Plus précisément, nous cherchons les $x \in \R^n$ tels que $Ax=\lambda x$, pour un ou plusieurs $\lambda \in \R$ qui sont aussi recherchés. Nous demanderons de plus que $x \neq 0$, car pour toute transformation matricielle $T$, $T(0)=0$ ce qui n'est pas intéressant.
\begin{enumerate}
    \item Soient $A \in \R^{n \cross n}$ et $\lambda \in \R$ tels qu'il existe $x \neq 0$ tel que $Ax = \lambda x$. Montrer que $\ker (A-\lambda I_n) \neq \{0\}$.
    \item Soit $A= \begin{bmatrix} -2 & -4 & 2 \\ -2 & 1 & 2 \\ 4 & 2 & 5 \end{bmatrix}$. \\
    Admettons que les valeurs de $\lambda$ tels que $Ax=\lambda x$ pour cette matrice sont $\lambda_1 = -5$, $\lambda_2 = 3$, $\lambda_3 = -6$. \\
    Calculer le noyau de $A-3 I_3$ et vérifier qu'il est bien différent de $\{0\}$.
\end{enumerate}
\underline{Remarque}: Les vecteurs calculés en 2) représentent l'ensemble des vecteurs tels que $Ax=3x$ pour cette matrice $A$. \\

\subsection*{Exercice 7}
\noindent Soient $b \in \R$, $n \in \N_{\geq 1}$, et soit $A = \begin{bmatrix} 1 & b \\ 0 & 1 \end{bmatrix} \in \R^{2 \cross 2}$. \\
Montrer par récurrence que $A^n = \underbrace{AAA\cdots AAA}_{n \text{ fois}} = \begin{bmatrix}
1 & nb \\ 0 & 1 \end{bmatrix}$.

\subsection*{Exercice 8}
\noindent Nous dirons que $D \in \R^{n \cross n}$ est \textit{diagonale} si et seulement si $\forall (i,j) \in \Iintv{1,n}^2 \ i \neq j \implies d_{i,j} = 0$, autrement dit :
$$
D = \begin{bmatrix}
d_{1,1} & 0 & 0 & \cdots & 0\\
0 & d_{2,2} & 0 & \cdots & 0 \\
\vdots & &\ddots & & \vdots \\
0 & 0 & \cdots & d_{n-1, n-1} & 0\\
0 & 0 &\cdots& 0 & d_{n,n}
\end{bmatrix}
$$ 
\noindent Montrer que si $A,B \in \R^{n \cross n}$ sont deux matrices diagonales de dimension $n\times n$ alors leur produit $AB$ est aussi une matrice diagonale de dimension $n\times n$. 

\subsection*{Exercice 9}
\noindent Une \textit{matrice triangulaire supérieure} est une matrice carrée telle que ses coefficients en dessous de sa diagonale sont tous nuls. Une matrice \textit{triangulaire inferieure} est une matrice de forme similaire, avec des coefficients nuls au dessus de la diagonale. C'est des matrices de la forme:
$$\begin{bmatrix}
* & * & * & \cdots & * \\
  & * & * & \cdots & * \\
  &   & \ddots & \ddots & \vdots \\
  & \bigzero  &  & \ddots & \vdots \\
  &   &   &  & *
\end{bmatrix} \text{ et }
\begin{bmatrix}
* &  &  &  &  \\
\vdots  & \ddots &  & \bigzero &  \\
\vdots & \ddots & \ddots &  &  \\
* & \cdots & * & * &  \\
*  & \cdots & * & * & *
\end{bmatrix}$$
Plus formellement, une matrice $A$ est triangulaire supérieure si $i > j \implies a_{i,j} = 0$, et est triangulaire inférieure si $i < j \implies a_{i,j} = 0$. Les matrices suivantes sont triangulaires supérieures :
$$
\begin{bmatrix}
1 & 2\\
0 & 3
\end{bmatrix}, \begin{bmatrix}
0 & 0\\
0 & 0
\end{bmatrix},
\begin{bmatrix}
0 & 3\\
0 & 1
\end{bmatrix}
$$

\noindent Montrer que le produit de deux matrices triangulaires supérieures est aussi une matrice triangulaire supérieure.

\subsection*{Exercice 10}
\noindent Soit $A \in \R^{n \cross p}$. Nous définissons la \textit{transposée} de $A$ comme étant la matrice $A^T \in \R^{p \cross n}$ telle que $(A^T)_{i,j} = A_{j,i}$. 
Par exemple : si $A = \begin{bmatrix}
1 & 2 & 3\\
4 & 5 & 6
\end{bmatrix}$, alors $A^T = \begin{bmatrix}
1 & 4\\
2 & 5 \\
3 & 6
\end{bmatrix}$.\\

\noindent Nous dirons que $A$ est \textit{symétrique} si $A^T=A$, et \textit{antisymétrique} si $A^T=-A$. \\
Par exemple, $\begin{bmatrix} 3 & 1 \\ 1 & 2 \end{bmatrix}$ est symétrique, et $\begin{bmatrix} 0 & 1 \\ -1 & 0 \end{bmatrix}$ est antisymétrique. \\

\begin{enumerate}
    \item Soient $A \in \R^{p \cross n}$, $B \in \R^{n \cross k}$. Montrer que $(AB)^T = B^T A^T$.
    \item En déduire, à l'aide du résultat de l'exercice 9, que le produit de deux matrices triangulaires inférieures est aussi une matrice triangulaire inférieure.
    \item Montrer que toute matrice carrée peut s'écrire comme somme d'une matrice symétrique et d'une matrice antisymétrique. \textit{Indication : pour $A \in \R^{n \cross n}$, utiliser $A$ et $A^T$ pour construire ces deux matrices. Poser un système linéaire si nécessaire.}
\end{enumerate}

%\subsection*{Exercice 11}
%\begin{enumerate}
%    \item Soient $A \in \R^{p \cross n}$, $B \in \R^{n \cross k}$. Montrer que $(AB)^T = B^T A^T$.
%    \item En déduire, à l'aide du résultat de l'exercice 9, que le produit de deux matrices triangulaires inférieures est aussi une matrice triangulaire inférieure. \\
%\end{enumerate}

\subsection*{Exercice 11}
\noindent Soit $A \in \R^{n \cross n}$. Nous définissons la \textit{trace} de $A$, notée $\Tr(A)$, comme la somme de ses éléments diagonaux. Autrement dit, $\displaystyle \Tr(A) \coloneqq \sum_{i=1}^{n} a_{i,i}$. Exemple : $\Tr\left(\begin{bmatrix}
1 & 2\\
0 & 5
\end{bmatrix}\right) = 1 + 5 = 6$.
\begin{enumerate}
    \item Soient $A,B \in \R^{n \cross n}$. Montrer que $\Tr(AB)=\Tr(BA)$
    \item Soient $A,B,C \in \R^{n \cross n}$. Montrer que $\Tr(ABC)=\Tr(CAB)$.
    \item L'égalité serait-elle toujours valable si on remplaçait son membre de droite par $\Tr(ACB)$?
\end{enumerate}
%hello louis 
%Hello Salim (c'etait moi (bassam) qui avait ecris ça ;-;)

\end{document}
