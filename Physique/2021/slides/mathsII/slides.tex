%%%%%%%%%%%%%%%%%%%%%%%%%%%%%%%%%%%%%%%%%
% Beamer Presentation
% LaTeX Template
% Version 1.0 (10/11/12)
%
% This template has been downloaded from:
% http://www.LaTeXTemplates.com
%
% License:
% CC BY-NC-SA 3.0 (http://creativecommons.org/licenses/by-nc-sa/3.0/)
%
%%%%%%%%%%%%%%%%%%%%%%%%%%%%%%%%%%%%%%%%%

%----------------------------------------------------------------------------------------
%	PACKAGES AND THEMES
%----------------------------------------------------------------------------------------

\documentclass[xcolor=table]{beamer}

\mode<presentation> {

% The Beamer class comes with a number of default slide themes
% which change the colors and layouts of slides. Below this is a list
% of all the themes, uncomment each in turn to see what they look like.

\usetheme{default}
%\usetheme{AnnArbor}
%\usetheme{Antibes}
%\usetheme{Bergen}
%\usetheme{Berkeley}
%\usetheme{Berlin}
%\usetheme{Boadilla}
%\usetheme{CambridgeUS}
%\usetheme{Copenhagen}
%\usetheme{Darmstadt}
%\usetheme{Dresden}
%\usetheme{Frankfurt}
%\usetheme{Goettingen}
%\usetheme{Hannover}
%\usetheme{Ilmenau}
%\usetheme{JuanLesPins}
%\usetheme{Luebeck}
%\usetheme{Madrid}
%\usetheme{Malmoe}
%\usetheme{Marburg}
%\usetheme{Montpellier}
%\usetheme{PaloAlto}
%\usetheme{Pittsburgh}
%\usetheme{Rochester}
%\usetheme{Singapore}
%\usetheme{Szeged}
%\usetheme{Warsaw}

% As well as themes, the Beamer class has a number of color themes
% for any slide theme. Uncomment each of these in turn to see how it
% changes the colors of your current slide theme.

%\usecolortheme{albatross}
%\usecolortheme{beaver}
%\usecolortheme{beetle}
%\usecolortheme{crane}
%\usecolortheme{dolphin}
%\usecolortheme{dove}
%\usecolortheme{fly}
%\usecolortheme{lily}
%\usecolortheme{orchid}
%\usecolortheme{rose}
%\usecolortheme{seagull}
%\usecolortheme{seahorse}
%\usecolortheme{whale}
%\usecolortheme{wolverine}
%\setbeamertemplate{footline} % To remove the footer line in all slides uncomment this line
\setbeamertemplate{footline}[page number] % To replace the footer line in all slides with a simple slide count uncomment this line

\setbeamertemplate{navigation symbols}{} % To remove the navigation symbols from the bottom of all slides uncomment this line 
}
\usepackage{amsmath} 
\usepackage{graphicx} % Allows including images
\usepackage{booktabs} % Allows the use of \toprule, \midrule and \bottomrule in tables
\usepackage[table]{xcolor}
\usepackage{pgfplots}
\usepackage{tcolorbox}
\pgfplotsset{compat=1.15}
\usepackage{mathrsfs}
\usetikzlibrary{arrows}
\usetikzlibrary{patterns}

\pgfplotsset{ % Define a common style, so we don't repeat ourselves
    MaoYiyi/.style={
        width=0.6\textwidth, % Overall width of the plot
        axis equal image, % Unit vectors for both axes have the same length
        view={0}{90}, % We need to use "3D" plots, but we set the view so we look at them from straight up
        xmin=0, xmax=110, % Axis limits
        ymin=0, ymax=110,
        domain=0:100, y domain=0:100, % Domain over which to evaluate the functions
        xtick={0,20,...,100}, ytick={0,20,...,100}, % Tick marks
        samples=11, % How many arrows?
        cycle list={    % Plot styles
                gray,
                quiver={
                    u={1}, v={f(x)}, % End points of the arrows
                    scale arrows=7.5,
                    every arrow/.append style={
                        -latex % Arrow tip
                    },
                }\\
                red, samples=31, smooth, thick, no markers, domain=0:110\\ % The plot style for the function
        }
    }
}
\usepackage{tikz}
\usepackage{pstricks-add}
\usetikzlibrary{arrows,shapes,positioning,shadows,trees}


\usepackage{tcolorbox}
\usepackage{wrapfig}


\usepackage{hyperref}
\hypersetup{
    colorlinks,
    citecolor=black,
    filecolor=black,
    linkcolor=black,
    urlcolor=black
}
\newcommand{\notimplies}{%
    \mathrel{{\ooalign{\hidewidth$\not\phantom{=}$\hidewidth\cr$\implies$}}}}


\DeclareMathAlphabet\mathzapf{T1}{pzc}{mb}{it}
\usepackage{amsmath,wasysym}
\usepackage{latexsym}

\usepackage{amssymb}
\usepackage{mathrsfs}
\usepackage{bm}
\usepackage{wrapfig}
\usepackage{fancybox}
\bibliographystyle{amsplain}
\usepackage{systeme}
\usepackage{pdfpages}


\usepackage{yfonts}
\usepackage[french]{babel}



\usepackage[T2A]{fontenc}




%\usepackage[style=ieee]{biblatex} %Use if necessary for citation
%\addbibresource{biblatex-examples.bib}
%----------------------------------------------------------------------------------------
%	TITLE PAGE
%----------------------------------------------------------------------------------------

\title[Physique]{Mathématiques II} % The short title appears at the bottom of every slide, the full title is only on the title page

\author{Team Physique} % Your name
\institute[S4S] % Your institution as it will appear on the bottom of every slide, may be shorthand to save space
{
initiative S4S\\ % Your institution for the title page
\medskip
}
\date{\today} % Date, can be changed to a custom date
\begin{document}
\begin{frame}{Approximation et développements limités : Introduction}
\begin{itemize}
    \item Approximer une fonction par un polynome
    \item Utilité d'un développement limité en physique : \newline
    -Approximer les fonctions trigonométriques autour de $0$
    \item Exemples usuels : \newline
    - $\cos(x) \approx 1$ \newline
    - $\sin(x) \approx x$
\end{itemize}    
\end{frame}
\begin{frame}{Approximation et développements limités : Pourquoi ces approximations, l'exemple du sinus et du cosinus}

\begin{figure}[ht]

\begin{center}
\definecolor{ccqqqq}{rgb}{0.8,0,0}
\definecolor{qqqqff}{rgb}{0,0,1}
\begin{tikzpicture}[line cap=round,line join=round,>=triangle 45,x=1cm,y=1cm]
\begin{axis}[
x=1cm,y=1cm,
axis lines=middle,
xmin=-2.5400689538405253,
xmax=2.536515566558789,
ymin=-2.111629827594053,
ymax=2.178011872488389,
xtick={-2,-1,...,2},
ytick={-2,-1,...,2},]
\clip(-2.8400689538405253,-2.111629827594053) rectangle (3.136515566558789,2.478011872488389);
\draw[line width=0.8pt,color=qqqqff,smooth,samples=100,domain=-2.8400689538405253:3.136515566558789] plot(\x,{cos(((\x))*180/pi)});
\draw [line width=0.8pt,color=ccqqqq,domain=-2.8400689538405253:3.136515566558789] plot(\x,{(--1-0*\x)/1});
\end{axis}
\end{tikzpicture}
%
\definecolor{ccqqqq}{rgb}{0.8,0,0}
\definecolor{qqqqff}{rgb}{0,0,1}
\begin{tikzpicture}[line cap=round,line join=round,>=triangle 45,x=1cm,y=1cm]
\begin{axis}[
x=1cm,y=1cm,
axis lines=middle,
xmin=-2.5400689538405253,
xmax=2.536515566558789,
ymin=-2.111629827594053,
ymax=2.178011872488389,
xtick={-2,-1,...,2},
ytick={-2,-1,...,2},]
\clip(-2.4558532849474526,-1.8811746892214536) rectangle (2.7186787673463244,2.092541068425681);
\draw[line width=0.8pt,color=qqqqff,smooth,samples=100,domain=-2.4558532849474526:2.7186787673463244] plot(\x,{sin(((\x))*180/pi)});
\draw [line width=0.8pt,color=ccqqqq,domain=-2.4558532849474526:2.7186787673463244] plot(\x,{(-0--1*\x)/1});
\end{axis}
\end{tikzpicture}
\end{center}

\caption{Approximation linéaire du cosinus (à gauche) et du sinus (à droite) autour de $0$}
    \label{fig:dl}
\end{figure}
\end{frame}

\begin{frame}{Approximation et développements limités : }
    

%Dessin de l'approximation
\begin{figure}
    
    
\definecolor{ffqqqq}{rgb}{1,0,0}
\definecolor{qqqqff}{rgb}{0,0,1}
\begin{center}

\begin{tikzpicture}[line cap=round,line join=round,>=triangle 45,x=1cm,y=1cm,scale=0.5
]
\begin{axis}[
x=2cm,y=1cm,
axis lines=middle,
xmin=-3.780437787628234,
xmax=4.220977016145955,
ymin=-3.0784578863074157,
ymax=3.0661263520169277,
xtick={-3.5,-2.5,...,3.5},
ytick={-3,-2,...,3},]
\clip(-3.780437787628234,-3.0784578863074157) rectangle (4.220977016145955,3.0661263520169277);

\draw[line width=0.8pt,color=red,smooth,samples=100,domain=-3.780437787628234:4.220977016145955] plot(\x,{(\x)});
\draw[line width=0.8pt,color=green,smooth,samples=100,domain=-5.1874758166367645:5.462407287186683] plot(\x,{(\x)-(\x)^(3)/6});
\draw[line width=0.8pt,color=yellow,smooth,samples=100,domain=-5.660273317658524:6.05459809654727] plot(\x,{(\x)-(\x)^(3)/6+(\x)^(5)/120});
\draw[line width=1pt,color=blue,smooth,samples=100,domain=-3.780437787628234:4.220977016145955] plot(\x,{sin(((\x))*180/pi)});
\begin{scriptsize}

\end{scriptsize}
\end{axis}
\end{tikzpicture}
\end{center}


\caption{Approximation de la fonction sinus avec de polynômes de Taylor de différents degré}
    \label{fig:my_label}
\end{figure}
*\begin{itemize}
    \item En rouge, l'approximation linéaire $f(x) = x$, qu'on a déjà vu
    \item En vert, l'approximation de degré 3 $g(x) = x - \frac{x^3}{6}$
    \item En jaune, l'approximation de degré 5 $h(x) = x - \frac{x^3}{6} + \frac{x^5}{120}$
\end{itemize}  
\end{frame}
\begin{frame}{Equations différentielles :}
\begin{itemize}
    
    \item  Equation où l'inconnue est une fonction et une plusieurs de ses dérivées sont présentes
    \item Très importante en physique
    \item visualisation graphique : les isoclines
    \end{itemize}
    
 \begin{figure}[H]
    \centering
    \includestandalone{Images/dif1}
    \caption{Le niveau d'eau dans le temps d'un réservoir qui fuit dépend de sa taille}
    \label{fig:my_label}

\end{figure}

    
\end{frame}
\begin{frame}{Equations différentielles: Exemple d'application}
Enoncé du problème : \newline
On considère un réservoir d'eau cylindrique, de hauteur $h$ et de rayon $R$. On remplit le réservoir d'un liquide et on se rend alors compte que le fond du réservoir est troué par un trou de rayon $r$, avec $r << R$. En tant que physicien, au lieu de prendre du scotch pour boucher le trou, vous décidez d'analyser la situation (oui oui, c'est parfaitement le bon moment). \\
Soit $y(t)$ une fonction qui représente le hauteur du liquide dans le réservoir au temps t. On suppose que le liquide quitte le réservoir à une vitesse $2\sqrt{gy}$. La question est donc la suivante: Si le réservoir est plein au temps t = 0, combien de temps faut-il pour qu'il se vide ?
    
\end{frame}
\begin{frame}{Equations différentielles: Exemple d'application (2)}
    
\end{frame}
\begin{frame}{Equations différentielles: Exemple d'application (3)}
    
\end{frame}




















\end{document}