\documentclass[a4paper,landscape,margin=1cm]{article}

% Language setting
% Replace `english' with e.g. `spanish' to change the document language
\usepackage[french]{babel}

% Set page size and margins
% Replace `letterpaper' with`a4paper' for UK/EU standard size


% Useful packages
\usepackage{tabularx}
\usepackage{enumitem,amssymb}
\usepackage{amsmath}
\usepackage{graphicx}
\usepackage{tikz}
\usepackage{tcolorbox}
\usepackage[margin=0.25in]{geometry}

\usepackage[colorlinks=true, allcolors=blue]{hyperref}

\begin{document}
\footnotetext[1]{pas besoin de noter tout ce que vous avez fait; au début du processus de relecture, ça risque d'être trop long et pas efficace en termes de temps}
\section*{Relecture}
\textit{Après une relecture, merci de compléter ce document comme décrit ci-après:}
\begin{itemize}
    \item Vous gagnez accès à la modification du document en tant que relecteur (particulièrement important pour la relecture grammaticale, comme l'ont vu nos relecteurs du proof of concept Leila et Christelle)
    \item Vous choisissez et relisez une partie sur laquelle peu de gens sont passés
    \item \textit{Vous écrivez ici} (voir dès la ligne 50 du fichier `Relecteurs.tex`)
\end{itemize}
N'oubliez pas de \textbf{Faire un retour à la ligne après ce que vous écrivez dans le tableau}\\
\boxed{\mbox{\textbf{Ne mettez un commentaire que si c'est nécessaire,}}} pour éviter une récursion infinie de vérification de la vérification de la vérification de la vérification de la vérification...

\begin{table}[h]
\begin{tabular}{l|l|l|l|}
\cline{2-4}
                                                                   Section relue vs Type de relecture& \multicolumn{1}{c|}{\begin{tabular}[c]{@{}c@{}}Français\\ name+endline\end{tabular}} & \multicolumn{1}{c|}{\begin{tabular}[c]{@{}c@{}}Mathématiques \\ name+endline\end{tabular}} & \multicolumn{1}{c|}{\begin{tabular}[c]{@{}c@{}}Commentaire (juste pour vérifier que tout va bien \footnote{pas besoin de noter tout ce que vous avez fait au début du processus de relecture, ça risque d'être trop long et pas efficace en termes de temps sinon})\\ (prénom+:+commentaire+endline)\end{tabular}} \\ \hline
                                            
                                            
                                            
                                            
                                            









%Début écriture des relecteurs                      
\multicolumn{1}{|l|}{1.1 Dérivées et intégrales}
&% français
\begin{tabular}[c]{@{}l@{}}
%Ecrire ici
Salim\\
Léo (Aryeth)\\
Josh\\
Thibault\\
\end{tabular}                        
&% maths 
\begin{tabular}[c]{@{}l@{}}
%Ecrire ici
Salim\\
Josh
\end{tabular}  
&% Commentaires  
\begin{tabular}[c]{@{}l@{}}
%Ecrire ici
Salim : "à une" $\to$ "à l'une" (p.3)\\

\end{tabular}  
\\\hline
\multicolumn{1}{|l|}{1.2 Produit scalaire et produit vectoriel}
&% français
\begin{tabular}[c]{@{}l@{}}
Léo (Aryeth)\\
Salim\\
Thibault\\
Josh\\
%Ecrire ici
\end{tabular}                        
&% maths 
\begin{tabular}[c]{@{}l@{}}
%Ecrire ici
Salim\\
Josh\\
\end{tabular}  
&% Commentaires  
\begin{tabular}[c]{@{}l@{}}
%Ecrire ici
Salim : "telle" $\to$ "tels" (1.2.1), $cos \to \cos$ (1.2.2)\\
\end{tabular}  
\\\hline
\multicolumn{1}{|l|}{1.3 Projections}   
&% français
\begin{tabular}[c]{@{}l@{}}
%Ecrire ici
Léo (Aryeth)\\
Salim\\
Thibault\\
Josh\\
\end{tabular}                        
&% maths 
\begin{tabular}[c]{@{}l@{}}
%Ecrire ici
Salim\\
Josh\\
\end{tabular}  
&% Commentaires  
\begin{tabular}[c]{@{}l@{}}
%Ecrire ici
Salim : x $\to$ $x$, r $\to$ $r$... \\
\end{tabular}  
\\\hline
\multicolumn{1}{|l|}{2.1 Approximations et développements limités}
&% français
\begin{tabular}[c]{@{}l@{}}
%Ecrire ici
Léo (Aryeth)\\
Salim\\
Thibault\\
Josh\\
\end{tabular}                        
&% maths 
\begin{tabular}[c]{@{}l@{}}
%Ecrire ici
Salim\\
Josh\\
\end{tabular}  
&% Commentaires  
\begin{tabular}[c]{@{}l@{}}
%Ecrire ici
Salim : parenthèse fermée, peut\textit{-il}, Et $\to$ Eh (intro 2.1), pente \textit{à}, développement\textit{s}, évalué\textit{e}\\
$= \to \approx$ dans DL $\cos$ et $\sin$ (pas de reste)\\
\end{tabular}  
\\\hline
\multicolumn{1}{|l|}{2.2 équations différentielles}
&% français
\begin{tabular}[c]{@{}l@{}}
%Ecrire ici
Léo (Aryeth)\\
Salim\\
Thibault\\
Josh\\
\end{tabular}                        
&% maths 
\begin{tabular}[c]{@{}l@{}}
%Ecrire ici
Salim\\
Josh\\
\end{tabular}  
&% Commentaires  
\begin{tabular}[c]{@{}l@{}}
%Ecrire ici
Salim : quel est le type $\to$ le type, de faire un petit exemple $\to$ de voir [...]\\ évaluation en $t$ = 0: $y(t) \to y(0)$\\
\end{tabular}  
\\\hline
\multicolumn{1}{|l|}{3.1 Référentiel et repère}
&% français
\begin{tabular}[c]{@{}l@{}}
Thibault\\
Salim\\
Josh\\
\end{tabular}   
&% maths 
\begin{tabular}[c]{@{}l@{}}
Salim\\
Josh\\
\end{tabular}  
&% Commentaires  
\begin{tabular}[c]{@{}l@{}}
Salim : qlqs pluriels\\
\end{tabular}
\\\hline
\multicolumn{1}{|l|}{3.2 Pré-requis}         
&% français
\begin{tabular}[c]{@{}l@{}}
Thibault\\
Salim\\
Josh
\end{tabular}                        
&% maths 
\begin{tabular}[c]{@{}l@{}}
% Ecrire ici
Salim\\
Josh\\
\end{tabular}    
&% Commentaires  
\begin{tabular}[c]{@{}l@{}}
% Ecrire ici
Salim : $m1 \to m_1$
\end{tabular}
\\\hline
\multicolumn{1}{|l|}{3.3 Lois de Newton}     
&% français
\begin{tabular}[c]{@{}l@{}}
Thibault\\
Salim\\
Josh\\
\end{tabular}                        
&% maths 
\begin{tabular}[c]{@{}l@{}}
% Ecrire ici
Salim\\
Josh\\
\end{tabular}    
&% Commentaires  
\begin{tabular}[c]{@{}l@{}}
% Ecrire ici
\end{tabular}
\\\hline
\multicolumn{1}{|l|}{3.4 Balistique}         
&% français
\begin{tabular}[c]{@{}l@{}}
Thibault\\
Salim\\
Josh\\
\end{tabular}                        
&% maths 
\begin{tabular}[c]{@{}l@{}}
% Ecrire ici
Salim\\
Josh\\
\end{tabular}    
&% Commentaires  
\begin{tabular}[c]{@{}l@{}}
% Ecrire ici
Salim : qlqs typos mineures dans les maths (ddot $\to$ dot)
\end{tabular}
\\\hline
\multicolumn{1}{|l|}{3.5 Méthodologie de résolution}
&% français
\begin{tabular}[c]{@{}l@{}}
Thibault\\
Salim\\
Josh\\
\end{tabular}                        
&% maths 
\begin{tabular}[c]{@{}l@{}}
% Ecrire ici
(pas de\\
maths ici)
\end{tabular}    
&% Commentaires  
\begin{tabular}[c]{@{}l@{}}
% Ecrire ici
Salim : small typos
\end{tabular}
\\\hline
\multicolumn{1}{|l|}{4.1 Coordonnées polaires}
&% français
\begin{tabular}[c]{@{}l@{}}
Yohan (Abhsr)\\
Thibault\\
\end{tabular}                        
&% maths 
\begin{tabular}[c]{@{}l@{}}
\end{tabular}    
&% Commentaires  
\begin{tabular}[c]{@{}l@{}}
% Ecrire ici
\end{tabular}
\\\hline
\multicolumn{1}{|l|}{4.2 Coordonnées cylindriques et sphériques}   
&% français
\begin{tabular}[c]{@{}l@{}}
% Ecrire ici
Yohan (Abhsr)\\
Thibault\\
\end{tabular}                        
&% maths 
\begin{tabular}[c]{@{}l@{}}
% Ecrire ici
Elias\\
\end{tabular}    
&% Commentaires  
\begin{tabular}[c]{@{}l@{}}
% Ecrire ici
Elias: ATTENTION: inverser theta et phi ! \\
\end{tabular}
\\\hline
\multicolumn{1}{|l|}{4.3 Oscillateurs harmoniques}              
&% Oscillateurs harmoniques - français
\begin{tabular}[c]{@{}l@{}}
Christelle\\
Leila\\
Thibault\\
\end{tabular}                        
&% Oscillateurs harmoniques - maths 
\begin{tabular}[c]{@{}l@{}}
% Ecrire ici
\end{tabular}    
&% Oscillateurs harmoniques - Commentaires  
\begin{tabular}[c]{@{}l@{}}
% Ecrire ici
\end{tabular}
\\\hline
\multicolumn{1}{|l|}{4.4 Exercice du pendule}
&% français
\begin{tabular}[c]{@{}l@{}}
Yohan (Abhsr)\\
Thibault\\
\end{tabular}                        
&% maths 
\begin{tabular}[c]{@{}l@{}}
% Ecrire ici
\end{tabular}    
&% Commentaires  
\begin{tabular}[c]{@{}l@{}}
Yohan : p22) "l'équation horaire de la masse" $\to$ "l'équation horaire du point M" \\
(ou sinon ne pas évoquer le terme "point", c'est confusing) 
\end{tabular}
\\\hline
\end{tabular}
\caption{Quel relecteur est passé par quelles parties ?}
\label{tab:my-table}
\end{table}

\section*{The big picture}
FYI : Voici l'ordre de priorité des choses à finir avant d'avoir officiellement fini la brochure\footnote{all hail the great Elias}:
\begin{enumerate}
    \item $\square$ Finir Ecriture
    \item $\square$ Finir relecture maths
    \item $\square$ Finir illustrations (dans l'ordre de grandeur d'une par page)
    \item $\square$ Finir mise en page
    \item $\square$ Finir relecture grammaire
\end{enumerate}

\end{document}
