\documentclass{article}

% Language setting
% Replace english' with e.g. spanish' to change the document language
\usepackage[english]{babel}

% Set page size and margins
% Replace letterpaper' witha4paper' for UK/EU standard size
\usepackage[letterpaper,top=2cm,bottom=2cm,left=3cm,right=3cm,marginparwidth=1.75cm]{geometry}

% Useful packages
\usepackage{amsmath}
\usepackage{graphicx}
\usepackage[colorlinks=true, allcolors=blue]{hyperref}

\title{Méthodes de résolutions d'exercices de mécanique newtonienne}
\author{Gaétan Kaouadji}

\begin{document}
\maketitle
\tableofcontents

\section{Introduction} 
Ce document a pour but de donner à tous une certaine méthode de résolution d'exercices de mécanique classique ainsi que des petits conseils pour vérifier vos résultats. J'espère qu'il pourra vous être utile lors de tout votre semestre et même au delà. 

\section{Énoncé}
\subsection{Lire l'énoncé}
Vous arrivez à votre table d'examen, près pour votre examen de physique. Le professeur donne le top départ et c'est parti, vous ouvrez le sujet! Vous vous retrouvez donc face à l'énoncé du premier exercice. C'est la première chose à analyser en profondeur car cela constitue l'unique ensemble de données dont vous allez devoir vous servir. On vous l'a répété des centaines de fois mais \textbf{lisez bien l'énoncé!!} Cela arrive encore à tous de louper des points importants. Petit conseil en plus, lisez l'énoncé jusqu'à la fin. Premièrement, certaines questions vous donneront des indications sur celles qui sont avant. Typiquement : \textit{"Dériver le résultat pour obtenu précédemment et vérifier qu'il s'écrit ..."} vous donne une indication sur le résultat à obtenir. De plus, des données sont parfois(voir souvent) écrites à la toute fin du sujet. En les lisant dès le début, ça vous évitera de vous arrachez les cheveux en disant que ce n'est pas logique, qu'il manque des données ! :) \\
\subsection{Les données}
Vous avez donc bien lu l'énoncé. Il s'agirait de pointer les données importantes. Premièrement, il faut repérer les constantes que l'on vous donne. Il y a les distances, les masses, les angles. Bien repérer quel objet a une masse $M$ et lequel a une masse $m$. Vous pouvez les surligner sur l'énoncé ou alors les rassembler dans un schéma! Le schéma est une des solutions les plus pratiques. Il peut vous être donné (notamment dans les situations compliquées comme des schémas en 3D) ou alors, à vous de le faire. C'est vraiment la base de la réflexion en mécanique parce que vous rassemblez toutes les données en un dessin et pouvez même comprendre ce qui va se passer juste en le regardant. Typiquement, si je vous parle d'une catapulte qui projette un boulet de masse $m$ sur un château et vous demande l'angle qu'il faut pour que le boulet touche le château, vous voyez bien qu'avec un schéma on comprend bien plus rapidement. \\
Deuxièmement, il faut repérer la situation dans laquelle on se trouve. Quelles forces sont négligées, lesquelles faut-il considérer? Typiquement, les frottements de l'air et du sol peuevent 

\section{Les calculs}
Maint







\end{document}