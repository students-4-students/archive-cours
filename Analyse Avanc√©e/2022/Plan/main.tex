\documentclass[11pt]{report}
\usepackage[utf8]{inputenc}
\usepackage[utf8]{inputenc}
\usepackage[T1]{fontenc}
\usepackage[french]{babel}

\usepackage{graphicx}


%  http://math.univ-lyon1.fr/irem/IMG/pdf/LatexPourLeProfDeMaths.pdf

%$\begin{array}{ccccc}
%f & : & E & \to & F \\
% & & x & \mapsto & f(x) \\
%\end{array}$
\usepackage{lmodern}
\usepackage{soul}
\usepackage{wrapfig}
% scinder en deux parties triangulaires de même aire une cellule initialement rectangulaire 
\usepackage{xcolor}
\usepackage{array}
\usepackage{tikz,tkz-tab}
\usepackage{lipsum} 
\usepackage[top=2cm, bottom=2cm, left=2cm, right=2cm]{geometry}
\usepackage{setspace}
 % pour insérer des images 
% Voici les commandes importants pour cadrer ton image ( ici mon image s'appelle Capture.jpeg j'ai déconné dans le nom mdr  
%   \includegraphics[scale=0.5]{Capture.jpg.jpeg} l'image est réduite de moitié                               \includegraphics[width=10cm]{Capture.jpg.jpeg} l'image est retaillée pour avoir une largeur de 10cm \includegraphics[height=10cm]{Capture.jpg.jpeg} l'image est retaillée pour avoir une hauteur de 10cm \includegraphics[angle=0]{Capture.jpg.jpeg} l'image est tournée de 0°

\usepackage{color}
\usepackage{colortbl}
\usepackage{multirow} % pour fusionner lignes tableaux
\usepackage{url} % pour les url lol
\usepackage{wrapfig}% pour countourner image 
% pour écrire des maths: 
%  “amsmath” “amssymb” et “mathrsfs”

\usepackage{amsmath}
\usepackage{amssymb}
\usepackage{mathrsfs}
\usepackage{makeidx}
\usepackage{amsfonts}



\usepackage{pgfplots}
\usepackage{pstricks}



\usepackage{hyperref}
\usepackage{pst-all}
\usepackage{pst-plot}
\usepackage{pstricks-add}

\usepackage{tikz}
\usepackage{mathrsfs}
\usepackage{xlop}

\title{S4S Analyse avancée }
\author{ }
\date{Printemps 2022}
\makeindex

\begin{document}

\layout
\maketitle
\newcommand{\bcks}[1]{\textbackslash{}}
\newcommand{\R}[2]{ $\mathbb{R}$ }





\\ 
 
 "\textbf{L'essence des mathématiques, c'est la liberté.}" \textbf{Georg Cantor}
 
 \paragraph{}
 


\paragraph{}
\paragraph{}


\renewcommand{\contentsname}{Sommaire} 
\setcounter{tocdepth}{1}
\tableofcontents
\chapter{Introduction Et Motivations}
\section{Présentation Du Programme Pour Les Mathématiciens et Physiciens}
\lipsum[1]
\section{Histoire De l'Analyse}
\lipsum[1]
\section{Les ensembles mathématiques $\mathbf{\mathbb{R},\mathbb{Z},\mathbb{N}, \text{ et }  \mathbb{Q}}$}
\subsection{\S \ L'ensemble $\mathbb{N}$ des nombres naturels}
\lipsum[1]
%On désigne l'ensemble $\{1,2,3,...\}$ de tous les nombres entiers positifs par $\mathbb{N}$. Chaque nombre entier positif $n$ a un successeur, à savoir $n+1$. Ainsi, le successeur de $2$ est $3$, et $37$ est le successeur de $36$. Vous conviendrez probablement que les propriétés suivantes de $\mathbb{N}$ sont évidentes; du moins les quatre premières le sont.
%\begin{enumerate}
 %   \item  $1$ appartient à $\mathbb{N}$.
%\item Si $n$ appartient à $\mathbb{N}$, alors son successeur $n+1$ appartient à $\mathbb{N}$. 
%\item $1$ n'est le successeur d'aucun élément de $\mathbb{N}$.
%\item Si $n$ et $m \in \mathbb{N}$ ont le même successeur, alors $n=m$.
\item Un sous-ensemble de $\mathbb{N}$ qui contient %1, et qui contient $n+1$ chaque fois qu'il contient %$n$, doit être égal à $\mathbb{N}$.
%\end{enumerate}
\subsection{ L'ensemble $\mathbb{R}$ des réelles (pour mathématicien, pas nécessairement présenté en amphi) }
\lipsum[1]
\section{Définition Et Propriété De La Valeur Absolue}
\lipsum[1]
\chapter{Introduction Aux Suites Numériques}
\section{Définition}
\lipsum[1]
\section{Les Suites Définis Par Récurrence}
\lipsum[1]
\section{Les Suites Arithmétiques}
\lipsum[1]
\section{Les Suites Géométriques}
\lipsum[1]

\lipsum[1]
\chapter{Notions De Convergence}
\section{$\epsilon$ et $\delta$}
\lipsum[1]
\section{Une Première Approche de La Notion De Convergence}
\lipsum[1]
\subsection{Exemples}
\lipsum[1]
\section{Le Théorème Des Gendarmes}
\lipsum[1]
\section{Les suites de Cauchy}
\lipsum[1]
\section{Concept De $\mathbf{\lim \sup}$ Et $\mathbf{\lim \inf}$}
\lipsum[1]
\chapter{Séries}
\section{Propriétés Des Sommes}
\lipsum[1]
\section{Sommes Arithmétiques Et Géométriques}
\lipsum[1]
\section{Démonstration Par Récurrence Sur Les Sommes}
\lipsum[1]
%==========================================================================
%==========================================================================
%==========================================================================
%==========================================================================
%==========================================================================
%==========================================================================
%==========================================================================
%==========================================================================%==========================================================================
%==========================================================================
%==========================================================================
%==========================================================================
%==========================================================================
%==========================================================================
%==========================================================================
%==========================================================================
%==========================================================================
%==========================================================================
%==========================================================================
%==========================================================================%==========================================================================
%==========================================================================
%==========================================================================
%==========================================================================
%==========================================================================
%==========================================================================
%==========================================================================
%==========================================================================
%==========================================================================
%==========================================================================
%==========================================================================
%==========================================================================%==========================================================================
%==========================================================================
%==========================================================================
%==========================================================================
%==========================================================================
%==========================================================================
%==========================================================================
%==========================================================================
%==========================================================================
%==========================================================================
%==========================================================================
%==========================================================================%==========================================================================
%==========================================================================
%==========================================================================
%==========================================================================

\end{document}
